\chapter{Information Gathering}

Research and previous work have been well examined to find out what has already been done within this field. Based on our foundational knowledge we developed different business models where we looked at four different customer segments, namely \emph{the end user (elderly}, \emph{training groups}, \emph{community centres} and \emph{physiotherapists (and the health sector in general)}. During our work on the business model we discovered many loose ends, and at one point we could discard some of our theories. We ended up with physiotherapists as our main customer segment.\\ \\
In this section we will describe the methods we have used to gather information. We will then provide a discussion of the information relevant for the analysis of the business model for this product. 

\section{Qualitative Research}
Qualitative research means to get an in-depth understanding of a phenomenon \cite{interview2}. We have performed this type of research with regard to prevention and rehabilitation of elderly and of the possible use of an exergame. After choosing a customer segment, we used interviews as a method for information gathering.  \\ \\
We interviewed three physiotherapists working with the older patient group. In addition we had a conversation with one physiotherapist by email . Our supervisor helped us get in contact with one physiotherapist, who again put us in contact with other possible interview objects. In addition to that, we contacted some physiotherapists ourselves. We choose to look at both clinics controlled by the government and also one private clinic. The reason why we did this, is because the two entities have two quite different economic models. In the private clinic, the owners are the payers, while in the public clinics, the government is the payer.

\subsection{Interview Methods}
There are different types of qualitative interview methods: \\ 
1. \emph{Structured Interview:} The main topic for the interview is decided and a complete interview guide is prepared beforehand. \\ 
2. \emph{Unstructured Interview:} This is a very flexible method where the topic is decided, but there is usually no interview guide. This allows the interviewer to improvise suitable questions during the interview. \\ 
3. \emph{Semi-structured Interview:} This is a mix between method 1 and 2. The interviewer has an interview guide with some prepared questions, but these questions serve more as guidelines, and allows the interviewer to improvise suitable questions during the interview. This is the most commonly used interview method, and is often called "qualitative interview". \cite{interview} \cite{interview2} \\ \\
We used semi-structured interview for several reasons. First, all of our interviewees where physiotherapists and we had some specific questions about their routines directed towards the business model we were developing. However, since they were all working in different clinics the questions had to be adapted towards their field of expertise, and it was therefore room for improvisation. Second, since neither of us are professionals in this field, it would be impossible for us to foresee everything that should be asked about. Last, we wanted to make the interview as a natural conversation, without locking ourselves to specific questions. This allowed the conversation to flow more naturally, providing us with some unexpected information that we did not think of beforehand. \\ \\
In accordance to the normal structure of an interview \cite{interview2}, we started with an introduction, telling about who we are, about our project, and the goal for the interview. We then followed up with some basic questions about the interviewee, like name, age, work and education. In this way, we got to know each other better before the questioning started. We had two different main topics we wanted to discuss. We wanted to learn about how they work and what kind of relationship they have with their patients. This was to identify if there was a need and also how the product could fit into their working situation. We also had some questions more directed towards the business model, like how they got to know about products and how they acquired them. In each topic, we had some defined questions to guide us, but not to limit us. There was room for improvisation at all times. We modified the interview guide between the interviews based on the experiences we gained.

\subsection{Possible Pitfalls}
When doing a qualitative interview there can always occur some unexpected problems or difficulties. We will go through some of the limitations that may have affected our interviews, based on a list of possible pitfalls we found in \cite{interview}:\\
\emph{Artificiality of the interview:} All our interviewees were strangers to us. The interviewee was asked to answer questions  and give opinions under time pressure. This might have made the interview artificial. \\
\emph{Lack of trust:} Because we did not know the interviewees, them not trusting us could have been an issue. This means that the interviewees may have held back what they think of as sensitive information. This information may have been important information for us, and a possible holdback of this information would make the gathering incomplete. \\
\emph{Lack of time:} In one of our interviews, we had a time-limit. Whether this had a positive or negative effect is hard to say. Time limit can result in an incomplete data gathering, but also lead to the opposite where the interviewee creates opinions under time pressure which can result in more data, but possibly less reliable data. In our case, we do not feel that time constraints resulted in any unreliable data. \\
\emph{Constructing Knowledge:} Interviewees may not have reflected over the questions asked during the interview before. Maybe they do not know that much about the topic, and therefore construct a story that is consistent to appear knowledgeable. This is hard to say if have been a problem in our case, but we believe that all of our interviewees had sufficient knowledge on all of the topics. In addition the answers from the different interviewees where consistent to each other. \\
\emph{Ambiguity of language:} Sometimes a meaning of a word can be ambiguous. Both the interviewer and the interviewee can misunderstand the meaning. Since our knowledge within this field was limited, it appeared some misunderstandings during the interview. The misunderstandings were discovered when the interviewee read through the interview report, and were fixed at the same time. In this way it did not affect our final report.  



\chapter{Discussion}
This chapter will provide a discussion on our work and our thoughts around the business model. The emphasis will be on the aspects that was considered, but not included in the business model. We will discuss in detail the reasons for the choices we made.  \\ \\
In Chapter 8.2.1 we recommend Cyberlab to focus on public clinics and private clinics with contribution from the government. Remaining private clinics is not at the moment considered as a target customer in the business model. These clinics have other decision making processes, different economic aspects and often a more limited customer base, compared to public physiotherapy clinics. With no support from the government these clinics might have a more restricted economy, which may complicates purchase of new products. This was also confirmed in the interview with the private clinic. As mentioned, private clinics do not have the same access to customers as public clinics, and one reason for this is that customers who consult private clinic have to pay for the therapy session themselves. Elderly may not have the money or willingness to pay for a therapy session at private physiotherapy clinics (see Appendix C). We believe it can be more challenging to reach elderly through private clinics and therefore we did not include this entity in the business model. However, we believe that a connection with private clinics might arise naturally, as the marketing towards the public arena also will reach the private arena (e.g. through the magazine "Fysioterapeuten"). In the interview with the physiotherapist working in the private clinic, it was mentioned that it would be of interest to use an exergame in training groups. In addition, it was also mentioned that it could be interesting to have "special" products to attract more customers. However, this will not have any influence if "every" other clinic also offers this product.\\ \\
The end users of the exergame are the older population. We also considered this group as a customer segment. The idea would be to sell the game directly to elderly so they can use this exergame at home. The fair of falling makes many elderly afraid of walking outside their own house. Therefore, it would be appealing for them to have the possibility to exercise at home. In order to become physically stronger, there is a need for a supplement to the weekly physiotherapy appointments or training group sessions. Regular workout at home can strengthen muscles and increase balance.  This may contribute to prevention of falling and increased self-confidence. However, after working with and studied the business model with this group as a customer segment, we experienced that they are not the proper target for Cyberlab at this point. Some reasons for this are worth mentioning. Most elderly do not have experience with computer games or video games. Even though they might be used to technology like television and mobile phones, for them to adopt a new technology like video games might be very difficult. Many have never seen or used video games before, and they might not see the purpose of using a game like this. A developer trying to sell an exergame by presenting the products value proposition will not be sufficient to convince an elderly to use their product. We believe that it will be very difficult for Cyberlab to reach directly out to this customer segment. This might be more realistic in some years, when the exergame has been used at physiotherapy clinics and elderly has experienced the game with assistance in a safe environment. In addition, will the next generation of elderly, be more familiar with this kind of technology.  \\ \\
Other arenas for this game, which we did not take into account in the business model, are the different training groups offered, separate from the ones offered by the physiotherapy clinics. We believe that after reaching the physiotherapy clinics, the way to these training groups will not be long. The participants pay a small fee to attend these groups. Rising this price a small amount, can cover the usage fee for the game. These training groups are offered by the government, organizations and individuals. During these workout sessions this exergame can be used as a supplement or a different alternative to ordinary exercise. Playing the exercise together with other elderly makes the game social and entertaining. \\ \\
The last possible customer segment could be community centres. With this game they can provide the elderly with an alternative activity compared to chess, card play or taking a walk. One possibility for the community centres is to rent the game to their patients, so the patients can use the equipment for a certain amount of time, alone or in a group. This might also be more relevant in the future. \\ \\
The discussion around the other possible customer segments supports our choice of customer segment. Physiotherapists have some kind of relationship with these entities, and can serve as a bridge to reach them. It could be possible for Cyberlab to expand their potential market by including one or more of these entities as customer segments. The exergame that Cyberlab will develop is a part of an EU-funded project. This suggests that the game can be sold to entities also outside Norway.  \\ \\
The value propositions mentioned in Chapter 8.1.1 includes physiotherapists' great advantages of owning an exergame, but after talking with physiotherapists and gathering knowledge about the new, reform, "Samhandlingsreformen", we will recommend Cyberlab to focus on another value proposition, as well. The game is not initially meant to have the ability for multiplay. However, we see this as an feature that would have value both for the end user and the physiotherapist. As already emphasized, the social aspects of the game are important for elderly. Playing together with more people can be an important social factor for this game. For the physiotherapists this will have the advantage of serving more than one player at the same time. There are often long waiting lists to get an appointment with a physiotherapist, so an important aspect for them is to be efficient to serve as many patients as possible, without losing quality in the work done. With a multi-player feature, the exergame can be used to ease physiotherapists' workload even more than with the initial game. Physiotherapists can then consult and exercise with more patients in one hour than a they can manage without the exergame. This will be valuable in the work of shortening the long waiting lists. \\ \\
In Chapter 8.2.2 we discuss how Cyberlab can reach their customers. In addition to the different channels we described, physiotherapists mention "NAV Hjelpemiddelsentralen" as a channel for getting to know about existing health-related equipment. "NAV Hjelpemiddelsentral" is a Norwegian entity with competence in the area of equipment for health-related purposes. Their responsibility is to have expertise about existing equipment, and communicate and facilitate for people who need help. They should know about different disabilities and provide customized solutions to help them. There exist different categories of equipment, where technical equipment is one of them. For a tool to be categorized as technical equipment is has to be helpful and necessary for increasing users' physical abilities. This imply; being a tool that can help a functionally disabled person to better manage everyday activities. The idea is that people with disabilities can apply for financial support from the government for buying this type of equipment. As Cyberlab’s exergame has the intention of increase physical strength and promote physical activity, and not qualifies as absolutely needed equipment, it will most likely not be categorized as technological equipment \cite{navhjelpemiddel} \cite{navhjelpemiddelsentralen}.  \\ \\
We have suggested that Cyberlab shall start focusing on delivering the game to clinics in Trondheim. Selling the game in only Trondheim to an acceptable price will not make profit. With our assumptions, Cyberlab is looking at some years before they will make profit, see Chapter 8.4.3. The physiotherapy community is small and they have a lot of similarities. They may attend the same conferences and they read the same magazines. By the time this game has become well-known in Trondheim, the game have most likely gotten attention from clinics in other cities as well. To overcome the difficulties of travel around the country to market the product, Cyberlab could consider hiring some of the physiotherapists that have gained experience with the game to promote and teach the game at for example conferences.\\ \\
In Chapter 8.4.2 we propose, describe and analyse two possible revenue models for Cyberlab, a fixed price model and usage fee model. We found the usage fee model to be the most profitable for Cyberlab. With this model, the game has to be monitored to be able to calculate how much time the game has been in use. This should be remotely controlled. In addition, the goal in the long run for this game is that patients eventually can start using this game at home as a supplement to their own regular treatment. This will also require remote control because in some cases the physiotherapists should be able to follow their patients’ progression. A system like this will most likely contain some private information of the user. To be able to customize the game, it will require a profile for each user, with all the information needed. This profile will also serve as a way for the physiotherapist to keep remotely track on the patient’s improvements and to adjust the program accordingly. It is important to acknowledge some of the issues introduced with the use of technology together with health care. The user profile will probably be provided on a webpage that can be retrieved from a remote server. Anything on the web always face some security issues, and we could spend a lot of time discussing all the different threats and vulnerabilities. However, this is out of the scope of this project and is rather a subject for future work. \\ \\ 
In addition to the two pricing models we analysed in Chapter 8.4.2, we considered two other pricing models. We will briefly describe them here, for Cyberlab to become aware of these options, and we will explain why we did not include them in the business model. A possible revenue model we studied was a license agreement, where Cyberlab could generate revenue by selling permissions to physiotherapists in exchange for a license fee. This would allow physiotherapists to generate revenue by reselling the exergame to elderly. Elderly trust physiotherapists and their products, and would therefore be a good channel for bringing the exergame home to elderly. However, we decided to not look deeper into this revenue model for the same reasons as for why elderly not is a suitable customer segment for Cyberlab. However, it could be something to consider in the future. The last revenue model we looked into was a game financed by fees for advertising. We also found this not to be appropriate for this game. Feedback from the interviews stated that a video game for elderly has to easy to use and very intuitive, with no disruptive elements. A video game with ads will not appear user friendly as it mos likely will be disturbing and maybe remove the focus from the game. We neither do believe this model will ever be an option for Cyberlab. \\ \\
One of the reasons for choosing Microsoft Kinect as the technology for this game is the free, open SDK, that enables a third party to develop games for the platform. What we have not been able to find out, is what rights developers have to sell this game with a profit. Since Microsoft is the owner of the Kinect sensor for Windows and the SDK, Cyberlab might have to form an agreement on how they can sell their hardware and software with their label on. If this is a requirement, Microsoft will also serve as a key partner for Cyberlab. \\ \\
Physiotherapists are Cyberlab's customer segment, but they can also serve as a partner. This game fits very well into where the Norwegian health system is going towards (see "Samhandlingsreformen" Chapter 3). We see a potential for this game to work as a tool for "everyday rehabilitation". A goal for Cyberlab should be to get this game in as a "regular" tool for prevention and rehabilitation. The government is responsible for the health sector, so they could be a natural partner for the future. Having the government as a partner will solve most of the financial issues. Getting this game into a medical program financed by the government will make this game very credible for the end user. The government could then provide the game to hospitals, care centers, physiotherapy clinics, training groups, and for special cases, when the elderly for example need the game in their own house. \\ \\ 
From the interviews we experienced that there is a demand for receiving a product package with services like delivery, installation and introduction. Our first thought was therefore to include these services in all the packages, and add salaries for installation and expenses related to travel in the package price. After studying this solution in more detail we saw that this could be very expensive for the customers. The price would also vary in accordance to distance and help needed. In addition, it was not possible for us to calculate reasonable numbers related to this cost, and it has therefore not been taken in to consideration in our analysis. However, we recommend that Cyberlab should offer the possibility to deliver this kind of services as there is a demand for it. \\ \\   
We have made some additional assumptions when it comes to what should be included in the package Cyberlab is going to sell to their customers. In our calculations we have included the Kinect sensor only, but ideally, they should offer a package with a Windows machine and a screen as well. If these packages were offered, Cyberlab would have to buy the extra equipment. If they could buy the equipment in big quantities, they could buy them from a wholesaler to cost price, and sell with profit. However, this would require Cyberlab to have a huge stock. To keep a stock is both risky and inconvenient. Having a number of TV-sets and computers stored in their premises will be very space consuming. In addition, it is risky in the sense of not knowing how much they will sell. Also, there might be new versions and upgrades that will make the "old" versions unattractive. This especially accounts for the Kinect sensor. Having a stock will also mean that Cyberlab needs to have the liquidity to be able to buy the quantities, because they will not sell everything right away. In addition, this means that Cyberlab has to predict how many customers they will sell to during a fixed period of time, which is very hard to do. \\ \\
Another prediction we have done in our calculations is how much workload is required for each task. It is hard to predict how many bugs will occur in software and therefore it is impossible to know how many hours are spent on this per year. To make these predictions more precise, we would have needed more experience with software developing. We assume Cyberlab has this kind of experience, and will probably be more suited to make this kind of predictions.\\ \\
The workload assigned the different tasks do not always equal full-time. For example a FTE=1.0 can mean that one person is assigned the project 100 percent for one year, it can mean that four people are assigned the project 100 percent for three months, or it can mean that four people is assigned this project for one year, but only with 25 percent workload each. We have not specified how many people will work on each task in this project. It will be up to Cyberlab to decide what is most efficient for them. The marketing task, is only assigned a FTE= 1/2 in the first year after the game’s release and FTE=1/5 the remaining four years. It Cyberlab does not already have a marketing person; they can consider to hire a marketing consultant to do this job. \\ \\
For us to be able to make an in-depth financial analysis we had to do many assumptions when it comes to demand and potential market. There is high uncertainty related to demand and to the market potential of this product. Since the exergame has not been developed yet it is very difficult to predict how successful it can become. Neither does it exist similar products on the market that we compare the game with. This makes it impossible to foresee how this new product will be accepted in the market. In the revenue analysis we have estimated that there exists a potential market of 1 200 physiotherapy clinics, where Cyberlab is likely to capture 33 percent of this share. This is explained in Chapter 8.4.3. However, this assumption is just a rough estimate done by us and is not based on any facts. This means that this not necessarily is a realistic estimate of the size of the potential market. \\ \\
Cyberlab could also expand their potential market by including other entities as possible customer segments. One example is for Cyberlab to include private physiotherapy clinics without any economic support from the government. These clinics have shown interest in this product, which makes them potential target customers.  After some time it could also be possible for Cyberlab to expand their market and start selling the product to end users, the elderly themselves. This will increase the market potential significantly. However, we believe this could first be done when the product has been on the market for a while. Physiotherapists have then had the time to work with the product and elderly has gotten to use the exergame with assistance in a safe environment. The package price for the end user has to become drastically lower, and with a greater market potential Cyberlab will have the opportunity to sell their products to a lower and more affordable price for a "normal" buyer. The exergame that Cyberlab will develop is a part of an EU-funded project. This suggests that the game can be sold to entities also outside Norway. However, this will complicate the financial analysis, as we have to include the other participants in this project. \\ \\
Our analysis is completely based on the assumptions that the exergame has been developed, well documented, and that there is a great demand for this type of game. These assumptions will not be reliable for Cyberlab when they are going to sell their game. However, we feel that our financial analysis is well-grounded and highly thought-through. Our cost analysis is based on existing numbers, and our revenue examples have wide range. We provide several possible solutions and price examples, and we analyse worst-case scenarios.  

Et lite avsluttende avsnitt her.



\chapter{Discussion}
IKKE VITS Å LESE. HER SKAL DET KOMME DISKUSJON. SIKKERHET INNE I DISKUSJON. 
\section{Security aspects}
There are some issues regarding the security of personal information associated to this game. When a clinic starts using this game, they will make a user profile for each patient. This profile will presumably contain name, age, physical and mental conditions, diagnosis, the patients exercise program and the history of games played. It might also contain other type of information, like home address, phone number, relatives etc. The purpose of this record is for the physiotherapist to easily keep track of the patients improvement and to be able to adjust a program in an efficient way. \\ \\
The goal in the long run for this game is that patients eventually can start using this game at home as a supplement to their regular treatment. It will then be possible for the physiotherapist treating these patient, to log on to and look at the patient’s profile at any time. This profile will probably be a webpage where the patient (or user) and the physiotherapist treating this patient can log in. This webpage will be retrieved from a remote server hosted by either Cyberlab or in the cloud. \\ \\
Anything on the web always face some security issues, and we could spend a lot of time discussing all the different treats and vulnerabilities. However, this is out of scope of this project, so we will only discuss some aspects to keep in mind for these user profiles.\\ \\
SSL should be used for secure communication between the server and the client. This prevents eavesdropping, tampering and message forgery. (siter securityboka).It is important to know that the right person with permission is getting access to the information. This require proper login and authentication. However, something that we can ask ourselves is how sensitive the information contained on the web page really is? Of course, no one would like anyone to read their health history. Except from that you can really get the other information anywhere else on the web. If an intruder steels someone's login information, they will get full access to the profile and can read and alter all the information.  \\ \\
One way of not linking the actual patient to the information, is to use patient numbers. This requires the physiotherapists to have their own database with patient names linked to patient numbers. We assume they already have this. However, this will make it more inconvenient for the physiotherapists to use, because they will have to look up every patient number and link it to the name themselves. It is very important for the physiotherapists that the information will be given to them in an easy and efficient way, so this solution will not hold. \\ \\
What harm will a possible information leakage do? First, if an attacker get access to the data and reads it, they can get some interesting information. For example they can read a medical description that says that the patient is in a wheelchair. If in addition they can get the address from the patients profile, this house can be a target for a robbery. ...Ikke ferdig.. (Første utkast)

\section{NAV Hjelpemiddelsentralen}
Hei

\section{Discussion}
Fra customer segment: In the starting phase we will recommend Cyberlab to focus on public clinics and private clinics with economic support from the government ("driftstilskudd"). Remaining private clinics will at the moment not be considered as a target customer in this business model. These clinics have other decision making processes, different economy aspects and often a more limited customer base compared to public physiotherapy clinics. With no support from the government these clinics might have a more restricted economy, which complicates purchase of new products. As mentioned, private clinics do not have the same access to customers as public clinics. One reason for this is that customers who consult private clinic have to pay for the therapy session themselves. Elderly may not have that kind of money or willingness to pay for a therapy session at a private physiotherapy clinic (see Appendix, mail fra Nina). Since elderly are the end user of this product, and this group will be challenging to reach through private institutes, we will recommend Cyberlab to not focus on this entity. Some reasons are worth mentioning. Elderly today usually do not have any connection to video games or technology at all. So a developer trying to sell this exercise game by presenting this products value proposition to a person who do not have any knowledge about this area, will be meaningless. The number of technological devices existing today are endless. It will be very difficult for Cyberlab to reach directly out to this customer segment. A solution for Cyberlab to reach their end users is to use someone elderly trusts and rely on. As many others, elderly relies on authorities. Physiotherapists are an example of that kind of authority, so if a physiotherapist tells an elderly that "this product will be good for you", they will most likely believe them. (Legge til eldrehjem, at eldre kan bli brukergruppe senere?). The customer will not always be the end user, and it is important to recognize and pay attention to this difference. For this business model physiotherapy clinics will be Cyberlab's customers, while elderly are the end users. For a physiotherapy clinic, elderly will be both the customer and the end user. (Komme med et eksempel?). A satisfied end user is important for the customer, although they are not the same person. In the beginning we were thinking about elderly as a target customer segment for Cyberlab as they are the end users of the product, but after working with and studied this business model we experienced that elderly is not the proper target for Cyberlab right now. \\ \\
Fra channels - awareness: Through conferences, magazines or just word of mouth one can also get an impression of which products that are popular right now. The popularity mark might attract some customers. Physiotherapists working at private clinics do not have the same access to customers as public clinics have, so they might chose to buy a popular product to attract new and more customers.  
\\ \\ 
(Say something about other possible revenue models we have decided not to focus on, like a license agreement revenue model or a advertisement revenue model.) \\ \\
Fra cost:
In the first year, we suggest that Cyberlab focus on delivering the game to clinics in Trondheim. Selling the game in only Trondheim to an acceptable price, will not make profit. We believe that Cyberlab is looking at some years before they will make profit. The physiotherapy community is small and they have a lot of similarities. By the time this game has become well-known in Trondheim, the game should have gotten attention from clinics in other cities as well. We suggest that Cyberlab hire some of the physiotherapists that are already using the game to present it on conferences and teach other physiotherapist how to use it. In this way, the game will be both more believable to other physiotherapists, as well as Cyberlab do not need to do the hazel with travel around promoting their game. There is still some costs associated to this.\\ \\
Fra revenue - about not including installation prices in our revenue discussion:  or wages for installation and introduction. In this section we will not take wages for installation and introduction into account, as it is very difficult for us to calculate all costs connected to this.
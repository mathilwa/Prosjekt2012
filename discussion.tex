\chapter{Discussion}
This chapter will provide a discussion on our work and our thoughts around the business model. The emphasis will be on the aspects considered that are not included in the business model, and we will discuss and state the reasons for the choices we have made.  \\ \\
In chapter 7, we describe and study four possible customer segments, where we recommend Cyberlab to focus on public clinics and private clinics with contribution from the government. Remaining private clinics will at the moment not be considered as a target customer in the business model. These clinics have other decision making processes, different economy aspects and often a more limited customer base compared to public physiotherapy clinics. With no support from the government these clinics might have a more restricted economy, which complicates purchase of new products. As mentioned, private clinics do not have the same access to customers as public clinics, and one reason for this is that customers who consult private clinic have to pay for the therapy session themselves. Elderly may not have that kind of money or willingness to pay for a therapy session at private physiotherapy clinics (Appendix, mail fra Nina). We believe elderly will be challenging to reach through private clinics, so we recommend Cyberlab not to focus on this entity. \\ \\
We were thinking about elderly as a target customer segment for Cyberlab as they are the end users of the product. However, after working with and studied the business model with this group as a customer segment, we experienced that they are not the proper target for Cyberlab right now. Some reasons for this are worth mentioning. Elderly today usually do not have any connection to computer games or video games. Even though they might be used to technology like television and mobile phones, for them to adopt a new technology like video games will be very difficult. Many have never seen or used video games before, and they might not see the purpose of using a game like this. A developer trying to sell an exergame by presenting the products value proposition will not be sufficient to convince an elderly to use their product. We believe that it will be very difficult for Cyberlab to reach directly out to this customer segment. This might be more realistic in some years, when the exergame has been used at physiotherapy clinics and elderly has experienced the game with assistance in a safe environment.\\ \\
With the usage fee pricing model, the game has to be monitored to be able to calculate how much time the game has been in use. This has to be remotely controlled. In addition, the goal in the long run for this game is that patients eventually can start using this game at home as a supplement to their own regular treatment. This will also require remote control because in some cases the physiotherapists should be able to follow their patients’ progression. A system like this will most likely contain some private information of the user. To be able to customize the game, it will require a profile for each user, with all the information needed. This profile will also serve as a way for the physiotherapist to keep remotely track on the patient’s improvements and to adjust the program accordingly. It is important to acknowledge some of the issues introduced with the use of technology together with health care. The user profile will probably be provided on a webpage that can be retrieved from a remote server. Anything on the web always face some security issues, and we could spend a lot of time discussing all the different threats and vulnerabilities. However, this is out of the scope of this project and is rather a subject for future work. \\ \\ 
The value propositions mentioned in chapter 7 provide physiotherapists great advantages of owning an exergame, but after talking with physiotherapists and gathering knowledge about the new “Samhandlingsreformen”, we will recommend Cyberlab to focus on another value proposition. This will be to include the ability to be a multiplayer game, which makes it possible for physiotherapists to serve more than one player at the same time. To get a consultation hour at a physiotherapy clinic there is often long waiting lists, so an important aspect for physiotherapists is to be efficient to serve as many patients as possible, without losing quality in the work done. With a multiplayer feature, the exergame can be used to ease the physiotherapists’ workload. A physiotherapist can then consult and exercise with more patients in one hour than a physiotherapist can manage without the exergame. This will be valuable in the work of shortening the long waiting lists. \\ \\
In addition to the channels mentioned for raising awareness about the product, physiotherapists mention "NAV Hjelpemiddelsentralen" as a channel for knowing about existing health-related equipment. "NAV Hjelpemiddelsentral" is a Norwegian entity with competence in the area of equipment for health-related purposes. Their responsibility is to have expertise about existing equipment, communicate and facilitate for people who need help. They should know about different disabilities and provide customized solutions to help them. There exist different categories of equipment, where technical equipment is one of them. For a tool to be categorized as technical equipment is has to be helpful and necessary for increasing a user’s physical abilities. This imply; being a tool that can help a functionally disabled person to better manage everyday activities. The idea is that people with disabilities can apply for financial support from the government for buying this type of equipment. As Cyberlab’s exergame has the intention of increase physical strength and promote physical activity, and not qualifies as absolutely needed equipment, it will most likely not be categorized as technological equipment \cite{navhjelpemiddel} \cite{navhjelpemiddelsentralen}.  \\ \\ 
Through conferences, magazines or just word of mouth physiotherapists can also get an impression of which products that are popular right now. The popularity mark might attract customers. Physiotherapists working at private clinics, with a more limited access to customers than public clinics has, might chose to buy a popular product to attract new and more customers. This will not have any functionality if every clinic owns the product, and it will neither be relevant for Cyberlab’s customer segment, as they do not need to attract customers. \\ \\ 
We have suggested that Cyberlab shall start with focusing on delivering the game to clinics in Trondheim. Selling the game in only Trondheim to an acceptable price will not make profit. We believe that Cyberlab is looking at some years before they will make profit. The physiotherapy community is small and they have a lot of similarities. By the time this game has become well-known in Trondheim, the game should have gotten attention from clinics in other cities as well. We suggest that Cyberlab hire some of the physiotherapists that are already using the game to present it on conferences and teach other physiotherapist how to use it. In this way, the game will be both more believable to other physiotherapists, as well as Cyberlab do not need to do the hazel with travel around promoting their game. \\ \\
In chapter 7 we propose, describe and analyse two possible revenue models for Cyberlab, a fixed price model and usage fee model. We will mention two other revenue models we have decided not to focus on, because we do not feel they were appropriate for Cyberlab in this business model. \\ \\
\emph{Licensing:} \\ A possible revenue model we studied was a license agreement, where Cyberlab could generate revenue by selling permissions to physiotherapists in exchange for a license fee. This will allow physiotherapist to generate revenue by resale of the exergame to elderly. Elderly trust physiotherapists and their products, and would therefore be a good channel for bringing the exergame home to elderly. However, we decided to not look deeper into this revenue model for the same reasons as for why elderly not is a suitable customer segment for Cyberlab.\\ \\
\emph{Advertising:}\\ Another revenue model we took into consideration was a game financed by fees for advertising. As with the license agreement, we experienced that this not is an appropriate option with regard to the end users of the product. Feedback from the interviews stated that a video game for elderly has to be as easy to use and as intuitive as possible, with no disruptive elements. A video game with ads will not appear user friendly as is would be confusing and maybe remove focus from the game.\\ \\
From our interviews we experienced that there is a demand for receiving a product package with services like delivery, installation and introduction. Our first thought was therefore to include these services in all the packages, and add wages for installation and expenses related to travel in the package price. After studying this solution in more detail we saw that this could be very expensive for the customers. The price would also vary in accordance to distance and help needed. In addition, it was not possible for us to calculate some reasonable numbers related to this cost, and it has therefore not been taken in to consideration in our analysis. However, we recommend that Cyberlab should offer the possibility to deliver this kind of services as there is a demand for it. \\ \\   
We have made some additional assumptions when it comes to what should be included in the package Cyberlab is going to sell to their customers. In our calculations we have included the Kinect sensor only, but ideally, they should offer a package with a Windows machine and a screen as well. If these packages were offered, Cyberlab would have to buy the extra equipment. If they could buy the equipment in big quantities, they could buy them from a wholesaler to cost price, and sell with profit. However, this would require Cyberlab to have a huge stock. To keep a stock is both risky and inconvenient. Having a number of TV-sets and computers stored in their premises will be very space consuming. In addition, it is risky in the sense of not knowing how much they will sell. Also, there might be new versions and upgrades that will make the "old" versions unattractive. This especially accounts for the Kinect sensor. Having a stock will also mean that Cyberlab needs to have the liquidity to be able to buy the quantities, because they will not sell it right away. In addition, this means that Cyberlab has to predict how many customers they will sell to during a fixed period of time, which there is very hard to do. \\ \\
Another prediction we have done in our calculations is how much workload is required for each task. It is hard to predict how many errors and bugs will occur in software and therefore it is impossible to know how many hours are spent on this per year. To make these predictions more precise, we would have needed more experience with software developing. We assume Cyberlab has this kind of experience, and will probably be more suited to make this kind of predictions.\\ \\
The workload assigned the different tasks do not always equal full-time. For example a FTE=1.0 can mean that one person is assigned the project 100 percent for one year, it can mean that four people are assigned the project 100 percent for three months, or it can mean that four people is assigned this project for one year, but only with 25 percent workload each. We have not specified how many people will work on each task in this project. It will be up to Cyberlab to decide what is most efficient for them. The marketing task, is only assigned a FTE= 0.5 in the first year after the game’s release and FTE=0.2 the remaining four years. It Cyberlab does not already have a marketing person; they can consider to hire a marketing consultant to do this job. \\ \\
One of the reasons of choosing Microsoft Kinect as the technology for this game is the free, open SDK, that enables a third party to develop games for the platform. What we have not been able to find out, is what rights developers have to sell this game with a profit. It is important for Cyberlab to form some kind of agreement with Microsoft telling who is getting what from the sales of the game. \\ \\
Cyberlab’s key partners are Norut and Microsoft. This means that they are dependent on them to be able to make this game.
Physiotherapists are Cyberlab's customer segment, but they can also serve as a partner. This game fits very well into where the Norwegian health system is going towards (see “Samhandlingsreformen” chapter 3). We mean that this game has a potential to work as a tool for “everyday rehabilitation”. A goal for Cyberlab should be to get this game in as a “regular” tool for prevention and rehabilitation. The government is responsible for the health sector, so they could be a natural partner for the future. Having the government as a partner will solve most of the financial issues. Getting this game into a medical program financed by the government will make this game very credible for the end user. The government could then provide the game to hospitals, care centers, physiotherapy clinics, training groups, and for special cases, when the elderly for example need the game in their own house. \\ \\ 
Et lite avsluttende avsnitt her.

\chapter{Discussion}
This chapter will provide a discussion on our work and our thoughts around the business model. The emphasis will be on aspects considered, but not included in the business model, as the business model itself provides a detailed description on what it contains. \\ \\
The value propositions of this game suggests numerous of advantages. But it is also important to acknowledge some of the issues introduced with the use of technology together with health care. A system like this will most likely contain some private information of the user. To be able to customize the game, it will require  a profile for each user. This profile will also serve as a way for the physiotherapist to keep track on the patients improvements and to adjust the program accordingly.The goal in the long run for this game is that patients eventually can start using this game at home as a supplement to their regular treatment. We believe there is a long way to this point, but it is worth keeping in mind. This means that it has to be possible for the physiotherapist to remotely follow their patient. The profile will probably be provided on a webpage that can be retrieved from a remote server. Anything on the web always face some security issues, and we could spend a lot of time discussing all the different threats and vulnerabilities. However, this is out of the scope of this project and is rather a subject for future work. \\ \\
We have made some assumptions when it comes to what is included in the package Cyberlab is going to sell to their customers. In our calculations we have included the Kinect sensor only, but ideally, they should offer a package with a windows machine and a screen as well. It these packages were offered, Cyberlab would have to buy these extra equipments. If they could buy the equipments in big quantities, they could buy them from a wholesaler to a cost price (innkjøpspris??), and sell with profit. However, this would require Cyberlab to have a huge stock. To keep a stock is both risky and inconvenient. Having a number of TV-sets and computers stored in their premises will be very space consuming. In addition, it is risky in the sense of not knowing how much they will sell, and therefore risking will be some risk associated to this. They do not know how much they will sell, and is therefore risking not selling everything they have in stock. Also, there might come new versions and upgrades, that will make the “old” versions unattractive. This especially accounts for the Kinect sensor. Having a stock will also mean that Cyberlab needs to have the liquidity to be able to buy the quantities, because they will not sell it right away. In addition, this means that Cyberlab has to predict how many customers they will sell to during a fixed period of time, which there is very hard to do. \\ \\
Another prediction we have done in our calculations is how much workload is required for each task. It is hard to predict how many errors and bugs will occur in software and therefore it is impossible to know how many hours are spent on this per year. To make these predictions more precise, we would have needed more experience with software developing. We assume Cyberlab has this kind of experience, and will probably be more suited to make this kind of predictions.\\ \\
The workload assigned the different tasks do not always equal full-time. For example a FTE=1.0 can mean that one person is assigned the project 100 percent for one year, it can mean that four people are assigned the project 100 percent for three months, or it can mean that four people is assigned this project for one year, but only with 25 persent workload each. We have not specified how many people will work on each task in this project. It will be up to Cyberlab to decide what is most efficient for them. The marketing task, is only assigned a FTE= 0.5 in the first year after the game’s release and FTE=0.2 the remaining four years. It Cyberlab does not already have a marketing person, they can consider to hire a marketing consultant to do this job. \\ \\
One of the reasons of choosing Microsoft Kinect as the technology for this game, is the free, open SDK, that enables a third party to develop games for the platform. What we have not been able to find out, is what rights developers have to sell this game with a profit. It is important for Cyberlab to form some kind of agreement with Microsoft telling who is getting what from the sales of the game. \\ \\
Cyberlab’s key partners are Norut and Microsoft. This means that they are dependent on them to be able to make this game.
Physiotherapists are Cyberlab's customer segment, but they can also serve as a partner. This game fits very well into where the Norwegian health system is going towards (see “Samhandlingsreformen” chapter 3). We mean that this game has a potential to work as a tool for “everyday rehabilitation”. A goal for Cyberlab should be to get this game in as a “regular” tool for prevention and rehabilitation. The government is responsible for the health sector, so they could be a natural partner for the future. Having the government as a partner will solve most of the financial issues. Getting this game into a medical program financed by the government will make this game very credible for the end user. The government could then provide the game to hospitals, care centers, physiotherapy clinics, training groups, and for special cases, when the elderly for example need the game in their own house. \\ \\ 




\section{NAV Hjelpemiddelsentralen}
“Nav Hjelpemiddelsentral” is a Norwegian entity with competence in the area of equipments for health-related purposes. Their responsibility is to have expertise about existing equipment, communicate and facilitate for people who need help. They should know about different disabilities and provide customized solutions to help them. There exist different categories of equipment, where technical equipment is one of them. For a tool to be categorized as technical equipment is has to be helpful and necessary for increasing a user’s physical abilities. This imply; being a tool that can help a functionally disabled person to better manage everyday activities. The idea is that people with disabilities can apply for financial support from the government for buying this type of equipment.\\ \\
As Cyberlab’s exergame has the intention of increase physical strength and promote physical activity, and not qualifies as absolutely needed equipment, it will most likely not be categorized as technological equipment. This means is that e.g. an elderly not would get financial support to buy Cyberlab’s exergame.

\section{Discussion}
Fra customer segment: In the starting phase we will recommend Cyberlab to focus on public clinics and private clinics with economic support from the government ("driftstilskudd"). Remaining private clinics will at the moment not be considered as a target customer in this business model. These clinics have other decision making processes, different economy aspects and often a more limited customer base compared to public physiotherapy clinics. With no support from the government these clinics might have a more restricted economy, which complicates purchase of new products. As mentioned, private clinics do not have the same access to customers as public clinics. One reason for this is that customers who consult private clinic have to pay for the therapy session themselves. Elderly may not have that kind of money or willingness to pay for a therapy session at a private physiotherapy clinic (see Appendix, mail fra Nina). Since elderly are the end user of this product, and this group will be challenging to reach through private institutes, we will recommend Cyberlab to not focus on this entity. Some reasons are worth mentioning. Elderly today usually do not have any connection to video games or technology at all. So a developer trying to sell this exercise game by presenting this products value proposition to a person who do not have any knowledge about this area, will be meaningless. The number of technological devices existing today are endless. It will be very difficult for Cyberlab to reach directly out to this customer segment. A solution for Cyberlab to reach their end users is to use someone elderly trusts and rely on. As many others, elderly relies on authorities. Physiotherapists are an example of that kind of authority, so if a physiotherapist tells an elderly that "this product will be good for you", they will most likely believe them. (Legge til eldrehjem, at eldre kan bli brukergruppe senere?). The customer will not always be the end user, and it is important to recognize and pay attention to this difference. For this business model physiotherapy clinics will be Cyberlab's customers, while elderly are the end users. For a physiotherapy clinic, elderly will be both the customer and the end user. (Komme med et eksempel?). A satisfied end user is important for the customer, although they are not the same person. In the beginning we were thinking about elderly as a target customer segment for Cyberlab as they are the end users of the product, but after working with and studied this business model we experienced that elderly is not the proper target for Cyberlab right now. \\ \\
Fra channels - awareness: Through conferences, magazines or just word of mouth one can also get an impression of which products that are popular right now. The popularity mark might attract some customers. Physiotherapists working at private clinics do not have the same access to customers as public clinics have, so they might chose to buy a popular product to attract new and more customers.  
\\ \\ 
(Say something about other possible revenue models we have decided not to focus on, like a license agreement revenue model or a advertisement revenue model.) \\ \\
Fra cost:
In the first year, we suggest that Cyberlab focus on delivering the game to clinics in Trondheim. Selling the game in only Trondheim to an acceptable price, will not make profit. We believe that Cyberlab is looking at some years before they will make profit. The physiotherapy community is small and they have a lot of similarities. By the time this game has become well-known in Trondheim, the game should have gotten attention from clinics in other cities as well. We suggest that Cyberlab hire some of the physiotherapists that are already using the game to present it on conferences and teach other physiotherapist how to use it. In this way, the game will be both more believable to other physiotherapists, as well as Cyberlab do not need to do the hazel with travel around promoting their game. There is still some costs associated to this.\\ \\
Fra revenue - about not including installation prices in our revenue discussion:  or wages for installation and introduction. In this section we will not take wages for installation and introduction into account, as it is very difficult for us to calculate all costs connected to this.
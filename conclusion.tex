\chapter{Conclusion}
There are much uncertainty related to the development of this exergame. It is a new technological solution which will face a small market with potential difficult customers. However, we have seen a great interest for the this type of game, and this together with the envolvement of the Norwegian health sector, shows that there are a great potential for use of Cyberlab’s exergame. Even though we have made many assumptions in our financial analysis, they are realistic enough make a conclusion about the economic aspect of this game. With the assumption that 400 physiotherapy clinics buy Cyberlab’s exergame, we conclude that this project shows economic promise.  
\section{Future Work}
The future work for Cyberlab will be to make a survey of other possible customer segments, in order to extend their market potential. Cyberlab should also look at including other countries in their research.
It should be put a great amount of work into developing and testing the product. The main focus of the development will be to discover what will be the best way to make the product. Which features should be included, and how is it made suitable for its purpose and end users.
Si noe om kostnader og inntekter? 
\chapter{Conclusion}
There is great uncertainty related to the potential of this game. It is a new technological tool which will face a small and immature market. In addition, the fact that the game is not yet developed made it hard for us to understand the game, as well as describe the game to the physiotherapists. However, with support in the Norwegian health sector's new focus and a positive attitude from physiotherapists we have talked to, we believe that the game has a potential on this market. The great amount of research that has been done one video games for health-related purposes the last couple of years, also suggest potential for the exergame. This requires Cyberlab to develop an entertaining and easy-to use game, customized with the right exercises for elderly. If they do this, they can expect to become successful in this market and gain significant profit.  
\section{Future Work}
The future work for Cyberlab will be to make a survey of other possible customer segments, in order to extend their market potential. Cyberlab should also look at including other countries in their research.
It should be put a great amount of work into developing and testing the product. The main focus of the development will be to discover what will be the best way to make the product. Which features should be included, and how is it made suitable for its purpose and end users.
Cyberlab also has to look into the ethical and security related aspects of storing and processing personal data. 
Si noe om kostnader og inntekter? 
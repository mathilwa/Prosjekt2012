\chapter{Motivation}
For us to be able to understand the exercise game and its potential we have to look at the main motivation for the "GameUp" project. The development of the game is based on the problems of falls in elderly. To be able to understand the consequences of this problem, we have done a thorough research in fall statistics and possible outcomes. This chapter review statistics related to problems of fall, which shows how serious this event is. We will describe the game, based on information from our supervisor from Cyberlab. Further, we will describe an example case, showing a typical environment where the exergame can be implemented.  
\section{The Problem of Falls}
Falls are very common in the older population. Even though it does not necessarily seems like a very serious event, it is actually the leading cause of injury in older people.  Fall is considered a public health problem because of the serious consequences for the person falling and the following considerable costs this brings to the country \cite{otago}.
It is estimated that around 30 percent of people over 65 years old and almost 50 percent of people over 80 years old fall at least once a year. 1/10 of these falls result in fracture and one-fifth needs medical treatment. Other serious outcomes of a fall includes pain, trauma and impaired function \cite{otago}.  The worse outcome of a fall is death. 25 percent of elderly getting hip fracture after a fall, dies within a year \cite{gruppetrening-trheim} \cite{larhalsbrudd}. It is shown that after a fall one-third will be afraid of falling again. Being afraid of falling could make them insecure, which can cause an even bigger risk of falling. For many elderly, the fair of falling can make them less active and and it can lead to loss of confidence in carrying out everyday activities. This again can result in fear of leaving their house, which can lead to total inactivity. The latter is a serious problem because a long time of inactivity will result in disabilities and increased risk of falling. Therefore, it is important to find ways to activate elderly and to offer a service that can prevent elderly from developing disabilities \cite{gruppetrening-trheim}. Another issue is that missing the ability to carry out everyday activity can result in loneliness and even depression \cite{exergamesforelderly}. Falls also lead to an increase in economic costs for the government, because of acute treatment after a fall and often also in long-term care \cite{otago}.\\ \\

\section{The Game}
The product Cyberlab is going to develop is an exergame for elderly who is afraid of falling. The game will be used for exercise and rehabilitation, where the focus on the exercises will be on improving physical strength and balance. The game will provide highly relevant training based on carefully designed exercises that professionals know is good for the target group. The idea is that the game will have one regular workout version for training and one version for rehabilitation. The latter will have the ability to customize the exercises. The game will be build for the Kinect sensor for Windows. This is a motion sensor device that can track human motions, so the player do not have to hold on to any controllers.

\section{Example Case}
To be able to understand how this exergame can be used as a tool for exercise and rehabilitation, we will provide the reader with an example case where the game can be used. \\ \\
78 year old Olga lives in her own apartment in central Trondheim. Everything she is in need of is situated in the area, but lately she has started  to feel unsteady and has trouble keeping her balance. She is afraid she will fall if she walks outside, and therefore she does not go out more than necessary. Trondheim is also very icy most of the winter, which increases Olga's fear for falling. Olga is a very social person, but lately there has been little contact with friends and other people because of the fear of going out. In addition, she has no close family nearby to visit her. Beside the unsteadiness, Olga is well and without any physical pain, and thus has no need for physiotherapists. Olga has a great desire to become more steady on her feet so that she can gain an increased social contact, and particularly increase her confidence. \\ \\
Olga has a grandson who is very into video games. Once a year he is visiting her and every time he tells her about the different games he is playing. Sometimes he also brings games to her house to play there, and she is watching enthusiastically. One time he told her about a new game he got for his birthday, a game where he did not need any controllers, but could only stand in front of the TV screen making moves that would appear on the screen. He also told her about the different exergames for the controllers and that this is starting to get very popular in the general population. Olga thinks this sounds very interesting, but at the same time it is too intimidating for her to even consider to buy.   \\ \\
Two months later, Olga feels that she has become weaker after being inactive for so long. Her daughter recommends her to visit a physiotherapist. The clinic is only a couple of blocks away from her house, but everyday before her appointment she is worrying about how she will get there. When the day arrives, she is so anxious that she ends up ordering a taxi. \\ \\
The physiotherapist meets her at the front door, and follows her to his office. After being examined the physiotherapist introduces her to a new project they have just started at the clinic. He tells about an exercise program as a video game that is specially made for elderly. At first it will just be provided at the clinic, but eventually, if the use of the game is a success, they plan to offer it for patients to rent or buy. The program will contain one playing session a week. At first, Olga is very sceptical, but then she suddenly remembers her grandson playing a similar game in her living room a couple of months ago. She is thinking that even though the game looks very intimidating, she will get assistance by the physiotherapist which makes it less scary. This can also be a nice way for her to meet other people. She decides to sign up, but has one concern. How will she get to the clinic? The physiotherapist tells her that as part of the program, they will offer the participants transportation to the meetings until they feel confident getting there themselves. The main goal for the game is to strengthen muscles and improve balance, so after some sessions the participants should see improvements. \\ \\
One week later, Olga visits her first meeting. None of the participants in her group have tried the game before, so everyone gets a thorough introduction. Then they start playing. Olga thinks the game is self-explanatory and very easy to understand, and she gets through the first level without any problem. A physiotherapist is watching them at all times and is guiding them through the game. After the session is over, Olga is tired, but she feels good. What she liked most about the game was how fun it was to compete with the other participant, who motivated and engaged each other. Olga likes the way the Norwegian health care system is heading. She now feels more seen and better taken care of than she has ever felt before. Olga is already looking forward to the next session.


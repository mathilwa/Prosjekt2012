\chapter{Case 1: Prevention for Elderly Afraid of Falling}
78 year old Olga lives in her own apartment in central Trondheim. Everything she might need is situated in the area, but lately she has started  to feel unsteady and has trouble keeping her balance, making sure she does not go out more than necessary. Trondheim is also very icy most of the winter, which increases Olga's fear for falling. Basically, Olga is a very social person, but there has been little contact with friends and acquaintances lately because of the fear of going out. Olga has no close family nearby. Beside the unsteadiness, Olga is very ambulatory without any physical pain, and thus has no need for physiotherapists. Olga has a great desire to be more steady on her feet so that she can gain an increased social contact, and particularly increase her confidence. 
\bigskip
There are many elderly people who are afraid of falling, so Olga is not alone with this problem. Fall is actually considered as a public health problem  \cite{otago}.
It is estimated that around 30 percent of people over 65 years old and almost 50 percent of people over 80 years old fall at least once a year. 1/10 of these falls results in fracture and one-fifth needs medical treatment. The worse outcome of a fall is death. 25 percent of elderly getting hip fracture after a fall, dies within a year \cite{gruppetrening-trheim} \cite{larhalsbrudd}.It is shown that after a fall one-third will be afraid of falling again. Being afraid of falling make them insecure which can result in a even bigger risk of falling. For many elderly, the fair of falling can result in them being less active and a loss of confidence in carrying out everyday activities. This can result in fair of leaving their house, which can lead to total inactivity. The latter is a serious problem because a long time of inactivity will result in disabilities and increased risk of falling. Therefore it is important to find ways to activate the elderly and to offer a service that can prevent the elderly from developing disabilities \cite{gruppetrening-trheim}. Another issue is that missing the ability to carry out everyday activity can result in loneliness and even depression (site dette? ).
\bigskip
To prevent developing disabilities elderly should regularly perform a training program that strengthen their muscles, improve balance and coordination, endurance and mobility \cite{gruppetrening-trheim}. These kind of training programs are offered in Trondheim. There has been established various fitness groups; do you want to get back in shape and become physically stronger? do you feel unsteady and see the need for better balance? Do you manage less now than you did a year ago? Do you find it difficult to go outside? These fitness groups find place at various locations around Trondheim and are offered 1-2 hours one day a week. In addition there exists senior dance, walking groups and water gymnastics \cite{trim}. These activities are good initiatives, but when Olga’s main problem is that she is afraid to go outside, how will she manage to engage in these fitness groups? It is also shown that 2 hours a week with physical activity is not nearly enough to increase Olga’s physical strength \cite{gruppetrening-trheim}. Regular physical activity is the key to become physically stronger and obtain better balance. 
\bigskip
Trondheim municipality did a study where they provided a once a week group training program for elderly. Their study showed that training once a week did not improve physical function for the participants, but the participants expressed that they were less afraid of falling after starting with the group training. The study suggests that this kind of program should be combined with home training programs or other extra physical training offerings \cite{gruppetrening-trheim}. \\

We found that there are already some offered training programs for elderly that can be implemented in their home:\\

\emph{Otago-program}\\
The “Otago”-program is a program developed as a home training program for elderly to prevent falls. It consists of exercises that take about 30 minutes to complete which should be performed three times a week in addition to a walk twice a week. Each customer receives a booklet with instructions for the individual exercises prescribed in addition to ankle cuff weights. The participants needs to record the days they complete the program for follow-up purposes. For follow-up an instructor should do home visits every six months and telephone them every month. The instructor can then increase the difficulty in the prescribed exercises for each individual. The program has been tested and evaluated for 1016 home living people aged 65 to 97. The program was shown to reduce falls and falls related injuries with 35 percent, with the highest effect on those over 80 years old and those that have had a previous fall. The participants experienced improved strength and balance, as well as they maintained their confidence so it was easier for them to do everyday activities without being afraid of falling. \cite{otago} \cite{gruppetrening-trheim}\\

\emph{Falls Management Exercise}\\
Falls Management Exercise (FaME) is an exercise program consisting of tailored group and home-based exercises and builds on the core exercises from the “Otago”-program.  There are a total of three group training sessions per week, in addition to two home-training sessions per week. The exercise intervention is designed to improve participants’ dynamic balance and core and leg strength.  In the United Kingdom a study was done where they examined the effectiveness of this program for home-living women aged 65 or older who had already fallen 3 or more times within the previous year. After using FaME for 36 weeks the fall rate was reduced by one-third. The conclusion of the study was that the exercise program should last for at least 36 weeks including at least 2 hours of training per week. For progression it is important that the intensity, resistance, and weight are continually increased, as well as the balance gets challange.\cite{fame}\\

\emph{{Ø}velsesbanken}\\
{Ø}velsesbanken is a Scandinavian project providing a user profile with different training programs. The different exercises are developed from the two previous described concepts and other relevant studies on balance and exercising for elderly. The program gives an idea on how you can put together an exercise program customized for each individual. It is primarily made as a tool for physiotherapists for putting together training programs for their patients to do at home. As we see it, it can also be used as a tool for each individual to put together their own program, because you can also log in as a private user and make your own program. The program offers the user a choice of different exercises that you can add to your exercise program. When all exercises are chosen you can print out pdf-files with pictures and descriptions or you can read from the computer screen. It is an easy, self-explanatory and straightforward program to use. {Ø}velsesbanken is in use in Scandinavia and the summer of 2012 it had reached 4300 users. \cite{ovelsesbank}
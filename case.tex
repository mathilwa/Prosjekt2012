\chapter{Motivation}
For us to be able to understand this game, why it is made and where it will fit in, we have to understand the problems the "GameUp" project is founded on.  \\ \\
This chapter describes the problems of fall which is the main motivation for the "GameUp" project and why Cyberlab will develop this game. We will describe the game, and to understand how this game would used in the real world we will describe a example case.   

\section{The Problem of Falls}
Falls are very common in the older population. Even though it does not necessarily seems like a very serious event, it is actually the leading cause of injury in older people.  Fall is considered a public health problem because of the serious consequences for the person falling and their considerable cost to the country \cite{otago}.
It is estimated that around 30 percent of people over 65 years old and almost 50 percent of people over 80 years old fall at least once a year. 1/10 of these falls results in fracture and one-fifth needs medical treatment. Other serious outcomes of a fall includes pain, trama and impaired function \cite{otago}.  The worse outcome of a fall is death. 25 percent of elderly getting hip fracture after a fall, dies within a year \cite{gruppetrening-trheim} \cite{larhalsbrudd}. It is shown that after a fall one-third will be afraid of falling again. Being afraid of falling could make them insecure which can result in an even bigger risk of falling. For many elderly, the fair of falling can result in being less active and loss of confidence in carrying out everyday activities. This can result in fear of leaving their house, which can lead to total inactivity. The latter is a serious problem because a long time of inactivity will result in disabilities and increased risk of falling. Therefore, it is important to find ways to activate the elderly and to offer a service that can prevent the elderly from developing disabilities \cite{gruppetrening-trheim}. Another issue is that missing the ability to carry out everyday activity can result in loneliness and even depression (site dette? ). Falls are also resulting in an increase in economic costs for the government, including both the acute treatment after a fall and often also in long-term care. \cite{otago}\\ \\

\section{The Game}
The product Cyberlab is going to develop is an exergame for the Kinect sensor for Windows. Kinect is a motion sensor device that can track human motions, so the player do not have to hold on to any controllers. The game will be used for prevention and rehabilitation, where the focus on the exercises will be on improving physical strength and balance. The idea is that the game will have one regular workout version for prevention and one version for rehabilitation, with ability to customize the exercises. 

\section{Example Case}
To be able to understand how this exergame can be used as a tool for exercise, prevention and rehabilitation, we will provide the reader with an example case. \\ \\
78 year old Olga lives in her own apartment in central Trondheim. Everything she might need is situated in the area, but lately she has started  to feel unsteady and has trouble keeping her balance, making sure she does not go out more than necessary. Trondheim is also very icy most of the winter, which increases Olga's fear for falling. Basically, Olga is a very social person, but lately there has been little contact with friends and other people in general because of the fear of going out. Olga has no close family nearby. Beside the unsteadiness, Olga is well and without any physical pain, and thus has no need for physiotherapists. Olga has a great desire to be more steady on her feet so that she can gain an increased social contact, and particularly increase her confidence. \\ \\
Olga has a grandson who is very into video games. Once a year he is visiting her and every time he tells her about the different games he is playing. Sometimes he also brings games to her house to play there, and she is watching enthusiastically. One time he told her about a new game he got for his birthday, a game where he did not need any controllers, but could only stand in front of the TV screen making movements that would appear on the screen. He also told her about how you could get different exercise games for the controllers and that this is starting to get very popular in the general population. Olga thinks this sounds very interesting, but at the same time it is too intimidating for her to even consider to buy.   \\ \\
Two months later, Olga feels that she has become weaker after being inactive for so long. Her daughter recommends her visiting a physiotherapist. The clinic is only a couple blocks away from her house, but everyday before her appointment she is worrying about how she will get there. When the day arrives, she is so anxious that she ends up ordering a taxi. \\ \\
The physiotherapist meets her in the door, and follows her to his office. After being examined the physiotherapist introduces her to a new project they have just started at his clinic. He tells about an exercise program as a video game that is specially made for elderly people. At first it will just be provided at the clinic, but eventually, if the use of the game is a success, they plan to offer it for patients to rent or buy. The program will contain one playing session a week, and will be played by up to four players. At first, Olga is very sceptical, but then she suddenly remembers her grandson playing a similar game in her living room a couple of months ago. She is thinking that even though the game looks very intimidating at first sight, here she will at least get some assistance. She decides to sign up, but has one concern. How will she get to the clinic? The physiotherapist tells her that as part of the program, they will offer the participants transportation to the meetings until they feel confident getting there themselves. The main goal for the game is to strengthen muscles and improve balance, so after a while the participants should see improvements. \\ \\
One week later, Olga visits her first meeting. None of the participants in her group have tried the game before, so everyone gets an thoroughly introduction. Then they start playing. Olga thinks the game is self-explanatory and very easy to understand, and she get through the first level without any problem. A physiotherapist is watching them at all times and is guiding them through the game. After the session is over, Olga is tired, but she feels good. What she liked most about the game was how fun it was to compete with the other participant, who motivated and engaged each other. Olga likes the way the Norwegian health care system is heading, when she now feels more seen and better taken care than she have ever felt before. She is already looking forward to the next session.


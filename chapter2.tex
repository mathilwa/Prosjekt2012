\chapter{Health Politics in Norway Today}

\section{National Health Politics in Norway}
The government states that it is a public responsibility to promote health and prevent diseases to make sure that the population get the health care they need. The goal is to overall get a healthier population. A good health is necessary for an individual to acquire good quality of life. It is also important for the society, especially economically. We are awaiting a boost of elderly in a couple of years, in it is important to prepare for that. In Norway there is a goal to offer everyone in the need of it, a place in care homes by 2015.  To be able to meet all all the requirements set, there is a need for a change in the health sector in Norway.\\ \\
\subsection{Samhandlingsreformen}
"Samhandlingsreformen", from now called "the reform", has been implemented from 1. January 2012. It presents a new way to organize the health service in Norway. In the reform there is an increased focus on prevention, early intervention and close collaboration between different entities. The health services should be offered closer to where people live and they should be offered more comprehended and coordinated treatment. Welfare technology shall support these strategies where possible. Technical solutions and methods can make it possible to treat more patients in a better way. Every 75-year old are offered supervision to promote health and own coping. Every person who is in need of health care services that can be deployed in their own house, should be provided this. The services offered in the private home should be improved. If the services in the private home could be improved to the level where it helps the user being able to live in their own home for three more months, it will be equal to 10 percent of the capacity in care homes. This will depend on better offers outside the clinics. The main focus now should be "hverdagsrehabilitering", from now on called  "everyday rehabilitation" in english. This means that a person will get health care first after an evaluation that proves that he or she actually has a need for rehabilitation. The target group are the ones with moderate limitations in functional level. The goal is to postpone their need for extensive help and help them to a more dependent everyday life.  "Everyday rehabilitation" has the home care staff as a basis and physio - and ergotherapy as an "engine". If a patient is considered to be in the need of help and support in their own home, they will first meet with a physio -or ergotherapist, who will examine them. The new strategy for the health care system just described is based on the "Fredericia-model". This model is developed in the Danish municipal "Frederica" and is about how physical, social and cognitive capabilities can be maintained and improved, so that functional disabilities in the older generation will be postponed. The experiences from this model have been very positive. 30 percent receive "everyday rehabilitation" instead of ordinary home care. Approximately 45 percent of these end the rehabilitation process with no need of any further help, and 40 percent end the rehabilitation process with need of less help than one can assume they would need.\cite{budsjett}\cite{regjering}
\subsection{Welfare Technology}
The use of welfare technology can be an important tool in the future demographic challenges and in the health promoting work. It can be used for example as a tool for preventing falls, loneliness and cognitive loss. \\ \\
The main reasons to focus on welfare technology are many. It can make it easier for an individual to cope with their own life and health better, and that people with a need for health care can live longer at home. Can increase the flexibility , improve the quality of services and can create a new arena for interaction with different actors. Can contribute to innovation, can give public and private value creation, and will give a positive socio-economic effect. \\ \\
The market for welfare technology is very immature. There are no initiators that can make sure it will be established public and private demand. To make public demand, this role should be taken by the municipal. Still, there is a long way before robust solutions is established. There is a need for increased emergence and testing of services. \cite{welfare} HER MÅ DET SKRIVES OM EN DEL OG LUTT MER

\section{Fall Prevention and Rehabilitation Today}
Physiotherapy is a science related to the medical field and central to this subject is body and movement. The theoretical basis for this subject is grounded in knowledge about science and society, and the recognition that different factors contribute to the maintenance of health. In addition to injuries and diseases, quality of life, experiences as well as social and cultural factors leads to pain and disability. This understanding along with practical and clinical knowledge forms the basis for evaluation of how injuries and pain can be treated and prevented. Physiotherapy can be described as examining, treating and preventive. Included are manual techniques, exercises and possible use of technical methods. The subject of physiotherapy is performed by a physiotherapist. Physiotherapists are traditionally working in municipalities, hospitals and in private institutions, and they are working with both individual treatment and treatment in groups. The goal for physiotherapists is to make their patients’ daily activities easier to manage \cite{physiotherapy1}\cite{physiotherapy2}.\\ \\
Elderly people consult physiotherapists' in terms of rehabilitation after surgery, after a fall, stroke or other injuries, or when they feel health problems make it hard to perform everyday tasks. Physical therapy usually includes exercise with focus on increasing the patients’ flexibility, endurance and strength. Physiotherapists set up customized training for each patient according to what kind of needs they have. Unfortunately, the time a patient spends with the physiotherapists per week is not sufficient in order to become stronger or to recover from injuries. Therefore, one sees the need for patients to practice outside of these hours at the physiotherapists. Physiotherapists may give their patients exercises they can practice at home. However, not everyone is motivated to exercise on their own and many skip the weekly exercise that is scheduled for them. \cite{physiotherapy2} 


\section{Existing Prevention Programs}
To prevent developing disabilities elderly should regularly perform a training program that strengthen their muscles, improve balance and coordination, endurance and mobility \cite{gruppetrening-trheim}. These kind of training programs are offered in Trondheim. There has been established various fitness groups; do you want to get back in shape and become physically stronger? do you feel unsteady and see the need for better balance? Do you manage less now than you did a year ago? Do you find it difficult to go outside? These fitness groups find place at various locations around Trondheim and are offered 1-2 hours one day a week. In addition there exists senior dance, walking groups and water gymnastics \cite{trim}. These activities are good initiatives, but when the main problem for Olga is that she is afraid to go outside, how will she manage to engage in these fitness groups? It is also shown that 2 hours a week with physical activity is not nearly enough to increase Olga’s physical strength \cite{gruppetrening-trheim}. Regular physical activity is the key to become physically stronger and obtain better balance. \\ \\
Trondheim municipality did a study where they provided a once a week group training program for elderly. Their study showed that training once a week did not improve physical function for the participants, but the participants expressed that they were less afraid of falling after starting with the group training. The study suggests that this kind of program should be combined with home training programs or other extra physical training offerings \cite{gruppetrening-trheim}. \\ \\
We found that there are already some offered training programs for elderly that can be implemented in their home:\\ \\
The \emph{“Otago”-program} is a program developed as a home training program for elderly to prevent falls. It consists of exercises that take about 30 minutes to complete which should be performed three times a week in addition to a walk twice a week. Each customer receives a booklet with instructions for the individual exercises prescribed in addition to ankle cuff weights. The participants needs to record the days they complete the program for follow-up purposes. For follow-up an instructor should do home visits every six months and telephone them every month. The instructor can then increase the difficulty in the prescribed exercises for each individual. The program has been tested and evaluated for 1016 home living people aged 65 to 97. The program was shown to reduce falls and falls related injuries with 35 percent, with the highest effect on those over 80 years old and those that have had a previous fall. The participants experienced improved strength and balance, as well as they maintained their confidence so it was easier for them to do everyday activities without being afraid of falling. \cite{otago} \cite{gruppetrening-trheim}\\ \\
\emph{\ac{fame}} is an exercise program consisting of tailored group and home-based exercises and builds on the core exercises from the "Otago"-program.  There are a total of three group training sessions per week, in addition to two home-training sessions. The exercise intervention is designed to improve participants dynamic balance and core and leg strength.  In the United Kingdom a study was done where they examined the effectiveness of this program for home-living women aged 65 or older who had already fallen 3 or more times within the previous year. After using FaME for 36 weeks the fall rate was reduced by one-third. The conclusion of the study was that the exercise program should last for at least 36 weeks including at least 2 hours of training per week. For progression it is important that the intensity, resistance, and weight are continually increased, as well as the balance gets challange.\cite{fame}\\ \\
\emph{{Ø}velsesbanken} is a Scandinavian project providing a user profile with different training programs. The different exercises are developed from the two previous described concepts and other relevant studies on balance and exercising for elderly. The program gives an idea on how you can put together an exercise program customized for each individual. It is primarily made as a tool for physiotherapists for putting together training programs for their patients to do at home. As we see it, it can also be used as a tool for each individual to put together their own program, because you can also log in as a private user and make your own program. The program offers the user a choice of different exercises that can be added to a exercise program. When all exercises are chosen PDF-files can be printed with pictures and descriptions of the exercises, or the user can read them from the computer screen. It is an easy, self-explanatory and straightforward program to use. {Ø}velsesbanken is in use in Scandinavia and the summer of 2012 it had reached 4300 users. \cite{ovelsesbank}\\ \\

\chapter{Interview Discussion}
In this chapter we will discuss the findings we did in the interviews. We will only take into consideration what we find important for the analysis of the business model. Everything is taken from the interviews conducted, in addition to our own opinions and perception of the interviews. For the interested reader, the interview reports can be found in Appendix A - D (only in Norwegian).\\ \\
Two of the interviewees were physiotherapists working in government controlled clinics, from now on called public clinics. The third interviewee worked in a private clinic, owned by herself and a partner. All of the interviewees had elderly people as a patient group, but not necessarily their only patient group. The private clinic offers an exercise program where elderly can meet and exercise with other elderly once a week. This is also something that is arranged in the public arena, called "Seniortrim" \cite{trim}. The latter is a training group specifically for fall prevention. One of the interviewees mentioned that one of the problems of motivating the patients is that they do not identify themselves as  persons being afraid of falling, so when working with fall prevention it is important to not mention the word "fall". Therefore, in promoting "Seniortrim", they are focusing more on encouraging them to improve their physical health. This is something to keep in mind when promoting Cyberlab's exergame. The typical feedback from these training groups is that the participants think it is nice to get some physical movement and that the social aspects are important. The training group in the private clinic costs 60 NOK per session and the one in the public arena costs 30 NOK per session, allowing most people to afford it. This is an arena where people can play together, as well as getting assistance from a professional. We see this as two interesting offerings, where there is a potential for the exergame to be implemented.  \\ \\
Two of the interviewees mentioned that they wish to get the patients in for an examination earlier. Most of the patients go to the physiotherapist first when their problem gets serious. Getting them in earlier means a chance for prevention instead of rehabilitation. Everything really depends on the patients' background; whether or not it is important for the patients to be able to continue working out everyday activities, or if they just accept the fact that they are getting older and are not able to do everything they did before. Typically, people that are used to be active in the sense of often taking a walk, go skiing during the winter etc., will be more eager to keep a good physical health. For those that are used to get physical activity from for example gardening, the relation to other type of physical activity may not be present. Thus, it is very important to not look at elderly as one common patient group. They are also different people with different interests. This is important to acknowledge when developing a game for this group.\\ \\
A normal problem is that elderly are afraid of walking outside their house, making it hard for them to attend their appointments and also other type of activities. Some people are very motivated to improve themselves while others are satisfied with how things are. Usually, one hour at the physiotherapist is not enough to improve physical strength and it is hard to get the patients to exercise at home, often because of the lack of motivation. It is common to provide the patients with an exercise program that can be performed at home, but there are some problems related to this. First, the patient may not be motivated enough to actually perform the program. Second, it is hard for the physiotherapist to give feedback to make sure that the exercises are done right. Finally, there is no one there to make sure that they do not fall and hurt themselves. The two first problems may be solved with an exercise game, while the latter problem also is related to the exergame.  \\ \\
With regard to our business model we asked all the interviewees where they hear about new products, treatment methods and tools. Different channels were mentioned, like "Fysioterapeuten", which is an academic magazine for physiotherapists, different conferences, "NAV Hjelpemiddelsentralen", and from suppliers they already had an established relationship with. Every one of them pointed out how important it is that the product Cyberlab is trying to sell is proved to work. It is not enough to have an ad in a magazine or newspaper, or show up demonstrating on a conference, if there is no well documented effect. If the product is proved to have good effect, physiotherapists' staff meetings could be an effective arena for promoting the product. Again, she emphasized the importance of well documented good effect. The threshold for the use of a product should be very low. It must be easy to use and must also ease the workload for the physiotherapist. If it is not better than what the physiotherapists can offer themselves, they would not use it. In the public clinics the government pays for everything. Every year, a certain amount of money is given to each clinic. The clinic is then responsible for how they want to spend the money. If a physiotherapist believes in and wishes to get a new product, the leader will be responsible to consider it and decide if they should buy it. This is different in the private clinics. Here every decision are made by the owner(s). \\ \\
The most common way to acquire a product is to buy and own it. Leasing is not that common, but not irrelevant. The private clinic was only one year old, so their economy had not had time to get stable yet. Because of this vulnerability they were careful about buying expensive new products and they were also sceptic to subscribe on a rental agreement. We represented a fictive scenario where they could get an opportunity of paying only for the use of the game. She was very positive to this possibility. All of the interviewees mentioned that it might be interesting to rent a product for a certain period of time to see if it was interesting to buy. It would also be interesting to try something for free.  One of the interviewees mentioned that for this specific game, it would be necessary to run a pilot project, where some clinics tried the game out for a couple of months. In this way they could test the game and document the effect. It is not enough for the potential buyer to know that this is an EU-funded project and that it seems like a nice product. The product must be well proved over a longer period of time. \\ \\
The main goal for the exergame is to prevent elderly from gradually losing their physical abilities. It is shown that exercise only once a week is not enough \cite{gruppetrening-trheim}. This could also be confirmed by one of the interviewees. At the same time, some exercise is better than none, and the physiotherapist who provided the group training once a week told that you could actually see improvements after six to seven sessions. For the elderly to be able to improve themselves even more, they will have to exercise more. A plan for the future should therefore be to provide the game for the elderly to use in their own home. It is not natural for us to believe that an older person would go to the store and buy this game themselves. Therefore, it was interesting for us to see if physiotherapists could work as a possible channel where the patient could get this game. In the public clinics it is normal to let patients borrow products, but the number are limited.  If the government believed in a product, they could buy some copies that patients could borrow. In the private clinic it is normal to sell products, like for example special shoe soles, but they do not typically let patients barrow products. \\ \\
It is not common for physiotherapists to physically go to the store and buy a product. An already established relationship with a supplier is the most common channel where new products are bought. Often after a sale, the supplier will contact the clinic with improved or new products. It is quite normal to maintain a relationship with the supplier after a product is bought. Normally, the supplier provides the clinic with brochures with news and improvements, as well as support if something goes wrong with the product. When the product is bought, it usually gets delivered at the clinic. Physiotherapists have already enough to do as it is, and have not always time to set up and learn a new product. It depends on the product whether they get an introduction or not. It would for instance often be necessary to get an introduction to technical products. When it comes to relationship with the supplier, all the physiotherapists we interviewed stated that the opportunity to come with feedback on the product was important. The reason for this was that they wanted the product to be customized for their and their patients need. \\ \\
One of the interviewees talked about how this game could fit into the Norwegian government’s new reform "Samhandlingsreformen" discussed in Chapter 3. Of course this depends on whether the game is proved to be effective in health improving purposes and on cost.  \\ \\
All of the interviewed physiotherapists were positive to Cyberlab’s exergame. At the same time, neither of them could say anything about whether they would use it or not, nor if they thought it would be a suitable tool for elderly to use. The reasons for this were that the game has not been developed yet, and they would have to see it and try it before they could make an opinion about it. They expressed that it is important that the game can ease their workload and also offer something better than what they can offer themselves. The ability to customize the program was an aspect that was very important for all of the interviewees. The game should provide the possibility to put together different type of exercises and change the degree of difficulty. To make it possible for elderly to use this kind of game it has to be easy to use and self-explanatory. In addition, the game should give some kind of feedback to the user with instructions on whether they are doing the exercise right or wrong. This can work both as a motivating factor for the user and as a tool for the physiotherapist to keep track on how the patient is doing. The opportunity to not only customize in the sense of the right exercises, but also different themes, should make the game more fun and motivating. \\ \\
It was pointed out that it is important to remember that elderly today are not familiar with technology. This game will probably be more relevant when the next generation gets older, especially with regard to the use in private homes. Today, we can assume that the older patient would not on their own initiative buy this game, but rather use it after a recommendation from their physiotherapist. \\ \\
Another issue pointed out was where the game could be used. The different clinics may not have space to have for instance four people playing a video game in one room. Most likely, to be able to take the game seriously, as well as not disturb other patients in the clinic, the game has to be played in a separate room. This must be up to each clinic to find a way to work around. An issue implementing this game in a private clinic might be that they do not have that big customer base. The reason for this is that patients have to pay for the treatment. At the same time, it could be an interesting tool for them to use to tempt customers to choose them as a clinic. Of course, this might not work, if "all other" clinics offered this game as well. \\ \\
From the interviews conducted, we can conclude that physiotherapists are the right customer segment for Cyberlab. Today, it would not be relevant to go straight to the end-user, because they are not familiar with this kind of technology. A physiotherapist is a professional who the patient can trust. If a professional say something, we are likely to believe in it.  The physiotherapists wish to get their patients in for examination earlier so they can prevent the patient from getting serious health problems. Offering the exergame in for example a group training session, can encourage elderly to go to the physiotherapist earlier. It is important that the game can be customized both in the favour of the physiotherapist providing it and the patient using it. This include different exercises, levels and themes. The game has to be shown to improve patients' physical health as well as their mental health. It should be fun, motivating and easy to use. And at last, it has to ease the physiotherapists workload. \\ \\
In the next chapter we will provide a business model where we analyse the potential of this game with physiotherapy clinics as the customer segment. We mean that the information gathered from the interviews support our choice of customer segment, even though there were some uncertainties around the game. There is a potential both in a training group situation and also for individual treatment. In the long run, physiotherapists can serve as a channel to sell the product to the end-user. This will no be taken into consideration in our assignment. 
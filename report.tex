\documentclass[b5paper,twoside,openright,11pt]{report}
\usepackage[printonlyused]{acronym}
\usepackage{graphicx}
\usepackage{cite}
\usepackage[utf8]{inputenc}
\usepackage[hyphens]{url}
\usepackage{rotating}
\usepackage{amsmath}
\usepackage{fancyhdr}
\setlength{\headheight}{25.23pt}
\renewcommand{\headrulewidth}{0pt}
\pagenumbering{gobble}
\begin{document}
\begin{flushleft}
\begin{figure}[htb]
\includegraphics[scale=0.6]{NTNU-logo}
\end{figure}
\bigskip
\bigskip
\bigskip
\bigskip
\begin{huge}
\textbf{Business Opportunities and Economics of an Exercise Game in the Health Sector}\\
\end{huge} 
\bigskip
\bigskip
\bigskip
\bigskip
\bigskip
\bigskip
\bigskip
\begin{Large}
\textbf{Kine Omholt and Mathilde Wærstad \\}
\end{Large}
\bigskip
\bigskip
\bigskip
\bigskip
\bigskip
\bigskip
\begin{large}
\textbf{Project Assignment\\}
\end{large}
Delivered: \today\\
Professor: Harald Øverby\\
Supervisor: Tor Ivar Eikaas, Cyberlab\\
\bigskip
\bigskip
\bigskip
\bigskip
\bigskip
Norwegian University of Science and Technology\\ 
Faculty of Information Technology, Mathematics and Electrical Engineering\\
Department of Telematics
\end{flushleft}
\cleardoublepage
\begin{abstract}
In today's society we are facing a huge amount of elderly with physical health problems. Many of these problems are related to decrease in physical strength and balance, which lead to decreased mobility and high risk of falling.  Falls are a serious event for elderly because it might lead to reduced quality of life due to inactivity, loneliness, loss of self-confidence and depression. The worst outcome is death. The different outcomes of a fall can also be costly for the society. Based on the problem of falls, there has been started an EU-funded project, called "GameUp". The goal of this project is to use technology to encourage elderly to become more physically active.  One of the participants in this project is Cyberlab, a company located in Trondheim. Their role is to develop a game for exercise and rehabilitation for elderly, by using the Microsoft Kinect sensor for Windows. In this assignment we will analyse the business potential of this game. To be able to see where this game could fit, research on relevant topics have been conducted, like the Norwegian health sector, and what solutions to the fall problem that exist today. We will also look at different technologies relevant for the game, and review research that have already been done on the use of games for exercise and rehabilitation. In the research conducted we found that the new reform, "Samhandlingsreformen", introduce a new focus in the Norwegian health sector, which involves increased focus on prevention and early intervention. In the work of achieving these goals welfare technology shall be implemented where possible. We observed that physiotherapists have an essential role in this new reform. There have been done a great amount of research when it comes to using video games for health-related purposes, and the conclusion in many of these studies is that use of video games for exercise can have a positive health effect. Previous research also suggest that video games may have potential for rehabilitation. However, there is a need for customized games for elderly, as the commercial games are not made for their needs. Cyberlab has chosen to use the Microsoft Kinect sensor for their exergame. We support this choice as this technology is proved to accurately measure body movements viable for clinical practice. In addition, the convenience of not having to hold on to any controllers is an advantage for the elderly population. To analyse the business potential for this game we used Alexander Osterwalder's Business Model Ontology. To be able to build the business model, interviews with three different physiotherapists were conducted. From this, we build a business model around the customer segment:  public physiotherapy clinics and private physiotherapy clinics with contribution from the government. The value proposition Cyberlab can offer this customer segment is: a tool with the ability to customize an exercise program and to offer an alternative, fun and motivating training method while at the same time ease the workload of the physiotherapist. A financial analysis is provided, where an appropriate pricing model is proposed. The chosen model is an usage fee model, which involves the user paying a low initial price for the game and an additional fee for the time used playing the game. We found that this pricing model generated the largest profit based on an estimated market of 400 physiotherapy clinics. There are uncertainties related to this exergame. However, with support in the Norwegian health sector's new focus we evaluate the exergame to have a successful market potential. Our financial analysis also suggest this project to have economic potential. 
\end{abstract}
\cleardoublepage
\chapter*{Preface}
This thesis is written as a project assignment submitted to the Norwegian University of Science and Technology (NTNU) as a part of our master degree in Communication Technology, Tele-economics. 
We would like to thank our professor Harald Øverby for valuable feedback, guidance and ideas on our work. We will also thank our supervisor, Tor Ivar Eikaas from Cyberlab, for providing us with information about the project, the exergame, and the company. His guidance has been very helpful. We would like to thank the three physiotherapists who were willing to let us interview them. They all provided us with highly useful information. The information we gathered have made it possible for us to give Cyberlab a well-grounded evaluation of their project.  
Finally, we will thank each other for good team-work and collaboration during the semester. It has been a good experience! 

\cleardoublepage
\pagenumbering{roman}
\tableofcontents
\cleardoublepage
\chapter*{Acronyms}
\begin{acronym}
\acro{nff}[NFF]{Norsk Fysioterapeutforbund}
\acro{sdk}[SDK]{Software Development Kit}
\acro{fte}[FTE]{Full Time Equivalent}
\acro{fame}[FaME]{Falls Management Exercise}
\acro{ddr}[DDR]{Dance Dance Revolution}
\acro{rgb}[RGB]{Red, Green, and Blue}
\acro{bbs}[BBS]{Berg Balance Scale}
\acro{fesi}[FES-I]{The Falls Efficacy Scale - International}
\acro{ee}[EE]{Energy Expenditure} 
\acro{ascm}[ASCM]{American College of Sports Medicine} 
\acro{nok}[NOK]{Norwegian Kroner} 
\acro{csrt}[CSRT]{Choice Stepping Reaction Time} 
\end{acronym}


\cleardoublepage
\listoffigures
\cleardoublepage
\listoftables
\cleardoublepage
\pagenumbering{arabic}
\pagestyle{fancy}
\fancyhead[LE]{\thepage}
\fancyhead[RE]{\leftmark}
\fancyhead[RO]{\thepage}
\fancyhead[LO]{\rightmark}
\fancyfoot{}
\chapter{Introduction}
\section{Scope}
\section{Problem Definition}
Games are becoming more and more important in the health sector. One specific genre of games is exercise games (exergames). In this assignment we will focus on the business opportunities and economics of games in the health sector, with special focus on exergames for rehabilitation and elderly. The work has been done in collaboration with Cyberlab, a company focusing on serious games for education and training. The work will provide input to an EU funded project focusing on elderly afraid of falling.

In particular, the following studies will be done: 
\begin{itemize}
\renewcommand{\labelitemi}{$\bullet$}
\item A background study of exergames in general and exergames for rehabilitation and elderly in particular.
\item	A description of specific business cases.
\item	Analyzing the business potential of this type of games in the health sector using Osterwalder business model.
\item	Miscellaneous aspects to consider about this type of game
\end{itemize}

\section{About the Project}
This project assignment is based on an EU-funded project called "GameUp". The purpose of "GameUp" is to use technologies that are proved to improve motivation to encourage elderly to be more physically active. The goal is to sustain and enhance mobility in older people so they can live longer at home which may result in a better quality of life. "GameUp" wants to increase motivation and self-efficacy towards mobility by focusing on physical activity capabilities. They will make convenient and easy to use exercise games and social games using low cost motion sensors and commercial modules and products \cite{gameup}.\\ \\ This project is a European cooperation with different partners involved. One of these partners is Cyberlab, a small company located in Trondheim. They are working with development of simulators and simulation-based games primarily for technical education and training, but also for promotion and exemplification of technical products and services (nesten avskrift fra deres hjemmeside, \cite{cyberlab}. Their work in the "GameUp" project is to develop a exercise game using Microsoft Kinect as a platform. (Tor-Ivar).

\section{Limitation of Scope}
The game has not been developed yet. Immature marked. Hard to predict demand. 

\section{Outline}
Hvordan kapittelene henger sammen..
\cleardoublepage
\chapter{Motivation}
For us to be able to understand the exercise game and its potential we have to look at the main motivation for the "GameUp" project. The development of the game is based on the problems of falls in elderly. To be able to understand the consequences of this problem, we have done a thorough research in fall statistics and possible outcomes. This chapter review statistics related to problems of fall, which shows how serious this event is. We will describe the game, based on information from our supervisor from Cyberlab. Further, we will describe an example case, showing a typical environment where the exergame can be implemented.  
\section{The Problem of Falls}
Falls are very common in the older population. Even though it does not necessarily seems like a very serious event, it is actually the leading cause of injury in older people.  Fall is considered a public health problem because of the serious consequences for the person falling and the following considerable costs this brings to the country \cite{otago}.
It is estimated that around 30 percent of people over 65 years old and almost 50 percent of people over 80 years old fall at least once a year. 1/10 of these falls result in fracture and one-fifth needs medical treatment. Other serious outcomes of a fall includes pain, trauma and impaired function \cite{otago}.  The worse outcome of a fall is death. 25 percent of elderly getting hip fracture after a fall, dies within a year \cite{gruppetrening-trheim} \cite{larhalsbrudd}. It is shown that after a fall one-third will be afraid of falling again. Being afraid of falling could make them insecure, which can cause an even bigger risk of falling. For many elderly, the fair of falling can make them less active and and it can lead to loss of confidence in carrying out everyday activities. This again can result in fear of leaving their house, which can lead to total inactivity. The latter is a serious problem because a long time of inactivity will result in disabilities and increased risk of falling. Therefore, it is important to find ways to activate elderly and to offer a service that can prevent elderly from developing disabilities \cite{gruppetrening-trheim}. Another issue is that missing the ability to carry out everyday activity can result in loneliness and even depression \cite{exergamesforelderly}. Falls also lead to an increase in economic costs for the government, because of acute treatment after a fall and often also in long-term care \cite{otago}.\\ \\

\section{The Game}
The product Cyberlab is going to develop is an exergame for elderly who is afraid of falling. The game will be used for exercise and rehabilitation, where the focus on the exercises will be on improving physical strength and balance. The game will provide highly relevant training based on carefully designed exercises that professionals know is good for the target group. The idea is that the game will have one regular workout version for training and one version for rehabilitation. The latter will have the ability to customize the exercises. The game will be build for the Kinect sensor for Windows. This is a motion sensor device that can track human motions, so the player do not have to hold on to any controllers.

\section{Example Case}
To be able to understand how this exergame can be used as a tool for exercise and rehabilitation, we will provide the reader with an example case where the game can be used. \\ \\
78 year old Olga lives in her own apartment in central Trondheim. Everything she is in need of is situated in the area, but lately she has started  to feel unsteady and has trouble keeping her balance. She is afraid she will fall if she walks outside, and therefore she does not go out more than necessary. Trondheim is also very icy most of the winter, which increases Olga's fear for falling. Olga is a very social person, but lately there has been little contact with friends and other people because of the fear of going out. In addition, she has no close family nearby to visit her. Beside the unsteadiness, Olga is well and without any physical pain, and thus has no need for physiotherapists. Olga has a great desire to become more steady on her feet so that she can gain an increased social contact, and particularly increase her confidence. \\ \\
Olga has a grandson who is very into video games. Once a year he is visiting her and every time he tells her about the different games he is playing. Sometimes he also brings games to her house to play there, and she is watching enthusiastically. One time he told her about a new game he got for his birthday, a game where he did not need any controllers, but could only stand in front of the TV screen making moves that would appear on the screen. He also told her about the different exergames for the controllers and that this is starting to get very popular in the general population. Olga thinks this sounds very interesting, but at the same time it is too intimidating for her to even consider to buy.   \\ \\
Two months later, Olga feels that she has become weaker after being inactive for so long. Her daughter recommends her to visit a physiotherapist. The clinic is only a couple of blocks away from her house, but everyday before her appointment she is worrying about how she will get there. When the day arrives, she is so anxious that she ends up ordering a taxi. \\ \\
The physiotherapist meets her at the front door, and follows her to his office. After being examined the physiotherapist introduces her to a new project they have just started at the clinic. He tells about an exercise program as a video game that is specially made for elderly. At first it will just be provided at the clinic, but eventually, if the use of the game is a success, they plan to offer it for patients to rent or buy. The program will contain one playing session a week. At first, Olga is very sceptical, but then she suddenly remembers her grandson playing a similar game in her living room a couple of months ago. She is thinking that even though the game looks very intimidating, she will get assistance by the physiotherapist which makes it less scary. This can also be a nice way for her to meet other people. She decides to sign up, but has one concern. How will she get to the clinic? The physiotherapist tells her that as part of the program, they will offer the participants transportation to the meetings until they feel confident getting there themselves. The main goal for the game is to strengthen muscles and improve balance, so after some sessions the participants should see improvements. \\ \\
One week later, Olga visits her first meeting. None of the participants in her group have tried the game before, so everyone gets a thorough introduction. Then they start playing. Olga thinks the game is self-explanatory and very easy to understand, and she gets through the first level without any problem. A physiotherapist is watching them at all times and is guiding them through the game. After the session is over, Olga is tired, but she feels good. What she liked most about the game was how fun it was to compete with the other participant, who motivated and engaged each other. Olga likes the way the Norwegian health care system is heading. She now feels more seen and better taken care of than she has ever felt before. Olga is already looking forward to the next session.


\cleardoublepage
\chapter{Health Politics in Norway Today}
The government states that it is a public responsibility to promote health and prevent diseases to make sure that the population gets the care they need. The overall goal is to get a healthier population. Strong health is necessary for an individual to acquire good quality of life. It is also important for the society, especially economically. There is a huge amount of elderly ahead of us, and the society has to prepare for that. In Norway there is a goal to offer everyone in the need of it a place in care homes by 2015.  To be able to meet all the requirements set, there is a need for a change in the health sector in Norway.\\ \\
In this chapter we will describe some of the goals for the future of the Norwegian health sector. This is done for us to be able to see the potential of this game within the health care system.  It is also necessary to look at different exercise programs existing today, which will help us understand how and where Cyberlab's exergame can be used. This will be provided in the last section. 

\section{Samhandlingsreformen}
"Samhandlingsreformen" \cite{budsjett}\cite{regjering}, from now called "the reform", presents a new way to organize the health services in Norway and was put in action 01.01.2012. In the reform there is an increased focus on prevention, early intervention and close collaboration between different entities. The health services should be offered closer to where people live and there should be offered more comprehended and coordinated treatment. Welfare technology shall support these strategies where possible (described in Section 3.2). Technical solutions and methods can make it possible to treat more patients in a better way. \\ \\
Every 75-year old are offered supervision to promote health and own coping. Every person who is in need of health care services that can be deployed in their own house should be provided this. The services offered in the private home needs to be improved. If the services in the private home could be improved to a level where it enables the user to live in their own home for three more months, it will be equal to 10 percent of the capacity in care homes. This will depend on better offers outside the clinics.\\ \\ The main focus in the health promoting work will be on "hverdagsrehabilitering", from now called  "everyday rehabilitation". This means that a person will get health care first after an evaluation that proves that he or she actually has a need for rehabilitation. The target group is the ones with moderate limitations in functional level. The goal is to postpone their need for extensive help and help them to achieve a dependent everyday life.  "Everyday rehabilitation" has the home care staff as a basis and physiotherapy and ergotherapy as "engine". If patients are considered to be in the need of help and support in their own home, they will first meet with a physiotherapist or ergotherapist, who will examine them and evaluate their need. The new strategy for the health care system just described is based on the "Fredericia-model". This model is developed in the Danish municipal "Frederica" and is about how physical, social and cognitive capabilities can be maintained and improved, so that functional disabilities in the older generation will be postponed. The experiences from this model have been very positive. 30 percent receive "everyday rehabilitation" instead of ordinary home care. Approximately 45 percent of these ended the rehabilitation process with no need of any further help, and 40 percent ended the rehabilitation process with need of less help than one can assume they normally would need \cite{budsjett}\cite{regjering}.
\section{Welfare Technology}
As a part of the new reform, welfare technology should be implemented where possible. Welfare Technology can be defined as: "Technological assistance that contributes to increased safety, social interaction, mobility, and physical and cultural activity. In addition welfare technology can help to strengthen an individual's ability to be independent despite sickness and social, mental or physical disabilities. Also it can work as technological support for relatives, as well as contribute to improving the services offered, when it comes to utilization of resources, availability and quality. In many cases welfare technology can prevent the need for health services or admission to an institution" (Translated from Norwegian from \cite{welfare}).\\ \\
Today there is huge attention around the topic of welfare technology in Norway. It is seen as an important tool in the future demographic challenge and in the health promoting work.  The goal is that the need for health services should be decreased and that people can take care of themselves longer. This means that there is a need for services that can be implemented in people's home. An example can be a technological tool to prevent fall and loneliness.  The use of welfare technology can improve the services offered, increase the flexibility and make it easier to interact with different actors. This introduces a new arena for innovation and value creation, and can give rise to positive socio-economic effects \cite{welfare}. \\ \\
The market for welfare technology is very immature. There are no initiators that can make sure it will be established public and private demand. It will be the different municipals' work to establish a public demand. There is a need for robust solutions when it comes to customization, entry level, support and maintenance. To find good solutions, there is a need for more developed products that can be tested \cite{welfare}. \\ \\
The Public Health Department in Norway, "Helsedirektoratet", recommends that laws for the use of welfare technology should be established \cite{welfare}. There are some new aspects that have to be taken into account when integrating technology into systems containing detailed personal information. The use of technology introduces some privacy issues, where for instance some systems will contain  sensitive information about the patients. This suggest that there should be a focus on making secure systems that will preserve people’s privacy. \\ \\
The new reform together with the increased focus on the use of welfare technology suggests that an exergame can serve as a tool in the health promoting work. We will now look closer into existing training programs to see where the game could be implemented.

\section{Fall Prevention and Rehabilitation Today}
From what we have learned from the two previous sections, we see physiotherapy as an interesting area. Physiotherapists are the main actors in the work of prevention and rehabilitation and are one of the first actors elderly meet with when they are facing a physical problem. This suggests that physiotherapy might be the right arena for the exergame. Therefore, we will look a little deeper into the subject of physiotherapy. \\ \\“A physical therapist seeks to identify and maximize quality of life and movement potential through prevention, intervention (treatment), promotion, habilitation, and rehabilitation“ \cite{physiocite}.\\ \\
Physiotherapy is a science with focus on body, movement and functionality to maintain, recover and improve physical health. The theoretical basis is grounded on knowledge about anatomy, neuroscience, and physiology. In addition to this, practical and clinical knowledge contribute to evaluation of how injuries and pain can be treated and prevented. Physiotherapists are traditionally working in municipalities, hospitals and in private institutions, and they are working with both individual treatment and treatment in groups. The goal for physiotherapists is to make their patients' daily activities easier to manage, which is done by using manual techniques, exercises and technical methods \cite{physiotherapy1}\cite{physiotherapy2}.  \\ \\
Elderly people consult physiotherapists in terms of rehabilitation after surgery, after a fall, stroke or other injuries, or when they feel health problems make it hard to perform everyday tasks. Physiotherapy usually includes exercise with focus on increasing the patients’ flexibility, endurance and strength. Physiotherapists set up customized training for each patient according to what kind of needs they have. Unfortunately, the time a patient spends with the physiotherapists per week is not sufficient in order to become stronger or to recover from injuries \cite{gruppetrening-trheim}. Therefore, there is a need for patients to perform additional training outside these hours. The physiotherapists may give their patients exercises they can practice at home. However, not everyone is motivated to exercise on their own and many skip the weekly exercise that is scheduled for them \cite{physiotherapy2}. 


\section{Existing Exercise Programs}
To prevent developing disabilities, elderly should regularly perform a training program that strengthen their muscles, improve balance, coordination, endurance and mobility \cite{gruppetrening-trheim}. We tried to find out if there are offered training programs for elderly. For convenience we only looked at training programs in Trondheim. We found that there have been established various fitness groups to become physically stronger and to achieve better balance. These fitness groups find place at various locations around Trondheim and are offered 1-2 hours one day a week. In addition there exist senior dance, walking groups and water gymnastics \cite{trim}. These activities are good initiatives, but when one problem is that elderly are afraid to go outside, how will they manage to engage in these fitness groups? It is also shown that physical activity only once a week is not enough to increase physical strength \cite{gruppetrening-trheim}. Regular physical activity is the key to become physically stronger and obtain better balance. \\ \\
Trondheim municipality did a study where they provided a once a week group training program for elderly. Their study showed that training once a week did not improve physical function for the participants, but the participants expressed that they were less afraid of falling after starting with the group training. The study suggests that this kind of program should be combined with home training programs or other extra physical training offerings. At the same time, the study shows that exercise once a week have some positive effect on the elderly \cite{gruppetrening-trheim}. \\ \\
We found that there are already some training programs offered for elderly that can be implemented in their home:\\ \\
The \emph{"Otago"-program} is a program developed as a home training program for elderly to prevent falls. It consists of exercises that take about 30 minutes to complete which should be performed three times a week in addition to a walk twice a week. Each customer receives a booklet with instructions for the individual exercises prescribed in addition to ankle cuff weights. The participants need to record the days they complete the program for follow-up purposes. For follow-up, an instructor should do home visits every six months and telephone them every month. The instructor can then increase the difficulty in the prescribed exercises for each individual. The program has been tested and evaluated for 1016 home living people aged 65 to 97. The program was shown to reduce falls and fall related injuries with 35 percent, with the highest effect on those over 80 years old and those that have had a previous fall. The participants experienced improved strength and balance, as well as they maintained their confidence so it was easier for them to do everyday activities without being afraid of falling \cite{otago} \cite{gruppetrening-trheim}.\\ \\
\emph{\ac{fame}} is an exercise program consisting of tailored group and home-based exercises and builds on the core exercises from the "Otago"-program.  There are a total of three group training sessions per week, in addition to two home-training sessions. The exercise intervention is designed to improve participants' dynamic balance and core and leg strength.  In the United Kingdom a study was done where they examined the effectiveness of this program for home-living women aged 65 or older who had already fallen 3 or more times within the previous year. After using FaME for 36 weeks, the fall rate was reduced by one-third. The conclusion was that the exercise program should last for at least 36 weeks including at least 2 hours of training per week. For progression it is important that the intensity, resistance, and weight are continually increased \cite{fame}.\\ \\
\emph{{Ø}velsesbanken} is a Scandinavian project providing a user profile with different training programs. The different exercises are developed from the two previous described concepts and other relevant studies on balance and exercising for elderly. The program gives an idea on how you can put together an exercise program customized for each individual. It is primarily made as a tool for physiotherapists for putting together training programs for their patients to do at home. How we see it, it can also be used as a tool for each individual to make their own program, because you can also log in as a private user and make your own program. The program offers the user a choice of different exercises that can be added to an exercise program. When all exercises are chosen, PDF-files can be printed with pictures and descriptions of the exercises, or the user can read them from the computer screen. It is an easy, self-explanatory and straightforward program to use. {Ø}velsesbanken is in use in Scandinavia and the summer of 2012 it had reached 4300 users \cite{ovelsesbank}.\\ \\
We have learned from this chapter that there are already existing projects and training programs with focus on elderly and their physical health. It is shown that exercise only once a week is not sufficient to improve physical health, but that it has positive effect on self-confidence in elderly \cite{gruppetrening-trheim}.  However, with supplements, like additional home training programs, improvements have been seen. One serious challenge is to motivate the elderly to exercise. We believe that "boring" exercises on a piece of paper are not motivating and that something more fun and entertaining could improve the motivating factor. The issue about the training groups offered today is that many elderly are afraid of leaving their house, and therefore may not attend the weekly meetings. Still, we see a potential for the exergame to be implemented in this kind of training groups. Maybe the introduction of a new and alternative training method would encourage the older population to attend these sessions? Offering this kind of sessions several days a week, could improve the participants' health and thus reduce their fear of leaving their house.  Another arena for the exergame could be in peoples home. We believe this will be a more motivating way of exercising on their own, as well as it will ease the workload of the instructor who can remotely monitor their patient instead of physically visit them. However, we are sceptical to the introduction of a tool like this in peoples' home, since this kind of technology is unfamiliar to many in this age group. This suggest that a starting point for this game should be in the health sector or in training groups.  The new focus on prevention and early intervention with the use of welfare technology, suggest that there is a potential for an exergame in this market. \\ \\
In the next chapter we will look into the technological aspects around the game and we will look at research done on the use of games for exercise and rehabilitation.


\cleardoublepage
\chapter{Technological Aspects}
To be able to understand the technological aspects of this game and the game's potential, we have to learn more about the technology it will be build upon.  The purpose of this chapter is to get a fundamental understanding of what video games, and in particular exergames, are, and how these type of games are being used today. We will look into different types of technologies related to exergames, as well as examples of exergames used for exercise an rehabilitation. Based on this, we will provide a brief evaluation of Cyberlab's choice of technology and its potential a tool for exercise and rehabilitation.  \\ \\ 
First, we will describe video games and exergames. Second, we will provide a brief description on different video game providers. Third, We will look at some related work where training and video gaming are brought together. Finally, we give our evaluation and recommendation.\\ \\

\section{Computer and Video Games}
"Video games are electronic, interactive games known for their vibrant colors, sound effects, and complex graphics" \cite{videogamedef}. Characters or objects are controlled by hand held game controllers, or by pure body movement captured by sensors or motion controllers. Since the first computer game was developed in 1952 there has been a tremendous evolution in the computer and video game market. Todays market consist of an endless amount of various computer games, video games and video game consoles, and this type of technology are widely used all over the world. It has been developed a game for almost every need and interest, and video games are used for many different purposes, like education and learning, exercising or just pure entertainment. In this section we will describe the history of computer and video games, and gaming statistics. \\ \\
The single-player game, “OXO”, was created in 1952 by A.S Douglas as the first graphical computer game. It was based on a version of Tic-Tac-Toe and designed for academic purposes.  Douglas wrote a PhD degree on Human-Computer interaction, and used feedback from the electronic "OXO" in his work \cite{abouthiginbotham}. Ralph Baer, a German-born television engineer, designed in 1967 a video game console for use on standard television, which was the first of its kind. The game was played by connecting to consoles to a television, which were controlled by two gamers. "Chase" was the name of the game, where the purpose was to chase each other by controlling two squares \cite{videogameHistory}. Various features were added to this idea, and this ended up in a series of 12 games known as the Brown Box. Baer introduced his idea to Magnavox, and in 1972 the Magnavox Odyssey was produced as the first commercial video game. The Odyssey did not get very popular and faced very low sales numbers. However, the market for video games totally changed in 1985 when the Nintendo Entertainment System was released. Retailers were sceptic to a new console so soon after the Odyssey failure, but the games Nintendo introduced got popular. Nintendo broke sale records and become the bestselling console in video game history \cite{consoleHistory}. \\ \\
The Nintendo Wii is one of many gaming consoles, and it is worldwide very popular. In 2010 it had sold over 30 million units in the US and in Japan there had been sold almost 10 million. These numbers combined with the international market gave a total sale of Nintendo Wii of 65.32 million units (see Figure \ref{fig:ConsoleWarWii}. Since Xbox 360 was released in 2006 it has experienced strong sales number, and it is today with Nintendo Wii one of the leading gaming consoles on the market. Figure \ref{fig:XboxWiiSales} shows the total sale of consoles and games worldwide in 2012, where we can see that Xbox 360 and Nintendo Wii have sold the extreme amount of approximately 700 million and 820 million games. However, the bestselling console ever is the Sony PlayStation 2, with over 138 million units sold. The bestselling video game series is the Mario franchise, with a sales number of over 225 million games \cite{statistics2012}. \\ \\

\begin{figure}[h!]
\begin{center}
\includegraphics[scale=0.5]{consolewarwii}
\caption[Nitendo Wii console sale]{Nintendo Wii console sale 2010 \cite{statistics2012}}
\label{fig:ConsoleWarWii}
\end{center}
\end{figure}

\begin{figure}[h!]
\begin{center}
\includegraphics[scale=0.7]{xboxwiisales}
\caption[Nitendo Wii and Xbox 360 sales]{Nintendo Wii and Xbox 360 approximate total console and software sale in 2012 \cite{nintendolife} \cite{microsoftxbox} \cite{vgchartzxbox} \cite{vgchartzwii} \cite{vgchartzhardware}}
\label{fig:XboxWiiSales}
\end{center}
\end{figure}

\begin{figure}
\begin{center}
\includegraphics[scale=0.5]{gamersus}
\caption[Gamer age distribution]{Gamer age distribution in the US in 2010 [modified from \cite{statistics2012}]}
\label{fig:GamersUS}
\end{center}
\end{figure}
DFC Intelligence is a marked research and consulting firm which focus on interactive entertainment and game markets. The global market for video games experienced revenue of 67 billion dollars in 2012, and in DFC Intelligence's new reports they forecast that the global video game market is expected to reach 82 billion dollars in 2017. This number includes revenue from console hardware and software, PC games and games for mobile devices \cite{videogameforcast} \cite{aboutdfcint}.\\ \\
In the last 30 years there has been a great evolution in video games. Video games have become widespread entertainment, and in 2010 as much as 65 percent of all households in the US played video games. The majority of these gamers are in the age of 18-49, and an interesting fact is that there are a greater share of gamers over the age of 50 than there are players under 18 \cite{statistics2010} \cite{statistics2012}, see Figure \ref{fig:GamersUS}. Another surprising fact about the gaming statistics in the US is that as much as 47 percent of all gamers are women, and that there are more female gamers over 18 than male gamers under the age of 17 \cite{statistics2012real}. \\ \\
Video games has become a great part of the everyday life, when gamers play about 8 - 18 hours a week \cite{statistics2010} \cite{statistics2012}. Games that has become very attractive are them involving social interaction. A third of all gamers plays social games and 78 percent are playing with others in-person or online. We can see this from the video game market, as the two best selling video games in 2011 is Call of Duty: Modern Warfare 3 and Just Dance 3, which both are social games where you have the possibility to play with others \cite{statistics2012real}. \\ \\ 
\begin{figure}
\begin{center}
\includegraphics[scale=0.7]{gendergameplayers}
\caption[Gender of gameplayer]{Video game statistics from 2012 shows that as much as 47 percent of gamers are women \cite{statistics2012real}]}
\label{fig:GenderGamePlayers}
\end{center}
\end{figure}
"Norsk mediebarometer", a report containing statistics around the topic of media use in Norway, show that 17 percent of the Norwegian population plays computer or video games on an average day in 2011. This includes not only children and teenagers; also a great part of elderly has started to use computer or video games. 8 percent of the population in the age 45 - 79 years use this kind of technologies on an average day, see Figure \ref{fig:GameStatisticsNorway}, where females are the most active gamers. Norway does not have the same share of elderly gamers as the US, but statistics shows that use of computer and video games in Norway has increased from 5 percent in 2010, and this measurement is the highest share ever recorded \cite{ssb2010} \cite{ssb2011}. One thing worth mentioning about the report of media use in Norway, is that when it came to looking at video games alone, 0 percent of the population in the age 45 - 79 said that they played video games on an average day in 2011. \cite{ssb2011}
\begin{figure}
\begin{center}
\includegraphics[scale=0.9]{gamestatisticsnorway}
\caption[Use of computer or video games, Norway, 2011]{Percentage of the Norwegian population that use computer or video games on an average day in 2011, sorted by age \cite{ssb2011}}
\label{fig:GameStatisticsNorway}
\end{center}
\end{figure}
       
\section{Exercise Games}
The new generation of video games combining game play and physical activity is called exercise games, or "exergames". Exergames use technology like motion sensors and remote control to track body movement. This requires the player to get up from the coach and physically move their body to be able to play the game, which stimulates exercise. Exergames is proved to be motivating because it is fun, accessible and easy to understand, and it has shown promise in effecting users health in a positive direction \cite{promotingexercise}. The combination of movement, amusement and social interaction provides exergaming great potential for new business opportunities for the entertainment, recreation and healthcare sectors \cite{gamingforhealth}. Today there exist numerous types of games and technologies related to exergames, where Nintendo Wii, Dance Dance Revolution, PlayStation Eye Toy and Xbox Kinect are some of the more familiar technologies. \\ \\
Due to the growing interests one has seen it as relevant to study the use of exergames in regard of health and education. The technology these games provide can help improving peoples health in a new and interesting way \cite{gamingforhealth}. In the past years exergames research has increased dramatically, which indicates that it will continue to do so \cite{chamberlin2008exergames}. Research shows tremendous promise in academic and physical progress of youth using exergames. Exergames has also shown an important social aspect, because of the possibility to play with others. This may especially be entertaining for elderly who are often alone and experience loneliness as a part of the everyday life \cite{exergamesforelderly}. \\ \\
The health sector is now more focused on prevention of illness instead of treatment, where this type of research can contribute to use of exergames in health care \cite{gamingforhealth}. Exergames has shown promise for use in rehabilitation, e.g. after stroke or damage to the spine \cite{lange2011development}. Exergames provides users with a feeling of accomplishment by reaching goals, completing exercises and being physically active, which again increases the users mood \cite{staiano2011exergames}. The fun and challenges the game provides could take focus away from boredom and physical pain which makes it appropriate as an exercise or rehabilitation tool \cite{exergamesforelderly}. Games like Wii Sports and Dance Dance Revolution were designed to encourage physical activity, but many other currently available exergames were not designed for this purpose. Few commercial games are suitable for the focused, controlled exercise required for therapy \cite{lange2011development}. Games today are too complicated, go too fast and are too difficult to handle for the elderly, in additition they have too complex and cumbersome consoles \cite{exergamesforelderly}. However, the popularity of exergames and the increasing customer appeal will improve design principles and physical requirements in the future\cite{chamberlin2008exergames}. \\ \\
We will now describe some of the different video game consoles that can be used for exergaming.
\subsection{Dance Dance Revolution}
\ac{ddr} is a series of video games created by Konami Corporation’s Bemani music games division. \ac{ddr} is a rhythmic dance simulation game and was first released as an arcade game in 1998. In few years it became very popular, and the game has had its appearance on several game console systems like Sony PlayStation, Nintendo 64, Microsoft Xbox and Nintendo GameCube \cite{bogost2005rhetoric}. DDR uses a touch-sensitive dance pad with sensors to register movements, where one shall press the right sensors in proper time with electronic dance music. Arrows on-screen gives direction on how and when to move around. The DDR games have varying difficulty, requiring different levels of physical activity. GetUpMove.com is an information website about the use of PlayStation Dance Dance Revolution as a weight loss tool. This site was launched in 2004, and one of the highlighted stories was about a young woman who lost about 95 pounds by using DDR as an exercise tool. This and similar stories got widespread exposure, and consumers started to buy DDR solely for the purpose of exercise \cite{bogost2005rhetoric}. In 2003, 5 years after the first release, Konami announce that DDR has reached a total sale of 6,5 million units worldwide \cite{gamespot}. 8 years later, in 2011, the number of sold units had reached over 13 million, which is  about 1 million units sold every year since the first release in 1998 \cite{gaygamer}. 
\begin{figure}[h!]
\begin{center}
\includegraphics[scale=0.5]{ddrpad}
\caption[Dance Dance Revolution]{The Dance Dance Revolution dance pad}
\label{fig:DDRPad}
\end{center}
\end{figure}

\subsection{PlayStation EyeToy}
In the early 2000s the PlayStation 2 EyeToy was released by Sony Inc. It was the first in this category of games to introduce a device that could translate human motions into a controller input and allow players to physically interact with virtual objects using their own body and without being connected to wires. Human body movements are translated real-time into the controller input by a USB camera  and can also map the player’s face onto in-game characters. Eye Toy is easy to set up and its applications offer a lot of different environment and can be played by one or more players. \cite{eyetoy}
\begin{figure}[h!]
\begin{center}
\includegraphics[scale=0.3]{pseyetoy}
\caption[PlayStation EyeToy]{The PlayStation EyeToy}
\label{fig:PSEyetoy}
\end{center}
\end{figure}

\subsection{PlayStation Move}
PlayStation Move was released in September 2010. The PlayStation Move’s interface consists of the Move Eye, a RGB camera with directive microphones, and the Motion Controller, a wand with an illuminating sphere attached to it. The camera can detect the sphere and determine where the wand is, which allows the players to interact with the game through motion and position. The sphere attached to the wand helps the camera to determine the distance from the wand to the camera and to track the controllers position in three dimensions. The wand is equipped with a three-axis accelerometer and a three-axis gyro sensor which are used to track rotation in overall motion and can also be used to detect if the wand is out of range (i.e. hidden behind the player back). \cite{comparison} Up to four wands are supported at one time, which makes it possible for four players to play together. The color of the sphere can be changed to any color and is usually used to show which player is active and to give visual feedback \cite{ppmove}. The SDK is not made public, so its difficult for a third party to make original applications \cite{comparison}. 
\begin{figure}[h!]
\begin{center}
\includegraphics[scale=0.3]{PSmove}
\caption[PlayStation Move]{The PlayStation Move}
\label{fig:PSMove}
\end{center}
\end{figure}

\subsection{Nintendo Wii}
Nintendo Wii was released in 2006 as the first motion sensor game. Only one year and 20 million units sold later, it became the market leader of that times generation of consoles. It consists of a Wii remote, which is the primary controller and a secondary controller called Nunchuk. The Nunchuk is connected at the bottom of the Wii remote control \cite{hackingwii}. The Wii remote contains 12 buttons, a 3-axis accelerometer, a high-resolution highspeed IR camera, a speaker, a vibration motor, and wireless Bluetooth connectivity.
The IR camera is placed on the remote's tip and can track up to four simultaneous IR light with high resolution and high speed. The accelerometer within the remote control provides the Wi remote’s motion-sensing capability. The 12 buttons on the remote are arranged symmetric so that both hands can be utilized. A vibrator motor, LED lights and a small speaker are used for different kinds of user feedback, like varying light strength and sound-effects. The four LED lights are also used to indicate the different players' ID. Communication is sent over the wireless Bluetooth connections, which enables up to four controllers to be connected at the same time.  The users of Nintendo Wii can make their own personal profile, called Mii, where the data of the player will be directly connected up on the remote used  \cite{hackingwii} \cite{whatiswii}. By November 2012 over 97 million Wii consoles were sold \cite{vgchartzhardware}.  To complete the original system with improved accuracy and response time, Nintendo made an enhanced version, Wii Motion Plus, which was released in November 2009 \cite{consoles}. There are several SDKs for Nintendo Wii open, which makes it possible for a third party to develop applications which utilize the controller \cite{comparison}. 
\begin{figure}[h!]
\begin{center}
\includegraphics[scale=0.3]{nintendowii}
\caption[Nitendo Wii]{The Nintendo Wii}
\label{fig:NintendoWii}
\end{center}
\end{figure}

\subsection{Wii Fit Plus}
Wii Fit Plus is a video game created for the Wii console. One of its main add-on accessories is the Wii Balance Board. The board can read your body movements and give them back on the screen as you are playing, by the use of multiple pressure sensors contained in the board \cite{whatiswiifit}. The board has a area of 55,1 cm {*} 31,6 cm. A third party can also build applications for the balance board using the SDK WiimoteLib \cite{comparison}. It has been shown that game-based balance programs like Wii Balance Board compared to traditional training is easier, more motivating and more enjoyable \cite{taylor2011activity}.
\begin{figure}[h!]
\begin{center}
\includegraphics[scale=0.4]{wiibalance}
\caption[Wii Balance Board]{The Wii Balance Board}
\label{fig:WiiBalanceBoard}
\end{center}
\end{figure}

\subsection{Microsoft Kinect for Xbox 360 and Windows}
Microsoft Kinect was released in 2010 and became quickly extremely popular. Only 25 days after its release it had sold 2.5 million units and by January 2012 Xbox 360 had sold over 66 million consoles and more than 18 million Kinect motion sensors \cite{consoles} \cite{kinectsold}. Kinect is a flexible low-cost motion sensor that can track human motions and it can be used with the Xbox 360 game console or with a Windows machine.  The sensor is webcam-based, which enables the user to play and interact with the game without physically holding a sensor device. Instead the player can interact with the game console through a natural user interface by moving their body and by using voice commands. The device have the ability to give full-body 3D motion capture capabilities and gesture recognition by help of its RGB camera and depth sensor \cite{kinect}. One advantage with Kinect is that it has an interface that senses players various motions and it also senses other objects in the field, which makes a natural environment where the players can interact with virtual objects in the real world. \cite{comparison}. The Kinect sensor for Windows is designed to operate on computer running Windows 7, Windows 8, Windows Embedded Standard 7, and Windows Embedded POSReady 7. All the users need is the Kinect sensor, a computer and a Kinect for Windows application. Kinect for Windows SDK was released in June 2011 and enables developers to build Kinect applications with C++, C\# or Visual Basic using Microsoft Visual Studio 2010. This enables any third party to develop Kinect for Windows applications. \cite{kinectwindows}.\\ \\
We will in the next section review research where video games, like those just described, has been used for health-related purposes. 
\begin{figure}[h!]
\begin{center}
\includegraphics[scale=0.3]{kinect}
\caption[Kinect Sensor]{The Microsoft Kinect Sensor}
\label{fig:KinectSensor}
\end{center}
\end{figure}

\section{Using Exergames for Fall Prevention and Rehabilitation: A background Study}
There has been done numerous of research on the use of video games for exercise and rehabilitation. The focus are mainly on the physical, social and cognitive benefits. In this section we will review the interesting findings we did with emphasis on the physical and psychological benefits. The physical part includes both "normal" exercise and rehabilitation. The social benefits are important aspects for the target group of Cyberlab's exergame, because they are people that may not have that much social interactions, often because they are afraid of leaving own house. The research review provided here, will also relate to the previous described different technologies. Dette avsnittet må fikses litt på ! \\ \\
Primack et al did a broad study of video games in the context of improving health, where they analyzed 1452 articles with the topic of video games.  For articles to be included in the study, they had to meet criterias for inclusion, like involving use of video games, showing health-promoting, clinically relevant outcomes, and being a RCT (randomized controlled trial). 38 articles met these criteria and was therefore included in the study. Studies was also considered a “health-promoting, clinically relevant health consequence” if they showed effect on health care providers to improve their patients health. Based on the purpose of the video game, each of the 38 articles was assigned to one of six categories of improvement; physical therapy, psychological therapy, health education, disease self-management, distract from discomfort, and increase of physical activity. In all of the categories, beside the category of self-management, 67 - 100 percent of the studies showed positive outcomes with the use of video games. The study with the best outcome was distraction from discomfort, where all of the outcomes showed positive effect. However, the result of these studies showed that the most common positive outcome was related to physical therapy, in e.g. rehabilitation after a stroke, and physiological therapy, in e.g. reducing post-trauma. The purpose of this study was to look into the ability of video games to address a variety of health conditions, and results showed that video games can have a positive effect in a variety of categories in all age groups. Video games has also showed positive outcomes when it comes to health-care personnel using this type of technology to train patients. Video games for elderly, individuals in the age group 50 - 80, often focus on age-related changes like decrease of balance and cognitive decline. However, the study showed that the age group with the best opportunity to improve health is in the age group of 30 - 50. The fact that video games shows potential health-related benefits is an important finding, as it represents a huge industry for the entertainment sector and as the use of video games has become very popular among people of all ages. \cite{roleofvideogames} \\ \\
Taylor et al. \cite{taylor2011activity} did a review on different studies related to gaming systems in exercise and rehabilitation. We will summarize some of their interesting findings here. From the studies they reviewed they found a trend; the  \ac{ee} while playing Wii was greater than when doing sedentary activities, but not greater than brisk walking. This suggests that playing Wii sports could not replace real sports activities. Playing \ac{ddr} on the other hand, maximum heart rate and oxygen consumption were greater compared with Wii sports, suggesting that DDR can substitute physical activity, based on \ac{ascm} guidelines for physical activity. In their research they also found a study on what attitudes people have against \ac{ddr} to encourage exercise. 40 postmenstrual women, aged 45-75 years old were asked. The overall attitude was positive; The game was fun and it gave potential to improve coordination. However, they also expressed a concern about a long learning process. It was also found that playing against a human gave greater arousal ratings and physiological responses to gaming than when playing against a computer, which benefit the enjoyment. This aspect is important to take into consideration when setting up a gaming environment for the older population, because they may benefit from the social interaction. Some games have already shown the potential for rehabilitation , like the EyeToy and the Wii. The main resons why these type of games a suitable are that the have the ability to increase motivation and produce distraction from daily, boring and painful treatments. Wii is seen as an attractive game for rehabilitation, both at home or in institutions. Wii is actually already in use within the National Health Service in UK and is commonly used for the elderly and patients with pathologies.  \cite{taylor2011activity} \\ \\
Another study they found was that non-disabled  elderly (70 +/- 5,7 years) was positive to the EyeToy. They enjoyed it and found it easy to use. For patients with stroke it appeared to be less suitable, which could be even worse if they had to hold on to a controller, for example with Wii. This suggests that EyeToy is more suitable for patients with stroke than games with remote controllers.  
Even though these type of games are initially meant as entertainment systems, there are a number of studies that have used the hardware and developed software to turn for example the Wii into a useful rehabilitation tool. The importance of these games are entertainment that motivates for actual sports. This is very important in for example rehabilitation. \cite{taylor2011activity} \\ \\
Staiano and Calvert write about how exergames are more and more used in the health sector. Gaming consoles are already integrated into equipments at gyms and health clubs. An example is Concept 2’s rowing machine. Here the people exercising are motivated through competition and through virtual trainers who monitor their progress and encouraging them to proceed to the next level. Feedback from a virtual trainer is also offered in Wii Fit. Also some schools are starting to integrate games into their curriculum. In all of West Virginia’s 765 public schools they have integrated DDR in their physical education. This has proven to be very effective and popular and some students lost 5-10 pounds after playing DDR daily. \cite{staiano2011exergames}. \\ \\
Williams et al. did a study to see if exergames, more specifically Nintendo Wii Fit, was an applicable type of exercise to reduce the falling statistics of community-dwelling people over 70 years. A group who attended Wii Fit exercise sessions was compared with a group who went to a local falls group. 77 percent of the participants said that if the exercise programme was more available, people like themselves would use it. 92 percent of the participants expressed that they wanted to exercise with the Wii Fit in the future, while 61 percent would choose to exercise with the Wii Fit rather than attend a falls group. An improvement in \ac{bbs}\footnote{Berg Balance Scale (BBS), a performance based measure using 14 activities of daily living (range 0-56)\cite{excell}} after 4 weeks was seen in the group that played Wii Fit, meaning that there is a potential to improve balance in this population. Despite this, there was no change in \ac{fesi})\footnote{The Falls Efficacy Scale - International (FES-I). This scale measures confidence in performing a range of activities of daily living without falling \cite{fes}} after 4 weeks. The qualitative data for the group that played Wii Fit showed improved confidence for the participants. The conclusion of the study was that Wii Fit is acceptable in  older people with a history of falls and that it has the potential to improve balance and confidence. Further work has to be done to find and develop an acceptable exercise programme with the potential to improve balance in older individuals. \cite{excell}\\ \\
Chang et al. did a study where they prototyped a Kinect game that was designed to help motivate people with motor disabilities to do their exercise more frequently and to improve the motor proficiency and quality of life. Because of the inconvenience of having to wear sensors in some of the other relevant technologies, Chang et al. chose to use Kinect. They developed a game, called "Kinerehab", that was meant to assist therapists in rehabilitating students in public school settings. To detect the students’ movements Kinerehab uses image processing technology of Kinect. To engage and motivate the student for physical rehabilitation, the system is made with an interactive interface that has both audio and video feedback. For making it easy for therapists to review the progress of each students quickly, the system also includes details of students rehabilitation conditions which is automatically recorded in the system. Two students, a 16 year old girl diagnosed with  muscle atrophy and insufficient muscle endurance, and a 17 year old boy diagnosed with cerebral palsy, were chosen to participate in the study. The girl used a wheelchair and could only stand with assistance. The study included two phases: a baseline phase  where no assistive technology was applied, and the intervention phase where the Kinerehab was used. Both phases were done twice, beginning with the baseline phase, continuing with the intervention phase and so on. In both phases the same exercises were done. The result showed that both participants increased the number of correct movements significantly in the intervention phase. On average the number of correct movements was 49 in the first baseline phase (5 sessions), while 170 in the first intervention phase (11 sessions). Both students indicated that the game motivated them to do the exercises and that they wanted to continue using it. The therapists said it would reduce their workload a lot. This suggests that Kinect can be a viable rehabilitation tool, but further work, where more people with disabilities participate, should be done. \cite{kinect} \\ \\ 
Garcia et al has developed a game-based exercise on a Microsoft Kinect platform, where the game has stepping exercises specially designed for elderly. Microsoft Kinect was chosen as platform because of the possibility to capture movement without the use om gaming consoles or wearable sensors. By freeing their hands elderly can lower the impact of damage if they were to fall during the exercise. Kinect also makes is possible to give full focus to the exercise, and not thinking about handling different consoles. The purpose of this Kinect-based exergame is increasing physical strength and improving balance. In the context of developing this game Garcia et al did a review of existing performance-based tests for prevention of fall in older people. They decided to use a \ac{csrt} task. The CSRT task has the ability to measure functions as balance and strength, but also cognitive functions as attention and reaction speed. The CSRT had been tested and validated by a group of elderly. The CSRT task involves people standing on the floor surrounded by four panels. The panels will light up randomly, where the participants shall move their feet in accordance to the color of the light. A green light signal that one step onto the green light, where the participant shall stand still on red light signal. The exercise was repeated 20 times, and reaction time was measured. The importance of the task was to identify the participant's step length and fall risk. This game-based system is based upon Kinect's advanced technologies, which makes it possible to continuously track body movement and give feedback. Parameters retrieved from the Kinect can be used for measurement of clinical data, which makes it appropriate to use in clinical practice. The first draft of the game is meant to be controlled in a clinical setting with physiotherapists or other medical personnel. \cite{garcia2012exergames} \\ \\
From the different research we can conclude that there is a lot of attention on the use of video games as an exercise and rehabilitation tool, both for the young, and older population. The games are primarily been made for fun and entertainment and are not designed specifically for rehabilitation. There is therefore a need for customized games for this purpose. Based on the study we did on the different technologies available and the research done by others, we conclude that the Kinect sensor is a viable choice of technology for the development of an exergame for elderly. The main reason is the convenience of not having to hold on to a device. In addition it is easy to develop a game for this platform, enabled be the free SDK offered. 






\cleardoublepage
\chapter{Osterwalder's Business Model Ontology}
The goal of our project is to analyse the business potential of the exergame Cyberlab is going to develop. To do this, we have decided to use Osterwalder's Business Model Ontology. We analysed both Alexander Osterwalder's phd thesis \cite{osterwalderthesis} and his textbook "Business Model Generation" \cite{osterwalder} and made our own synthesis of them. The thesis is a thoroughly description, while the book is easy to understand and on a more user friendly level. The foundation of our description lies in the textbook, with some details that we found important from the thesis. In this way the description is well suited for our analysis.\\ \\
Osterwalder defines a business model like this: "A business model describes the rationale of how an organization creates, delivers, and capture values" \cite{osterwalder}. Osterwalder came up with a way to describe business models through nine building blocks. Going through these building blocks allows us to describe and think through the business model of any enterprise by covering four main areas of a business:  Product, Customer Interface, Infrastructure Management and Financial Aspects. The nine different building blocks are: Customer Segments, Value Propositions, Channels, Customer Relationships, Revenue Streams, Key Resources, Key Activities, Key Partnerships and Cost Structure. The nine business model elements are the core of the model, see Figure \ref{fig:TheBusinessModelCanvas}. In this chapter we will go through every of the nine building blocks in more detail. \cite{osterwalder}

\begin{figure}
\label{fig:TheBusinessModelCanvas}
\begin{center}
\includegraphics[angle=90, scale=0.7]{osterwaldersbmmodified}
\caption[The Business Model Canvas]{The Business Model Canvas [modified from \cite{osterwalder}]}
\end{center}
\end{figure}

\newpage
\section{Product}
Product is what the company offers to its customer and how it differentiates itself from its competitors. This area covers the building block Value Proposition \cite{osterwalderthesis}.

\subsection{Value Proposition}
Osterwalder's definition is: "The Value Propositions Building Block describes the bundle of products and services that create value for a specific Customer Segment" \cite{osterwalder}. This is what the organization actually offers to their customer or customer segments and are suppose to satisfy the customers needs. It might be different value propositions for the different Customer Segments. The values can be both quantitative and qualitative, meaning that the value can rely on for example price or on design. The value propositions have to be so good that the organization's defined customer segments turn to them over another company. It can be either something new, an improvement of already existing products or services, customized products and services or simply just helping a customer to get a certain job done. Something to also consider is design and brand. These two aspects are more important in some type of products than other. It is also important to compare price levels with their competitors. A common way to satisfy the needs of the customer is to offer them the same value to a lower price. The firm can also keep-up with the market price, offer luxury goods to a higher price or simply offer a value proposition for free. For the latter, the model is based on an other source of income, for example advertising. \\ \\
There are different ways of creating value for the customer. Reducing the costs will for most customer be experienced as valuable. Also reducing the risk when buying something, by for example offering them one-year guarantee is very satisfactory for customers. Other ways of creating value are to make products and services available for customers that did not have access to them before and to make products and services easier and more convenient to use. \cite{osterwalder}

\section{Customer Interface}
Customer Interface covers everything that have to to with customers: who they are, what kind of relationship the firm has with them and how the firm reach out to them. The three building block covered by this area are thus: Customer, Channel and Customer Relationships. \cite{osterwalderthesis}

\subsection{Customer Segments}
Osterwalder's definition is: "The Customer Segments Building Block defines the different groups of people or organizations an enterprise aims to reach and serve" \cite{osterwalder}. To make a good business, you have to understand who the business are meant to create value for, which is all about segmentation. It is important to carefully choose the most important customers and to focus on them and their needs. A business can have more than one customer segment, but they can not always serve all segments. Therefore a careful valuation has to be done to choose the organizations most important segment(s). \cite{osterwalder} \cite{osterwalderthesis} \\ \\
A firm can deliver a value proposition to different types of customer segments. They can choose to not distinguish between customer segments and rather focus on the mass market, they can distinguish their customers into segments with slightly different needs or problems, or sharpen it even more by targeting a niche market with specialized customers. The firm can also serve unrelated customer segments or even independent customer segments. \cite{osterwalder}

\subsection{Channels}
Osterwalder's definition is: "The Channel Building Block describes how a company communicates with and reaches its Customer Segments to deliver a Value Proposition" \cite{osterwalder}. This is about finding the best and most cost-efficient way of reaching the right customers, at the right place and right time \cite{osterwalderthesis}. We distinguish between five channel phases, shown in figure \ref{fig:ChannelPhases}. A channel should be studied over all these phases. It is important for an organization to think about how the customers want to be reached in all of these phases and how the channels can be integrated with the customers routines. The way the organization communicates with the customers is an important role in the customer experience. The value proposition can be delivered either directly, through for example sales force, or indirectly through intermediaries. They can also be delivered through owned channels, partner channels or a mix of both, see figure  \ref{fig:ChannelTypes}.
 \cite{osterwalderthesis}. \\ \\

\begin{figure}
\begin{center}
\includegraphics[angle=90,scale=0.6]{kjopskjede}
\caption[Channel Phases]{The 5 Channel Phases [modified from \cite{osterwalder}\cite{osterwalderthesis}]}
\label{fig:ChannelPhases}
\end{center}
\end{figure} 

\begin{figure}
\begin{center}
\includegraphics[scale=0.7]{channeltypes}
\caption[Channel Types]{Channel Types [modified from \cite{osterwalder}]}
\label{fig:ChannelTypes}
\end{center}
\end{figure} 

\newpage
\subsection{Customer Relationships}
Osterwalder's definition is: "The Customer Relationships Building Block describes the types of relationships a company establishes with specific Customer Segments" \cite{osterwalder}. The customer relationship is very important for the customers overall experience. This can range from personal assistance, where a real customer representative communicates with the customer, to a more automated service, where typically the customer helps himself,  to a more community based service that allows customer to exchange experiences with each other. For every type of Customer Segments defined, the organization has to keep in mind what kind of relationship the Customer wishes to have. At the same time, the organization has to keep in mind how this relationship is integrated with the rest of their business model and how costly they are. The Customer Relationship is based on customer equity. There are three different customer equity goals: customer acquisition, customer retention and boosting sales (upselling) \cite{osterwalderthesis}.

\begin{itemize}
\renewcommand{\labelitemi}{$\bullet$}
\item \emph{Acquisition:} A company needs customers to do business. The customer acquisition is a very expensive affair and must be carefully managed and evaluated because the relationship developed with its customers will strongly influence the two next equity goals.
\item \emph{Retention:} After acquired customers, a goal should be to retain them. The customer acquisition is usually more expensive than customer retention. Because of this, ways to extend the duration of the relationships between the company and its (profitable) customer should be found. High switching costs is an element that can help retention. This means that the cost of ending the relationship and building a new one is so high that the customer does not want to switch.
\item \emph{Boosting sales (upselling):} This means adding on to the initial sale with additional products and services.
\end{itemize}

\section{Infrastructure Management}
Infrastructure management describes the companies capabilities and resources that are necessary to deliver the value proposition and maintain customer interface. This block also describes who provides and own the capabilities and resources, as well as who executes the activities and the relationship between them \cite{osterwalderthesis}.

\subsection{Key Resources}
Osterwalder's definition is: "The Key Resources Building Block describes the most important assets required to make a business model work" \cite{osterwalder}. This means all the resources you need to make all the 4 described building blocks work. The resources can be physical (e.g. buildings and machines), intellectual (e.g. brands, patents and copyrights), human (e.g. in an industry where knowledge is in particular important) and financial (e.g. cash). The company does not need to have all the resources within their organization, they can also be acquired from outside the company. A Resource can be linked to one or more Activities, described next.

\subsection{Key Activities}
Osterwalder’s definition is: "The Key Activities Building Block describes the most important things a company must do to make its business model work" \cite{osterwalder}. This means all the actions that have to be done to make all the 4 first building blocks described work and to generate profit. The main purpose of a company is the creation of value that customers are willing to pay for. This value is the outcome of a configuration of inside and outside activities and processes. Depending on what kind of company it is the configurations can by categorized as a \emph{Value Chain}, a \emph{Value Shop} or a \emph{Value Netork}. Osterwalder distinguish between primary and support activities. Primary activities are involved in the creation of the value proposition and its marketing and delivery. Support activities are the underlying activities that have to be in-place for the primary activities to take place (e.g. firm infrastructure, technology). All the three different types of configurations have different primary activities, as described in  figures \ref{fig:ValueChain}, \ref{fig:ValueShop} and \ref{fig:ValueNetwork} \cite{osterwalderthesis}:\\ \\
\emph{Value Chain:}
\begin{figure}[h]
\centering
\begin{center}
\includegraphics[scale=0.6]{valuechainnew}
\caption[Value Chain]{Value Chain (5 primary activities) \cite{osterwalderthesis}}
\label{fig:ValueChain}
\end{center}
\end{figure} \\ \\
A value chain describes how a firm creates value from taking an input, transforming it to the final product (refined output), distribute the product  to the customers and maintain the product. At each step there are added value (e.g. production and manufacturing). The main property for a value chain is being cost-efficient.
\newpage
\emph{Value Shop:}
\begin{figure}
\begin{center}
\includegraphics[scale=0.8]{valueshopnew}
\caption[Value Shop]{Value Shop (5 primary activities) \cite{osterwalderthesis}}
\label{fig:ValueShop}
\end{center}
\end{figure}
Value Shop describes how a firm can create value for its customers by understanding their problem and finding a solution for it (e.g. consultancies and doctors). The main property of a value shop is rumour.  \\ \\
\emph{Value Network:}
This is about network effects, which means that the more people a network has the more value it gets. (e.g. banks and telecom operators). It consists of getting potential customers to the network, establishing links between customers and billing for value received, and maintaining and running a physical and information infrastructure so it is ready to serve customers requests. 
\begin{figure}
\begin{center}
\includegraphics[scale=0.7]{valuenetworknew}
\caption[Value Network]{Value Network (3 primary activities) \cite{osterwalderthesis}}
\label{fig:ValueNetwork}
\end{center}
\end{figure}

\subsection{Key Partnerships}
Osterwalder’s definition is: “The Key Partnerships Building Block describes the network of suppliers and partners that make the business model work” \cite{osterwalder}. Not always can a company do everything on their own. The motivation for creating partnerships can be divided in three: 
\\
1. \emph{Optimizing their business model:} Sometimes it is not profitable for a company to own all resources and do everything in-house. Cooperating with other firms can reduce costs and optimize the allocation of resources and activities. 
\\
2. \emph{Reduce risk:} In a very competitive market it can be safer to cooperate with the competitors in one area, even though they are competing in another.
\\
3. \emph{Acquire resources:} Usually it is not very profitable for a company to have all resources and to have the knowledge to do all the activities. Cooperating with other firms by buying/lending resources is often more profitable than having everything in-house. \cite{osterwalder}

\section{Financial Aspects}
All of the other blocks already described influence this last block in the framework, thus this block is an outcome of the rest of the business model configuration. This area covers the Revenue Streams and Cost Structure elements. \cite{osterwalderthesis}

\subsection{Revenue Streams}
Osterwalder’s definition is: “The Revenue Streams Building Block represents the cash a company generates from each Customer Segment” \cite{osterwalder}. This is where the company earns its money. It is important to keep in mind what the customers are willing to pay, as well as what they are currently paying. A firm can have one or more revenue streams where each revenue stream can have different pricing mechanisms, shown in table blabla. There are several ways of generating revenue streams, listed in table \ref{tab:revenue}  \\ \\ 
The pricing mechanism chosen is very important and can make a huge difference on how much revenue that is generated. Osterwalder distinguish between two types of pricing mechanisms: fixed and dynamic pricing, where fixed pricing means that the prices are based on static variables, while dynamic means that prices changes with market conditions. \cite{osterwalder}

\begin{table}
\centering
    \begin{tabular}{|l|l|}
        \hline
       \textbf{Ways to generate revenue} & \textbf{Example}  \\ \hline
       \emph{Asset sale} & Selling a car \\ \hline
       \emph{Usage fee} & Customer pays telecom operator for minutes \\ & spend on the phone \\ \hline
	 	\emph{Subscription fee} & Users of Spotify pay a monthly fee to access \\ & Spotify Premium \\ \hline
	   \emph{Renting} & Renting a car for the weekend \\ \hline
	   \emph{Licensing}	& Companies have to pay a license fee to get \\ & access to patented technology  \\ \hline
	   \emph{Brokerage fee}	& A seller that earns a commission each time \\ & he sells a product  \\ \hline
	   \emph{Advertising} & A newspaper takes a fee from companies \\ & who wants to promote their product \\ &in the newspaper \\ \hline
    \end{tabular}
    \caption[Different ways to generate revenue streams]{Different ways to generate revenue streams}
    \label{tab:revenue}
\end{table}

\subsection{Cost Structure}
Osterwalder’s definition is: “The Cost Structure describes all costs incurred to operate a business model” \cite{osterwalder}. The costs in the business model come from Key Resources, Key Activities and Key Partnerships. The book \cite{osterwalder} defines two cost structures: cost-driven business model, which focus on minimizing costs, and value-driven business model, which are focussing on value creation by for example making personalized services. Both cost structures can have different characteristics, shown in table \ref{tab:cost} \cite{osterwalder}.

\begin{table}
\centering
    \begin{tabular}{|l|l|}
        \hline
        \textbf{Type} & \textbf{Description} \\ \hline
       Fixed costs & Costs stay the same regardless of the volume  \\ \hline
       Variable costs & Costs depend on volume \\ \hline
       Economies of scale & Less cost as output increases \\ \hline
	   Economies of scope & Less cost due to larger scope of operations \\ \hline
    \end{tabular}
    \caption[Characteristics of cost structures]{Characteristics of cost structures}
    \label{tab:cost}
\end{table}

\cleardoublepage
\chapter{Information Gathering}
In the initial phase of this project assignment we discussed what could be the proper customer segment for the exergame. We developed different business models for four different customer segments that we came to think about. The customer segment's considered were: \emph{the end user (elderly)}, \emph{training groups}, \emph{community centers} and \emph{physiotherapists (and the health sector in general)}.  We did this on a canvas with the same structure as \ref{fig:TheBusinessModelCanvas}. Already at this point, we acknowledge that not every customer segment were suitable. Based on this canvas we continued our research on the topic. From the previous chapters we can conclude that physiotherapy clinics are the right customer segment for Cyberlab at this point. To build a business model around this customer segment, we have to understand more about the segment. \\ \\
In this section we will describe the methods we have used to gather information about the customer segment.

\section{Qualitative Research}
Qualitative research means to get an in-depth understanding of a phenomenon \cite{interview2}. We have performed this type of research to find out more about the physiotherapists, their working situation, their buying routines and their attitudes towards new equipments. To do this, we used interview as a method. \\ \\
We interviewed three physiotherapists working with the older patient group. In addition we had an e-mail-conversation with one physiotherapist. Our supervisor helped us get in contact with one physiotherapist, who again put us in contact with other possible interview objects. In addition to that, we contacted some physiotherapists ourselves. We chose to look at both clinics controlled by the government and also one private clinic. The reason why we did this, is because the two entities have two quite different economic models. In the private clinic, the owners are the payers, while in the public clinics, the government is the payer.

\subsection{Interview Methods}
There are different types of qualitative interview methods \cite{interview} \cite{interview2}: \\ 
1. \emph{Structured Interview:} The main topic for the interview is decided and a complete interview guide is prepared beforehand. \\ 
2. \emph{Unstructured Interview:} This is a very flexible method where the topic is decided, but there is usually no interview guide. This allows the interviewer to improvise suitable questions during the interview. \\ 
3. \emph{Semi-structured Interview:} This is a mix between method 1 and 2. The interviewer has an interview guide with some prepared questions, but these questions serve more as guidelines, and allows the interviewer to improvise suitable questions during the interview. This is the most commonly used interview method, and is often called "qualitative interview".  \\ \\
We used semi-structured interview for several reasons. First, all of our interviewees where physiotherapists and we had some specific questions about their routines directed towards the business model we were working on. However, since they were all working in different clinics the questions had to be adapted towards their field of expertise, and it was therefore room for improvisation. Second, since neither of us are professionals in this field, it would be impossible for us to foresee everything that should have been asked about. Last, we wanted to make the interview as as natural as possible, without locking ourselves to specific questions. This allowed the conversation to flow more naturally, providing us with some unexpected information that we did not think of beforehand. \\ \\
In accordance to the normal structure of an interview \cite{interview2}, we started with an introduction, telling about who we are, about our project, and the goal for the interview. We then followed up with some basic questions about the interviewee, like name, age, work and education. In this way, we got to know each other better before the questioning started. We had two different main topics we wanted to discuss. We wanted to learn about how they work and what kind of relationship they have with their patients. This was to identify if there was a need and also how the product could fit into their working situation. We also had some questions more directed towards the business model, like how they got to know about products and how they acquired them. In each topic, we had some defined questions to guide us, but not to limit us. There was room for improvisation at all times. We modified the interview guide between the interviews based on the experiences we gained.

\subsection{Possible Pitfalls}
When doing a qualitative interview there can always occur some unexpected problems or difficulties. We will go through some of the limitations that may have affected our interviews, based on a list of possible pitfalls we found in \cite{interview}:\\
\emph{Artificiality of the interview:} All our interviewees were strangers to us. The interviewees were asked to answer questions  and give opinions under time pressure. This might have made the interview artificial. \\
\emph{Lack of trust:} Because we did not know the interviewees, them not trusting us could have been an issue. This means that the interviewees may have held back what they think of as sensitive information. This information may have been important information for us, and a possible holdback of this information would make the gathering incomplete. \\
\emph{Lack of time:} In one of our interviews, we had a time-limit. Whether this had a positive or negative effect is hard to say. Time limit can result in an incomplete data gathering, but also lead to the opposite where the interviewee creates opinions under time pressure which can result in more data, but possibly less reliable data. In our case, we do not feel that time constraints resulted in any unreliable data. \\
\emph{Constructing Knowledge:} Interviewees may not have reflected over the questions asked during the interview before. Maybe they did not know that much about the topic, and therefore constructed a story that was consistent to appear knowledgeable. It is hard to say if this was a problem in our case, but we believe that all of our interviewees had sufficient knowledge on all of the topics. In addition the answers from the different interviewees where consistent to each other. \\
\emph{Ambiguity of language:} Sometimes a meaning of a word can be ambiguous. Both the interviewer and the interviewee can misunderstand the meaning. Since our knowledge within this field was limited, it appeared some misunderstandings during the interview. The misunderstandings were discovered when the interviewee read through the interview report, and were fixed at the same time. In this way it did not affect our final report.  \\ \\
The information gathered from the interviews appeared very valuable for our work, and made the result credible. However, interviews with more people would be preferable to make an even more credible business model. We did not have time or resources for this in our project assignment. 



\cleardoublepage
\chapter{Interview Discussion}
In this chapter we will discuss the findings we did in the interviews. We will only take into consideration what we find important for the analysis of the business model. Everything is taken from the interviews conducted, in addition to our own opinions and perception of the interviews. For the interested reader, the interview reports can be found in Appendix A - D (only in Norwegian).\\ \\
Two of the interviewees were physiotherapists working in government controlled clinics, from now on called public clinics. The third interviewee worked in a private clinic, owned by herself and a partner. All of the interviewees had elderly people as a patient group, but not necessarily their only patient group. The private clinic offers an exercise program where elderly can meet and exercise with other elderly once a week. This is also something that is arranged in the public arena, called "Seniortrim" \cite{trim}. The latter is a training group specifically for fall prevention. One of the interviewees mentioned that one of the problems of motivating the patients is that they do not identify themselves as  persons being afraid of falling, so when working with fall prevention it is important to not mention the word "fall". Therefore, in promoting "Seniortrim", they are focusing more on encouraging them to improve their physical health. This is something to keep in mind when promoting Cyberlab's exergame. The typical feedback from these training groups is that the participants think it is nice to get some physical movement and that the social aspects are important. The training group in the private clinic costs 60 NOK per session and the one in the public arena costs 30 NOK per session, allowing most people to afford it. This is an arena where people can play together, as well as getting assistance from a professional. We see this as two interesting offerings, where there is a potential for the exergame to be implemented.  \\ \\
Two of the interviewees mentioned that they wish to get the patients in for an examination earlier. Most of the patients go to the physiotherapist first when their problem gets serious. Getting them in earlier means a chance for prevention instead of rehabilitation. Everything really depends on the patients' background; whether or not it is important for the patients to be able to continue working out everyday activities, or if they just accept the fact that they are getting older and are not able to do everything they did before. Typically, people that are used to be active in the sense of often taking a walk, go skiing during the winter etc., will be more eager to keep a good physical health. For those that are used to get physical activity from for example gardening, the relation to other type of physical activity may not be present. Thus, it is very important to not look at elderly as one common patient group. They are also different people with different interests. This is important to acknowledge when developing a game for this group.\\ \\
A normal problem is that elderly are afraid of walking outside their house, making it hard for them to attend their appointments and also other type of activities. Some people are very motivated to improve themselves while others are satisfied with how things are. Usually, one hour at the physiotherapist is not enough to improve physical strength and it is hard to get the patients to exercise at home, often because of the lack of motivation. It is common to provide the patients with an exercise program that can be performed at home, but there are some problems related to this. First, the patient may not be motivated enough to actually perform the program. Second, it is hard for the physiotherapist to give feedback to make sure that the exercises are done right. Finally, there is no one there to make sure that they do not fall and hurt themselves. The two first problems may be solved with an exercise game, while the latter problem also is related to the exergame.  \\ \\
With regard to our business model we asked all the interviewees where they hear about new products, treatment methods and tools. Different channels were mentioned, like "Fysioterapeuten", which is an academic magazine for physiotherapists, different conferences, "NAV Hjelpemiddelsentralen", and from suppliers they already had an established relationship with. Every one of them pointed out how important it is that the product Cyberlab is trying to sell is proved to work. It is not enough to have an ad in a magazine or newspaper, or show up demonstrating on a conference, if there is no well documented effect. If the product is proved to have good effect, physiotherapists' staff meetings could be an effective arena for promoting the product. Again, she emphasized the importance of well documented good effect. The threshold for the use of a product should be very low. It must be easy to use and must also ease the workload for the physiotherapist. If it is not better than what the physiotherapists can offer themselves, they would not use it. In the public clinics the government pays for everything. Every year, a certain amount of money is given to each clinic. The clinic is then responsible for how they want to spend the money. If a physiotherapist believes in and wishes to get a new product, the leader will be responsible to consider it and decide if they should buy it. This is different in the private clinics. Here every decision are made by the owner(s). \\ \\
The most common way to acquire a product is to buy and own it. Leasing is not that common, but not irrelevant. The private clinic was only one year old, so their economy had not had time to get stable yet. Because of this vulnerability they were careful about buying expensive new products and they were also sceptic to subscribe on a rental agreement. We represented a fictive scenario where they could get an opportunity of paying only for the use of the game. She was very positive to this possibility. All of the interviewees mentioned that it might be interesting to rent a product for a certain period of time to see if it was interesting to buy. It would also be interesting to try something for free.  One of the interviewees mentioned that for this specific game, it would be necessary to run a pilot project, where some clinics tried the game out for a couple of months. In this way they could test the game and document the effect. It is not enough for the potential buyer to know that this is an EU-funded project and that it seems like a nice product. The product must be well proved over a longer period of time. \\ \\
The main goal for the exergame is to prevent elderly from gradually losing their physical abilities. It is shown that exercise only once a week is not enough \cite{gruppetrening-trheim}. This could also be confirmed by one of the interviewees. At the same time, some exercise is better than none, and the physiotherapist who provided the group training once a week told that you could actually see improvements after six to seven sessions. For the elderly to be able to improve themselves even more, they will have to exercise more. A plan for the future should therefore be to provide the game for the elderly to use in their own home. It is not natural for us to believe that an older person would go to the store and buy this game themselves. Therefore, it was interesting for us to see if physiotherapists could work as a possible channel where the patient could get this game. In the public clinics it is normal to let patients borrow products, but the number are limited.  If the government believed in a product, they could buy some copies that patients could borrow. In the private clinic it is normal to sell products, like for example special shoe soles, but they do not typically let patients barrow products. \\ \\
It is not common for physiotherapists to physically go to the store and buy a product. An already established relationship with a supplier is the most common channel where new products are bought. Often after a sale, the supplier will contact the clinic with improved or new products. It is quite normal to maintain a relationship with the supplier after a product is bought. Normally, the supplier provides the clinic with brochures with news and improvements, as well as support if something goes wrong with the product. When the product is bought, it usually gets delivered at the clinic. Physiotherapists have already enough to do as it is, and have not always time to set up and learn a new product. It depends on the product whether they get an introduction or not. It would for instance often be necessary to get an introduction to technical products. When it comes to relationship with the supplier, all the physiotherapists we interviewed stated that the opportunity to come with feedback on the product was important. The reason for this was that they wanted the product to be customized for their and their patients need. \\ \\
One of the interviewees talked about how this game could fit into the Norwegian government’s new reform "Samhandlingsreformen" discussed in Chapter 3. Of course this depends on whether the game is proved to be effective in health improving purposes and on cost.  \\ \\
All of the interviewed physiotherapists were positive to Cyberlab’s exergame. At the same time, neither of them could say anything about whether they would use it or not, nor if they thought it would be a suitable tool for elderly to use. The reasons for this were that the game has not been developed yet, and they would have to see it and try it before they could make an opinion about it. They expressed that it is important that the game can ease their workload and also offer something better than what they can offer themselves. The ability to customize the program was an aspect that was very important for all of the interviewees. The game should provide the possibility to put together different type of exercises and change the degree of difficulty. To make it possible for elderly to use this kind of game it has to be easy to use and self-explanatory. In addition, the game should give some kind of feedback to the user with instructions on whether they are doing the exercise right or wrong. This can work both as a motivating factor for the user and as a tool for the physiotherapist to keep track on how the patient is doing. The opportunity to not only customize in the sense of the right exercises, but also different themes, should make the game more fun and motivating. \\ \\
It was pointed out that it is important to remember that elderly today are not familiar with technology. This game will probably be more relevant when the next generation gets older, especially with regard to the use in private homes. Today, we can assume that the older patient would not on their own initiative buy this game, but rather use it after a recommendation from their physiotherapist. \\ \\
Another issue pointed out was where the game could be used. The different clinics may not have space to have for instance four people playing a video game in one room. Most likely, to be able to take the game seriously, as well as not disturb other patients in the clinic, the game has to be played in a separate room. This must be up to each clinic to find a way to work around. An issue implementing this game in a private clinic might be that they do not have that big customer base. The reason for this is that patients have to pay for the treatment. At the same time, it could be an interesting tool for them to use to tempt customers to choose them as a clinic. Of course, this might not work, if "all other" clinics offered this game as well. \\ \\
From the interviews conducted, we can conclude that physiotherapists are the right customer segment for Cyberlab. Today, it would not be relevant to go straight to the end-user, because they are not familiar with this kind of technology. A physiotherapist is a professional who the patient can trust. If a professional say something, we are likely to believe in it.  The physiotherapists wish to get their patients in for examination earlier so they can prevent the patient from getting serious health problems. Offering the exergame in for example a group training session, can encourage elderly to go to the physiotherapist earlier. It is important that the game can be customized both in the favour of the physiotherapist providing it and the patient using it. This include different exercises, levels and themes. The game has to be shown to improve patients' physical health as well as their mental health. It should be fun, motivating and easy to use. And at last, it has to ease the physiotherapists workload. \\ \\
In the next chapter we will provide a business model where we analyse the potential of this game with physiotherapy clinics as the customer segment. We mean that the information gathered from the interviews support our choice of customer segment, even though there were some uncertainties around the game. There is a potential both in a training group situation and also for individual treatment. In the long run, physiotherapists can serve as a channel to sell the product to the end-user. This will no be taken into consideration in our assignment. 
\cleardoublepage
\chapter{Business Model for the Microsoft Kinect Based Exergame}
In this chapter we will use the information gathered in the previous chapters to  make an analysis of the business opportunities of the exergame. We will provide a detailed description of a business model that describes how the game should be created and delivered, and how it can create value. As a framework for this description, we will use Osterwalder's Business Model Ontology, as described in Chapter 5. At the end, we will provide a detailed financial analysis. \\ \\ 
 
\section{Product}
A product covers all aspects of what a company offers to its customers. The product is composed of value propositions, which are services and values offered to the customer. Cyberlab’s product is an exergame that can be used as a tool for exercising and rehabilitation for elderly. The focus of the exergame will be to improve strength and balance in elderly to prevent falls and injuries. By using the exergame, this can be done in a more amusing and motivating way. The game will have one general workout version for training and one customized version for rehabilitation. The exergame will be developed for the Microsoft Kinect sensor for Windows, which tracks body movement without the use of any controllers. This makes it convenient to use for elderly, as it allows them to only focus on the tasks provided by the game. Our proposal is that the exergame can be used in physiotherapy clinics, where physiotherapists can set up customized programs and guide their patients through the exercise. The game is not meant to replace ordinary physiotherapy, but will serve as a supplement. The Microsoft Kinect sensor provides useful feedback to the physiotherapists when it comes to measuring the execution of the exercises and their patients' progression. 
\subsection{Value Proposition}
The exergame's value proposition can be described as: \\ \\
\emph{A tool with the ability to customize an exercise program, and to offer an alternative, fun and motivating training method, while at the same time ease the workload of the physiotherapist.}\\ \\
Value propositions refer to the value a company offers to a specific customer segment. The product offered to the physiotherapists is something completely new. It provides them the possibility to offer an alternative, fun and motivating training and rehabilitation method. The exergame can be used as a supplement in training programs or as an exercise motivator. Social aspects around the game will be provided, because this is seen as highly important aspects for elderly \cite{exergamesforelderly}. This includes offering different games for different interests. The game can also be used as a tool for physiotherapists to make it easier to customize training programs for their patients, which will ease their work. Every patient is different, with individual problems and needs, and it is therefore necessary to provide personalized exercise program for each patient. An important value the product has to serve, is that the exergame can set up training programs and be more motivating than a physiotherapist can. The game should be self-explanatory and easy to use for the end-user. It will also provide useful feedback to inform and encourage the participant.
\section{Customer Interface}
This section will describe the customer segment in more detail, in addition to how Cyberlab can create value for them. In particular we will discuss who the customers are, how to reach out to them and what kind of customer relationships should be maintained after the sale. 
\subsection{Customer Segments}
In general, a company generates value for one specific customer segment or several different customer segments. We believe at this point this game is best suited as a tool for physiotherapists to use when working with elderly. However, we also considered some other possible customer segments: \emph{elderly}, \emph{training groups} and \emph{community centers}. We will briefly discuss why these were not included as additional customer segments in Chapter 9.\\ \\ 
Physiotherapists is the first entity elderly meet with when facing a physical problem. One of their jobs is to help their patients decrease the risk of falling by using mobility techniques to improve balance and physical strength. Physiotherapists are also a group with a certain authority appearance, which makes them trustworthy. Physiotherapists can buy this product, install it in their environment and use it as a tool for training and rehabilitation, both in an individual therapy session and in a training group. \\ \\
We recommend Cyberlab to focus on physiotherapy clinics as customers. More precisely, we will recommend starting with public clinics and private clinics with contribution from the government. The goal of their work fits well with the purpose of Cyberlab's exergame, and the authority they have is very valuable when trying to get elderly to adapt the new technology this game provides. We will discuss in Chapter 9 why we excluded private clinics. The recommended physiotherapy clinics have access to a wide customer base consisting of elderly. It is also important to emphasis the importance of the new reform "Samhandlingsreformen", which encourage the use of welfare technology where possible, discussed in Chapter 3.1 and 3.2. This suggests that there is a place for an exergame in the health sector.  The economy in this entity is controlled by the government. With the support from the new reform, we believe there should be no economic issues around investing in a game like this. 

\subsection{Channels}
In this subsection we will describes how Cyberlab should deliver and market their value propositions to their customer segment. We will describe this by going through the five channel phases, see Figure \ref{fig:Channels}.\\ \\

\begin{sidewaysfigure}
\centering
\scalebox{0.60}
{\includegraphics{channels}}
\caption[The 5 Channel Phases for the Cyberlab]{The 5 Channel Phases [modified from \cite{osterwalder}\cite{osterwalderthesis}]}
\label{fig:Channels}
\end{sidewaysfigure}

\emph{Awareness:} \\ 
We will look into some activities Cyberlab should perform to raise awareness about their product and how they can get their customers attention. "Fysioterapeuten" is a magazine targeting physiotherapists in Norway. This magazine is published by \ac{nff} and distributed to the whole country. "Fysioterapeuten" contains mostly scientific papers, and the idea is that this magazine shall contribute to evolvement of the physiotherapy profession. This magazine is read by 9 000 physiotherapists around the country, and articles printed here are perceived as scientific and are therefore taken seriously \cite{fysioomoss}. Cyberlab should use "Fysioterapeuten" as a medium to promote the exergame by printing articles or ads. Printing research papers and articles about the exergame in other credible magazines or newspapers is also a way to get physiotherapists attention. \\ \\
Another solution for Cyberlab is to promote their product by taking direct contact with the customer segment. This could be done by taking part in conferences related to subjects like e.g. welfare technology, or by visiting physiotherapy clinics. Cyberlab will then get the opportunity to present the product and show direct interest in establishing a customer relationship. An example of a highly relevant conference for Cyberlab to attend is "Velferdsteknologikonferansen" \cite{conference}. This is a conference that has been held in Trondheim three times, and is describes as a conference for people that are interested in welfare technology and "Samhandlingsreformen". Physiotherapists see it as very relevant to try a product for an amount of time before they decide to buy. Cyberlab can raise awareness around the product by announcing that they need participants for a pilot project. Awareness can also arise from an already ongoing pilot project. \\ \\
\emph{Evaluation:}\\
Physiotherapists we have talked to pointed out two very important aspects in evaluating a product. These two are; documented effect of the product and own experience by testing the product. If a physiotherapist should even think about trying the product, it is necessary to provide research papers and statistics that show positive effect on the use of this kind of game. Documentation will give them a security in the choice of buying the product. Other physiotherapists providing positive feedback after trying the game also contributes to this security. We suggest that Cyberlab should start pilot projects at one or more physiotherapy clinics, as this is a good way to be evaluated. A pilot project will provide the physiotherapists with the possibility to test, experience and evaluate the product themselves for a period of time. The experiences from the pilot projects should be used as documentation for the product.\\ \\
\emph{Purchase:} \\
Most physiotherapists have already established connections with suppliers. Ordering and buying products are usually done online, but it can also be done by phone or when interesting products are discovered on conferences. Going to the store to buy products is very unusual. To play this game there are software and hardware needed, like Microsoft Kinect sensor and the exergame. It is unreasonable to think that a physiotherapy clinic already owns a Kinect sensor, so the best strategy for Cyberlab is to sell packages containing both hardware and software. Packages should include various agreements with appropriate pricing. Cyberlab should sell the game through their own channels. We suggest that the ordering of this game should be done on Cyberlab's web page. This is convenient for Cyberlab, who does not need to pick up the phone every time someone wants to order the product. It is also convenient for the customer, because ordering online is not a very time-consuming matter, and most people are familiar with online purchase. \\ \\
\emph{Delivery:}\\
Feedback from the interviews shows the huge importance of start-up help. They expressed a desire for installation and introduction help for technical products. With busy days at work, physiotherapists do not have time to pick up deliveries at the postal office, or to setup and learn new products all by themselves. Buying, receiving, installing and learning should not be difficult or time-consuming. \\ \\
The product should be delivered at the door, by someone who can install the product and teach the physiotherapists how to use it. However, this is a complicated task for Cyberlab if we look at every physiotherapy clinic in Norway as their customer segment. It will be very inconvenient and costly if one representative from Cyberlab will travel around in Norway based on where they get the order from. However, Cyberlab should offer these services to those customers who are willing to pay for it.  \\ \\
\emph{After sales:}\\
When taking a new product into use, it is important for the physiotherapists to have the possibility to come with feedback. Therefore, Cyberlab should have some kind of support that can take these feedback into consideration. Feedback can be comments on direct errors or directions on how to make the exergame more suitable for its use. Cyberlab should follow their customer in the process of learning, and they should inform them of new features and improvements. We suggest that Cyberlab should accept feedback on e-mail and that they provide the buyers with a phone number for critical support.  
\subsection{Customer Relationships}
Customer relationship is an important part of the customer experience, and it describes what kind of relationship the company establishes with the customers. Support, follow-up and feedback handling are some aspects in establishing customer relationship. Cyberlab should establish a personal relationship with their customers, and be available when the customers have problems and need help. When using a new product one may discover errors or find the product not suitable for its use, so many physiotherapists have an eager to provide feedback on this. Cyberlab should handle these feedback and fix errors as soon as possible and take comments on improvements into consideration. Using feedback to make a better product shows customers that Cyberlab takes their comments seriously.  In addition, the customers will hopefully get a more suitable product.  If Cyberlab maintains a direct and personal contact with the customers, it will show the customers that they care about the use and experience of the game.  All this can contribute to a good customer relationship. Cyberlab should also give their customers a heads ups on updates, new features or products. As mentioned under \emph{after sales} in Section 8.2.2, this communication should be carried out over e-mail. We mean that this will make the relationship personal, but not inappropriate (e.g. annoying customers with telephone calls). We suggest that Cyberlab should provide their customers with updates and new features of the game (e.g. a new theme), which will make the customers use the product for a longer period of time. They will then become more familiar with the product, and engage a relation to Cyberlab. This is a way to lock the customers to them as a provider. If exergaming becomes popular and common, more providers can appear. Then it will be important for Cyberlab to retain their customers. 

\section{Infrastructure Management}
This section is about how Cyberlab creates value. What resources needed and what activities that have to be performed are described here, as well as if they will get them in-house or from a partner. 

\subsection{Key Resources}

In this section we will describe all the resources needed to make the business model work. The resources are divided into four different types, described in Table \ref{tab:Resources}.
\newpage

\begin{table}
\centering
    \begin{tabular}{|l|l|}
        \hline
       \textbf{Type of Resource} & \textbf{Resource}  \\ \hline
       \emph{Intellectual} & Insight and experience with fall problematic \\ & in elderly \\ \cline{2-2}
        & Programming skills \\ \cline{2-2}
	 	& Creativity \\ \hline
	   \emph{Physical} & Premises \\ \cline{2-2}
	   	& Equipments, i.e. desks and computers  \\ \cline{2-2}
	   	& Microsoft Kinect sensor \\ \cline{2-2}
	   	& Windows machines \\ \cline{2-2}
	   	& Projector and screen \\ \cline{2-2}
	   	& Working environment, Kinect for Windows SDK \\ \cline{2-2}
	   	& Internet connections \\ \cline{2-2}
	   	& Web page \\ \hline
	   \emph{Human} & System developers, i.e. programmers and \\ & interaction designers \\ \cline{2-2}
	   	& Administration, i.e. for marketing, customer related \\ &tasks \\ \cline{2-2}
	   	&  People for pilot project follow-up \\ \cline{2-2}
	   	& Support person(s) \\ \hline
	   \emph{Financial} & The European Union \\
        \hline
    \end{tabular}
    \caption[Resources]{Different types of resources}
    \label{tab:Resources}
\end{table} 
\emph{Intellectual} \\ The developer team needs insight and knowledge about different exercises that will strengthen muscles and improve balance in elderly. Cyberlab is provided with research information from other entities in this project, so their job will be to process this information. When they have enough knowledge to form the foundation of an exercise program, they can start to get creative. Creativity is needed to make the game entertaining and easy to understand and conduct. In addition, good programming competencies are needed to develop the game. To make it as cost-efficient as possible, an experienced team should be put together. \\ \\
\emph{Physical} \\ To be able to conduct this project, the team needs premises and everything that comes with it, like desks, chairs, computers, internet connection, lights etc. Cyberlab is an already established business, so we can assume they already have these premises and equipment established, and that this will not provide any additional costs. For this project, they will need specific hardware. The hardware consists of the Kinect sensor, a Windows machine, a server for storing and running the game and a projector and screen for testing purposes. In addition they will need the Kinect for Windows SDK to be able to develop a game for this platform. Last they will need to expand their web page with sites for information about the game, news and ordering/purchasing. \\ \\
\emph{Human} \\ Programming skills and creativity are already described above as intellectual resources. This means they will need system developers and interaction designers. An administration is needed for marketing, customer related tasks and resource management. When the game is finished it needs to be operated and maintained. These tasks can be done by one or more of the system developers. They will also need people to follow the pilot project. We believe this can be done by one of the project team members. \\ \\
\emph{Finance} \\ This project is financed by the European Union. However, we will not take this into account when looking at the financial aspects of the game. There are two main reasons for this. First, as explained in the introduction, Cyberlab is part of a project called "GameUp". There are several entities involved in this project. Without any more knowledge about the project and the different entities' roles it is hard for us to know how the costs and revenues will be divided between them. Second, there was a desire from Cyberlab to look at how they can make it on their own in the Norwegian market. 

\subsection{Key Activities}
Cyberlab's work can be described as a Value Chain, which means transforming inputs into a final product. From the knowledge and experience Cyberlab has acquired they want to make the game as good, cost efficient and price-competitive so that their customers would choose their product instead of a product with similar value. A description of the different stages in the value chain is depicted in Figure \ref{fig:ValueChainCase}. The development of the game should be test-driven, meaning that they will test the product both on the end-users and the customer segment during the development, and adapt the game based on the experiences acquired during the testing. Activities that need to be done include research processing, development, testing, maintenance and updates, support, marketing and administrative tasks, shown in Table \ref{tab:activities}. \\ \\
Cyberlab is provided with research information from Norut, a national research group from Tromsø. Cyberlab's first task will be to process this information to understand the requirements for the game. Cyberlab will also be provided with suitable exercises that professionals know is good for the end-user. Cyberlab has to find a way to build the game around these exercises. An important task is to make the game user-friendly and easy to use. Prototypes should be made and tested on the end-user to see if the game is easy to understand and conduct. This includes menu choices, information feedback and the execution of the game. The physiotherapists should also take part in this. After the game has been tested, they have to modify the game according to the feedback, and test it again. This should be done until every actor is satisfied. During the development, they should get one or more clinics to participate in a pilot project. This should start as soon as the game is finished, but they can also be involved in the testing during the development phase. The documentation from this pilot project is important when marketing the product. The marketing task is very important, and should be put much thought into. Marketing should happen through the channels described in Chapter 8.2.2.  We will discuss marketing in more detail in Chapter 8.4.1. An additional important part of the development is to find out how to set up the infrastructure (e.g. databases, servers, communication between entities). After the game is finished and delivered to the customers, it has to be maintained and enhanced. New versions and updates should be provided for keeping a customer relationship. In addition, they must provide support to the customers.  

\begin{figure}
\begin{center}
\includegraphics[scale=0.7]{valuechaincase}
\caption[Value Chain for the Kinect based exergame]{Value chain for the Kinect based exergame [modified from \cite{osterwalderthesis}]}
\label{fig:ValueChainCase}
\end{center}
\end{figure}

\begin{table}
\centering
    \begin{tabular}{|l|l|}
        \hline
        \textbf{Type of Activity} & \textbf{Activity} \\ \hline
        \emph{Primary} & Research \\ \cline{2-2}
        & Development \\ \cline{2-2}
	 	& Testing \\ \cline{2-2}
	 	& Marketing \\ \cline{2-2}
	 	& Support \\ \hline
	 	 \emph{Support} & Maintenance \\ \cline{2-2}
	   	& Administrative tasks \\ 	 
       \hline
    \end{tabular}
    \caption[Different types of activities ]{Different types of activities}
    \label{tab:activities}
\end{table}

\newpage

\subsection{Key Partnerships}

Norut is a national research group located in Tromsø. Cyberlab depends on them because they provide them with research information. This is the only entity Cyberlab depends solely on, so this is the only key partner. \\ \\ 

\section{Financial Aspects}
In this section all the outgoing and incoming cash flows will be described. All the previous blocks are contributing to a cost or an income. We will try to provide a realistic and detailed estimate of both cost and income, and then make an analysis of potential profit. It is important to mention that we have done many assumptions and that not all costs are taken into consideration. We will describe this as the assumptions are made.

\subsection{Cost Structure}
Cost structure takes into account all elements that generates costs specific to this game. Cyberlab is an already established business and we can therefore assume that there are not any additional costs associated with premises and some of the "regular" equipment (e.g. desks, chairs, computers etc.). We will not distinguish between fixed and variable costs, but rather look at every cost as fixed, annual costs. Variable costs are salaries associated with support and administrative tasks. Here we will assume that these tasks have assigned a fixed amount of workload for each year. Further, we will distinguish between investment costs and ongoing costs. Cyberlab's costs have the characteristic \emph{economies of scale}, meaning that the cost per unit will fall when output rise. This is due to a marginal cost close to 0.  The cost structure of this business will be both value-driven and cost-driven. They will make a cost-efficient product, that will serve all its value propositions. The value of this game is important to distinguish it from "ordinary" video games. \\ \\
\textbf{Investment Costs}\\
Investment costs are all the costs associated with the development of the game. This includes salaries for the development team, the hardware and software needed to develop the game and the cost of the pilot project. We will assume that the development of the game will take 6 months, and that they will carry out a pilot project for 6 months. This means that Cyberlab is looking at a whole year without revenue.\\ \\
\emph{Hardware and Software:}\\
The commercial price for the Kinect sensor is 1 790 NOK \cite{pricekinect}, and the \ac{sdk} for the sensor is free. In addition they should have a screen for testing purposes. The screen should be of significant size, so we suggest they invest in a projector and a 90" projector screen, which should be sufficient for its purpose. We found that the cost of an average screen is 895 NOK and 2 449 NOK for a projector \cite{priceprojector}\cite{pricescreen}.\\ \\
\emph{Development:}\\
Cyberlab has estimated that for the development of this game they need a \ac{fte} = 1.0, meaning that the workload is equivalent to one person working full time for a year. We assume that this will cover development, testing and administrative tasks during the development. In addition, Norut who provides research has also assigned a \ac{fte} = 1.0. How many people assigned to the project is unknown and also irrelevant for the cost prediction (assuming every employee has the same salary). We have done an estimate of how much the cost of having an engineer with a \ac{fte} = 1.0 in the private sector is. From \cite{tekna} we found statistics of salary in the private sector in Norway. Assuming that the "average" employer on this project graduated in the end of the 90's, we look at an average gross salary of approximately 730 000 NOK a year. From this we can calculate the average cost of a \ac{fte} = 1.0, based on \cite{altinn}, which will be 1 003 349, see Table \ref{tab:costofFTE} for calculations. We did the same calculations for researchers in the government controlled sector and for marketers in the private sector. They both ended up on a cost of 715 577. See Appendix E for calculations.\\ \\
\begin{table}
\centering
    \caption[Cost of FTE = 1.0]{The cost of FTE = 1.0 in the private sector (for Cyberlab)}
    \begin{tabular}{|l|l|l|r|}
        \hline
       1&Gross Salary & & 730 000 NOK \\ \hline
       2&Holiday Pay & 12.1\% of 1  & 88 330 NOK \\ \hline
	   3&Employee Fee & 14.1\% of 1+2  & 115 385 NOK \\ \hline
	   4&Pension Costs & 8.0 \% of 1 & 58 400 NOK\\ \hline
	   5&Employee Fee of Pension Costs & 14.1\% of 4 & 8 234 NOK \\ \hline
	   6&Insurance & & 2 000 NOK \\ \hline
	   7&Mobile and Internet & & 1 000 NOK \\ \hline
	   & \textbf{SUM} & & \textbf{1 003 349 NOK} \\
	    \hline
    \end{tabular}
    \label{tab:costofFTE}
\end{table}
\emph{Pilot Project:}\\
As mentioned we suggest that Cyberlab should run a pilot project to document the effect of the game. This will also serve as a very effective way of marketing the game. We suggest that the pilot project should be carried out in one or more clinics in Trondheim for convenience, and that it should run for six months. The effect has to be documented both during and after the project. This documentation should be published in scientific articles and distributed to physiotherapists. The pilot project will most likely provide Cyberlab with valuable feedback on the game, where they can both test the usability in a real environment as well as discover bugs. This will require close monitoring from Cyberlab, so we suggest that this will require a \ac{fte} = 1/5 for  six months (this equals \ac{fte} = 1/10 seen in a whole year). In addition they will have to pay the physiotherapists working with the pilot project. We assume this will be the same amount as their hourly salary. For these six months we will recommend the amount physiotherapists are working with this equal to a FTE=2/5 (or FTE=1/5 seen in a whole year).\\ \\
The investment costs are summarized in Table \ref{tab:investmentcosts}.\\ \\
\begin{table}
\centering
  \caption[Investment costs associated to the development of the game]{Investment costs (in \ac{nok}) associated to the development of the game. For calculations on FTE, see Appendix E}
    \begin{tabular}{|l|l|r|}
        \hline
       \textbf{Investment costs}  & &\\ \hline
       Hardware: & & \\ \hline
	   & Kinect sensor & 1 790  \\ \hline
	   & Projector & 2 449 \\ \hline
	   & Screen & 895 \\ \hline
	   Storage & & 1 087 \\ \hline
	   Development team & & \\  
	   FTE=1 &  & 1 003 349   \\ \hline
	   Research from & & \\ 
	   Norut FTE=1 & & 715 577 \\ \hline
	   Pilot Project: & & \\ \hline
	   & Representative(s) from & \\
	   & Cyberlab FTE =1/10 & 100 335  \\ \hline
	   & Representative(s) from & \\
	   & the physiotherapy clinic & \\
	   & FTE = 1/5 & 70 000  \\ \hline
	   \textbf{SUM} & & 1 895 482 
 \\ \hline
    \end{tabular}
    \label{tab:investmentcosts}
\end{table}
\newpage
\textbf{Ongoing Costs}\\
We will now look at the ongoing costs on a per year basis. With the rapid evolution of technology, we believe that Cyberlab can offer this game for five years after its release. After these years, they will probably have to start making new versions, even for new types of technology. We do not take the development of new versions into account in our calculation, and we will set the lifetime of this game to be five years. Table \ref{tab:ongoing} summarises the annual costs. \\ \\
\emph{Storage:}\\
The game has to be operated on a server. This can be on a server located in Cyberlab’s office, a server located at one of the physiotherapy clinics or on a cloud hosted server. The size of a Kinect game will vary, depending on quality, colors, how many levels etc. At this stage it is hard to make an exact assumption on how big the game will become. There will be a need for a database with user data and log-data, as well as some web content for portal interface (configuration for the games etc.). The dynamic part of the space needed on the server is associated with how much log-data it needs to store. However, this will not be very space consuming. Based on this, we believe that Cyberlab can make it with a small server of fixed size. From Gogrid Servers \cite{priceserver} we found a small server with storage space of 25 GB. We believe this will be sufficient for Cyberlab's purpose. There is also reasonable to believe that Cyberlab has some space available on their servers. However, if they would have to rent this kind of server space, we are looking at an annual price of \$181.25 which is roughly 1 087 NOK (with a currency of \$1 = 6 NOK).\\ \\
\emph{Support and Maintenance:}\\
With new software and technology there will always appear some errors after the product or service is delivered. We can assume that the first six months are the most critical months, and will require a \ac{fte} = 1/5. The remaining lifetime will only need support for some minor problems that might appear (e.g. customer service). We assume this period will require a \ac{fte} = 1/10 (each year). This is roughly predicted numbers, because this may vary over time. However, our predicted numbers are reasonable as "average" numbers, taking unexpected events into account.  \\ \\
\emph{Marketing:}\\
Marketing is one of the most important parts of selling a product or service. This is especially important in the first year of the game's lifetime. The cost of marketing is difficult to analyse because it depends on how long it will take to acquire customers. New products or services need to acquire customers quickly, and therefore more resources need to be put into the marketing tasks. We can look at the exergame as a niche product that is targeting a specific customer segment. The marketing task needs to be customized for this specific customer segment. When a critical mass (the number of customers needed to survive economically in the market) is reached, the market will somehow be self-supported \cite{informationrules}. We believe that after this critical phase, the marketing costs will be rather low and close to constant. We assume that the first year right before, during and after the release, the marketing task will contribute to a \ac{fte} = 1/2. After being on the market for one year, the customer base should have reached critical mass.  We believe that in this type of community (the physiotherapy community), words spread fast. If someone starts using a product that is proven good, it will soon appear in magazines and by word of mouth, resulting in the interest from others. Even after critical mass is reached, there will be some marketing related tasks (e.g. keep up with the market, look for new customer segments), so we suggest that the marketing tasks should contribute to a \ac{fte} = 1/5 after the first year. With this low workload, Cyberlab could consider to hire a marketing consultant instead of having a permanent employee. However, in this analysis, we assume they have employed a marketing person for this task. Other costs related to marketing are the costs of promoting the product on different arenas (e.g. ad in a magazine). For simplicity, we will assume these costs are contained in the cost of having employed a marketing person. \\ \\
\emph{Costs associated to sales:}\\
The exergame will be sold to the customer as a package with the Kinect sensor and the game included. For convenience, a more comprehensive package with everything else needed to play the game (e.g. screen and a Windows machine), should be offered to the interested buyer. Here Cyberlab could gain some profit. However, for simplicity we will assume that the package includes the Kinect sensor and the game only, and that Cyberlab will not gain any profit on the Kinect sensor sold. We will also assume that Cyberlab buys the sensors on demand, meaning they do not have the sensors in stock. This is because the risk of having a stock, discussed in Chapter 9. In addition, Cyberlab has to pay tax. This is not taken into account here. With our simplifications and assumptions there are no costs associated to the specific sales. \\ \\
\begin{sidewaystable}
\centering
\caption[Ongoing costs on a per year basis]{Ongoing costs (in \ac{nok}) on a per year basis. For calculations on FTE and present value, see Appendix E}
    \begin{tabular}{|l|r|r|r|r|r|r|}
        \hline
       \textbf{Ongoing costs}  & & & & & & \\ \hline
      \textbf{Year} & \textbf{1} & \textbf{2} & \textbf{3} & \textbf{4} & \textbf{5} & \textbf{Total}\\ \hline
	   Storage & 1 087 & 1 087 & 1 087 & 1 087 & 1 087 &\\ \hline
	  Support & 150 502 & 100 335 & 100 335 & 100 335 & 100 335 & \\ \hline
	  Marketing & 357 789 & 143 115 & 143 115 & 143 115 & 143 115 & \\ \hline
	   \textbf{SUM} & 509 378 & 244 537 & 244 537 & 244 537 & 244 537 & \textbf{1 487 526} \\ \hline  
	   \textbf{PV} & 489 787 & 226 089 & 217 393 & 209 032 & 200 992 & \textbf{1 343 293}  \\ \hline
    \end{tabular}
    \label{tab:ongoing}
\end{sidewaystable}
\emph{Total Cost:}\\
Taking all the described costs into account, the project with six years of lifetime (including the development phase and the pilot project) will have a total cost of 3 238 775 NOK, see Table \ref{tab:totalcosts}.

\begin{table}[h]
\centering
\caption[Total costs]{Total costs in \ac{nok}}
\begin{tabular}{|l|r|}
\hline
\textbf{Total Costs} & \\ \hline
Investment Costs & 1 895 482 \\ \hline
Sum Ongoing Costs PV & 1 343 293 \\ \hline
\textbf{SUM} & 3 238 775 \\ \hline
\end{tabular}
\label{tab:totalcosts}
\end{table}
\newpage
\subsection{Revenue Streams}
Revenue streams describe how a company can earn money. There can exist one or several revenue streams with different pricing mechanism. Cyberlab will generate revenue by selling their product as a package, containing software, hardware, and services, to physiotherapy clinics. There are various ways to sell this product package, and we will present two possible ways to generate revenue; a fixed price model and a usage fee model. \\ \\ 
We recommend Cyberlab to sell their exergame as a part of a package, consisting of both software and hardware. Hardware needed to play this exergame is the Kinect sensor, a Windows computer and a screen. What physiotherapy clinics need and already have of technology varies. Some clinics might already possess a television; other might request a screen and a projector. Cyberlab should therefore offer the possibility to customize packages in accordance to their customers' needs. The hardware that has to be included in all the packages is the Kinect sensor; as this is what we least expect a physiotherapy clinic to already own. Another feature to be included in the package is delivery, installation assistance and introduction help, which we have experienced as an important offer for physiotherapists. It is useful to give an introduction for this type of technology equipment. Physiotherapist might not have time to learn how the products work, which often results in buying a product they never use due to lack of information. A package containing an offer like this would have a higher price as it covers wages and expenses related to travel. However, all of the interviewees mentioned delivery, installation and introduction as important, so we believe this is a potential revenue stream for Cyberlab. \\ \\ 
We will now present our two chosen revenue models. The recommended prices in the two models will only cover the software and the Kinect sensor expenses. Additional hardware and services requested are for simplicity not taken in to consideration as it most likely will vary from customer to customer. 25 percent \ac{vat} will be added on top of the package price, but this will not be shown in our calculations. In section 8.4.3 we will show how we arrived on the prices suggested below.\\ \\
\emph{Proposal 1 - Fixed Price Model}\\  
A fixed price model implies selling a product to a fixed price. Cyberlab will sell their exergame as a part of a package consisting of both software and hardware. The price for a package should be 12 000 NOK. The total package price for the physiotherapy clinics will depend on additional hardware and services needed. \\ \\
\emph{Proposal 2 - Usage Fee Model}\\
With a usage fee model the customers will pay a low start price for the package and a certain amount of money for each time they use the product. A suitable start price for a package with this model is 2 000 NOK, and an additional 50 NOK for each hour the game is used. The low start price will not alone cover all of Cyberlab's total costs, even if they sell to all their potential customers, so they depend on customers using the product. This model will have the same possibility to include additional hardware and services as the fixed price model. \\ \\
The two revenue models and their price proposals will be discussed and analysed in more detail in the next section.

\subsection{Financial Analysis}
In this section we will make a financial analysis based on the costs and the different revenue models just described. We will give a recommendation on prices suitable for the different models in order for Cyberlab to generate profit, while still remaining competitive in the market. \\ \\
\textbf{The Potential Market} \\ \\
Before pricing the product it is necessary to observe today's market. To understand the market, we have to look at prices on existing games and physiotherapy tools, and try to find a place in between where the game fits. We also have to predict a potential demand for this product. Demand will depend on the documentation of the product and popularity within the physiotherapy community, as well as the product price. Calculating an exact demand for this product is impossible due to the lack of existing games in the same genre and information about the market. It is also hard because the game Cyberlab is going to sell does not exist yet, which makes it difficult for the customers to show their interest. Cyberlab’s market potential can be roughly calculated by looking at the number of public clinics and private physiotherapy clinics with support from the government, from now on referred to as physiotherapy clinics or just clinics. We looked at four municipalities in Norway; Oslo, Trondheim, Fredrikstad and T{ø}nsberg, where we for each found the number of clinics and compared that number to the population in each municipality. The average ratio we got, describes inhabitants per clinic in Norway. Finding the total population in Norway and dividing this by the average ratio, gave us an approximation of physiotherapy clinics suitable for Cyberlab's customer segment. Our calculation shows that Cyberlab has a potential market of approximately 1 200 physiotherapy clinics in Norway. Calculations are shown in Appendix G. We contacted \ac{nff} to ask about the actual number of clinics in Norway, but they could not provide us with an exact number. Because of this we made our own calculation.\\ \\
We suggest that Cyberlab should start with Trondheim as their main focus. The reason for this is that the product has been developed in Trondheim and that they already have established relationships with some entities through the EU-funded project. This will make it convenient for Cyberlab to follow up and to respond quickly to requested changes or possible errors, as well as monitoring the progression. We will now assume that the game has been developed, is well-documented, has received a great amount of positive feedback and has been accepted by the physiotherapy community. With these conditions in place we believe that there will be a diffusion of the game, starting slowly in Trondheim and as it gets more and more attention it will spread to the rest of the country. This means that it might take some time before Cyberlab will generate any profit, as we will demonstrate later in this section. We can describe the diffusion of a product with an S-curve, depicted in Figure \ref{fig:scurve}. \\ \\
The S-curve describes how an innovation will adopt customers over time. In the introduction of a product or service it will take some time to adopt a critical mass. There are different types of people adapting to the technology at different time. These types can be defined as innovators, early adopters, early majority, late majority, and laggards. As the actors are adapting to the technology, the market share will grow and eventually reach its saturation level, meaning it has reached to all potential adopters. Most innovations can be described with an S-curve, but the curve will look different for different innovations\cite{scurve}.\\ \\
With the S-curve in mind, we will try to put a number on what we believe Cyberlab can expect to sell in each of the five years after the game’s release. Even though the potential market consist of 1 200 clinics, we do not think it is reasonable to reach out to all of them. A maybe more realistic sales number would be to cover one-third of the potential market share, which is 400 clinics. The reason for this is that it is a great amount of uncertainty related to the exergame. The video game itself is not a new technology, but the use of video games as an equipment in physiotherapy treatment is. The market is immature and inexperienced, so the exergame could meet a great amount of doubt when launched. We have to take into consideration that there are physiotherapy clinics that do not see the need for an exergame. In one of the interviews we were also told that physiotherapists could be a conservative group of people. It could be difficult to convince them into trying something new and different from what they are already used to. It should be mentioned that the market potential can become bigger if they decide to include other customers segments. This will be discussed in Chapter 9. As already explained the six months before the release will consist of the pilot project only. We assume that this will be carried out in two different clinics. This will not generate any revenue. Now we will assume that the pilot project has been successful and that the product is well-documented. In all the municipals there are close collaboration between the different entities in the health sector. Therefore we can assume that other physiotherapy clinics in Trondheim also will adapt to the game. In addition, the game will be adopted by the innovators that are eager to try new things. Let us say this will count for 48 sales, or 12 percent of the potential market share of 400 clinics. As clinics are starting to use the game and the word spreads about an useful and effective tool, more people will adopt. Roughly estimated, we believe the game will reach its potential market share in its fourth year. Then even some of the laggards who were sceptical to the product at first, start using it. We believe that most of the customers will adopt the game during the third year, because then it has had sufficient time to mature, see Figure \ref{fig:scurve2}. \\ \\ 
\begin{figure}
\begin{center}
\includegraphics[scale=0.4]{scurve}
\caption[The S-curve]{The S-curve \cite{scurve}}
\label{fig:scurve}
\end{center}
\end{figure}
\begin{figure}
\begin{center}
\includegraphics[scale=0.5]{scurve2}
\caption[S-curve for the exergame]{S-curve for the exergame. In this graph year 0 to 1 means the first year after the release.}
\label{fig:scurve2}
\end{center}
\end{figure}
\begin{figure}
\begin{center}
\includegraphics[scale=0.7]{fixedlowprice}
\caption[Price related to commercial video games]{Total cost and revenue with price at 300 NOK. We observe that Cyberlab has to sell about 10 800 units to gain some profit}
\label{fig:FixedLowPrice}
\end{center}
\end{figure}
\newpage
\textbf{Pricing}\\ \\
The product price depends on existing games and tools, and Cyberlab's costs. Estimating a suitable product price will be done by financial analysis and by looking at related, existing products on the market. Cyberlab's exergame falls under two definitions, a video game and a tool used in physiotherapy for training and rehabilitation. The video game market today exists of a huge amount of various games. They are mostly in an affordable price range, where e.g. Nintendo Wii games are priced between 99 - 499 NOK \cite{elkjopwii} and Xbox Kinect games are priced between 199 - 399 NOK \cite{elkjopkinect}. In addition there will be a cost of buying the needed hardware. Physiotherapy tools have more variation in price range as the definition of this tools are quite wide. Prices can vary from a fixed price of 120 NOK for a stretch pulley \cite{stretchpulley}, 11 000 NOK per month for shockwave therapy leasing (see Appendix C), up to 75 000 NOK for a treadmill \cite{treadmill}. Trying to sell the exergame for more than the existing video games on the market can be difficult. Therefore it is important for Cyberlab to promote their product as more than just a regular game to justify the price difference. This should be done by emphasizing the products value propositions, which relates the product more to physiotherapy equipment than "just" a video game. \\ \\
We will now present a detailed analysis of the two revenue models proposed in Chapter 8.4.2, and provide different price recommendations for the product. It is important to distinguish between Cyberlab's revenue and the package price presented for the customers. The product package shall as mentioned include the exergame and the Kinect sensor, but it is not the intention that Cyberlab shall make profit from reselling the Kinect sensors. The price a customer pays when buying a product package will balance Cyberlab's expense from purchasing a Kinect sensor, which makes it possible to neglect this cash flow from the analysis. Therefore, only the software price will be taken into consideration in our analysis. A suitable price for the software should cover all expenses related to the game. While Cyberlab's total cost is 3 238 775 NOK, we have for simplicity,  rounded the cost up to 3 240 000 NOK in our calculations. The Kinect sensor price is for the same reason rounded up to 2 000 NOK. \\ \\ 
\emph{Proposal 1 - Fixed Price Model}\\ \\
As the exergame in some way can be compared with video games, a reasonable price would be a price similar to other video games on the market. The price range of existing Xbox Kinect games are 199 – 399, so we choose to price the game at 300 NOK. How Cyberlab’s revenue will emerge with this price is shown in Figure \ref{fig:FixedLowPrice}. We observe that Cyberlab has to sell about 10 800 units to cover their total cost, which is an amount almost ten times higher than the potential market share of 1 200 physiotherapy clinics. We can therefore conclude that it will be impossible for Cyberlab to sell their game for a price as low as 300 NOK without experiencing huge economic loss.\\ \\
\begin{figure}
\begin{center}
\includegraphics[scale=0.8]{revenuestreamprice}
\caption[Price example]{The lowest possible package price Cyberlab can take, provided that they sell the max amount of 1 200 packages. This is 2 700 NOK per package}
\label{fig:RevenueStreamPrice}
\end{center}
\end{figure}
If we assume that Cyberlab has the possibility to sell their product to all of the 1 200 physiotherapy clinics in Norway, they could have a price as low as 2 700 NOK, see Figure \ref{fig:RevenueStreamPrice}, without gaining negative profit. However, it is very risky, and almost unreasonable, to assume that Cyberlab could reach out to all potential target customers. \\ \\
Figure \ref{fig:RevenueStreamQuantity} shows three income lines related to three different prices, and how much Cyberlab, with each price, has to sell to cover their costs. With a price of 6 000 NOK Cyberlab needs to sell at least 540 units to achieve non-negative profit, but with a price of 10 000 NOK they only have to sell 324 units, see Table \ref{tab:unitsfixed}. Selling to the expected market share, Cyberlab can price 405 units at 8 000 NOK, which will balance Cyberlab's profit. Figure \ref{fig:RelationPriceAndUnits} shows every combination of price and quantity that will make a revenue of 3 240 000 NOK, which covers all of Cyberlab's costs. However, we believe that Cyberlab is interested in earning more than 0 NOK, so we recommend to sell the software for 10 000 NOK. This price will be seen by the customer as 12 000 NOK, because of the additional cost of the Kinect sensor. We see that a fixed price package like this has a much higher price than existing video games on the market; however, it still is in the affordable range compared to other physiotherapy equipment. \\ \\
The sale of the product package will follow an earlier described S-curve, where we have distributed a potential sale of 400 games over a five year period. Table \ref{tab:revfixed} shows the expected revenue when selling 400 product packages. We observe that Cyberlab will gain a profit of 1 099 097 NOK by selling the software at a fixed price of 10 000 NOK. Figure \ref{fig:ProfitFixed} shows the relationship between the accumulated revenue, cost and profit over a five year period. From this figure we can see that Cyberlab will start to gain profit after about two and a half year. \\ \\
\begin{figure}
\begin{center}
\includegraphics[scale=0.7]{revenuestreamquantity}
\caption[Quantity examples]{This graph shows total cost and three revenue examples, which shows minimum number of units Cyberlab has to sell to gain profit}
\label{fig:RevenueStreamQuantity}
\end{center}
\end{figure}
\begin{table}
\centering
\caption[Price and unit examples with the fixed price model]{Relation between package price and units to be sold to cover the total cost of 3 240 000. See Appendix H for calculations.}
    \begin{tabular}{|l|r|r|r|r|r|r|}
        \hline
       \textbf{Fixed price}  & & & & & \\ \hline
      \textbf{Price} & 300 & 2 700 & 6 000 & 8 000 & 10 000 \\ \hline
	   \textbf{Units to be sold} & 10 800 & 1 200 & 540 & 405 & 324 \\ \hline	
    \end{tabular}
    \label{tab:unitsfixed}
\end{table}
\begin{sidewaystable}
\caption[Revenue with use of fixed price model]{Revenue (in \ac{nok}) on a per year basis with the fixed price model. Package price is 12 000 NOK, which generates a revenue of 10 000 NOK per package for Cyberlab. In this Table Cyberlab sells 400 units. See Appendix F for calculations}
    \begin{tabular}{|l|r|r|r|r|r|r|}
        \hline
       \textbf{Fixed price}  & & & & & & \\ \hline
      \textbf{Year} & \textbf{1} & \textbf{2} & \textbf{3} & \textbf{4} & \textbf{5} & \textbf{Total}\\ \hline
	   \textbf{Units sold} & 48 & 132 & 160 & 56 & 4 & \textbf{400}\\ \hline
	   \textbf{Revenue} & 576 000 & 1 584 000 & 1 920 000 & 672 000 & 48 000 & \textbf{4 800 000} \\ \hline  
	   \textbf{PV} & 553 846 & 1 464 497 & 1 706 873 & 574 824 & 39 453 & \textbf{4 339 097}  \\ \hline
    \end{tabular}
    \label{tab:revfixed}
\end{sidewaystable}

\begin{sidewaysfigure}
\includegraphics[scale=0.6]{relationpriceandunits}
\caption[Relation between price per unit and number of sold units]{This figure shows every combination of unit price and number of sold units which will cover Cyberlab's total costs}
\label{fig:RelationPriceAndUnits}
\end{sidewaysfigure}
\begin{sidewaysfigure}
\begin{center}
\includegraphics[scale=0.8]{profitfixed}
\caption[Profit, revenue and cost for a fixed price solution]{This graph shows relation between revenue, cost and profit in a fixed price solution. See Appendix F for calculations}
\label{fig:ProfitFixed}
\end{center}
\end{sidewaysfigure}
\newpage
\emph{Proposal 2 - Usage Fee Model}\\ \\
We will start by making an estimate of how much the exergame potentially will be used at physiotherapy clinics. Usually, a physiotherapist works approximately 8 hours each day, which includes 30 minutes lunch-break. Not every patient visiting the physiotherapy clinic during a regular day falls under the category "elderly", and not every elderly visiting the clinic has the need for or wants to use this exergame. We roughly estimate that the exergame will be used approximately 2 hours each day. A physiotherapist works 47 weeks a year, assuming they have 5 weeks of vacation. This add up to 470 hours a year where the exergame will be used.\\ \\
Our first package price proposal in this usage fee model is a start price of 2 000 NOK for the package and a price of 10 NOK for each hour physiotherapists use the exergame. During a year this will add up to 4 700 NOK if the game is used as much as estimated. Here the Kinect sensor is included in the package price. Therefore, since Cyberlab’s revenue should be calculated without consideration of the Kinect sensor, we subtract the Kinect sensor expense from the expected income. Each customer choosing the usage fee model will give Cyberlab an estimated revenue stream of approximately 2 700 NOK. For Cyberlab to be able to cover all costs they have to sell 1 200 packages. This means that they have to sell to all of the potential customers, which we already have evaluated to be unrealistic. We therefore provide another price proposal. We use the same startup price, but choose a price of 50 NOK for each hour the exergame is used. This will add up to 25 500 NOK, and without the Kinect sensor expenses Cyberlab will gain a revenue of approximately 23 500 NOK. In this case, Cyberlab only needs to sell 138 units to make a profit, a significant difference from the previous price example. See Appendix H for calculations. \\ \\

\begin{sidewaystable}
 \caption[Revenue with use of usage fee model]{Revenue (in \ac{nok}) on a per year basis with the usage fee model. We assume that the exergame will be used as estimated. For calculations, see Appendix F}
\begin{tabular}{|l|r|r|r|r|r|r|}
        \hline
       \textbf{Usage fee}  & & & & & & \\ \hline
      \textbf{Year} & \textbf{1} & \textbf{2} & \textbf{3} & \textbf{4} & \textbf{5} & \textbf{Total}\\ \hline
	   \textbf{Units sold} & 48 & 132 & 160 & 56 & 4 & \textbf{400}\\ \hline
	   \textbf{Revenue} & 1 128 000 & 3 102 000 & 3 760 000 & 1 316 000 & 94 000 & \textbf{9 400 000} \\ \hline  
	   \textbf{PV} & 1 084 615 & 2 867 973 & 3 342 626 & 1 124 922 & 77 261 & \textbf{8 497 399}  \\ \hline
    \end{tabular}
    \label{tab:revusage}
\end{sidewaystable}
\begin{figure}
\begin{center}
\includegraphics[scale=0.8]{profitusagefee}
\caption[Profit, revenue and cost for a usage fee solution]{This graph shows relation between revenue, cost and profit in a usage fee solution with startup price. See Appendix F for calculations}
\label{fig:ProfitUsageFee}
\end{center}
\end{figure}
\newpage
As with the fixed price model we assume that Cyberlab is able to reach the realistic market share of 400 physiotherapy clinics. Table \ref{tab:revusage} shows Cyberlab's revenue when selling 400 units over a five year period. We assume that the exergame is used as much as estimated. To be able to calculate a revenue stream created by a sold package, we have looked at the start price and usage fees throughout one year as one total income. See Appendix H for details. We observe that the income increases significantly in this usage fee example compared to the fixed price example. Where we in the fixed price example showed that Cyberlab achieved a profit of 1 099 097 NOK, they can with this usage fee model sell the same amount of units and achieve an almost five time greater profit of 5 257 399 NOK. Figure \ref{fig:ProfitUsageFee} shows the relationship between the accumulated revenue, cost and profit over a five year period. We observe that Cyberlab will gain profit after about one and a half year, one year earlier than with the fixed price model. \\ \\
Another solution for the usage fee model is for Cyberlab to give away their product package for free and just rely on the customers using the product. Figure \ref{fig:UsageWithWithout} compares the revenue streams when having an initial start price and when the product package is given away for free. We can see that there is not that much of a difference between these two examples. If Cyberlab gives their product package away for free, the usage fee still is 50 NOK, and customers use the exergame as much as we have estimated, they need to give away 151 packages to gain profit. \\ \\
\begin{figure}
\begin{center}
\includegraphics[scale=0.8]{usagewithwithoutstartprice}
\caption[Usage fee example]{Revenue with and without start price for the product package with a usage fee model. We assume that the exergame is used as much as estimated}
\label{fig:UsageWithWithout}
\end{center}
\end{figure}
We can conclude that the best solution for Cyberlab will be to use the usage fee model. There is a high risk related to profit, but the profit has the potential of becoming significantly higher than with a fixed price model. As mentioned, the low startup price for the usage fee model is not alone enough to cover all of Cyberlabs costs, so even if Cyberlab manage to cover the estimated market of 400 clinics, they will experience an economic loss if no one use the product. If the product never gets used, Cyberlab will not gain any revenue at all, which will make a negative profit equal to the estimated total cost. This is because the income of 2 000 NOK per package will be used to cover the Kinect sensor expenses. The loss will become even bigger if Cyberlab chooses the usage fee model that includes giving the product package away. However, in both cases, if all of the 400 customers use the exergame only one hour each day for one year, it will be enough for Cyberlab to gain profit. Also, if there is only one exergame at a physiotherapy clinic, the game would be shared among several physiotherapists, and we can expect the exergame to be used more. Expected lifetime for the exergame is approximately five years, so Cyberlab is actually only dependent on customers using the game one hour per week, which is very likely. \\ \\   
Choosing the model with startup price of 2 000 NOK instead of the give-away solution, gives Cyberlab a difference in profit of approximately 723 000 NOK when selling 400 units. From the customer's point of view there will be very attractive to get this product package for free. A possible drawback with the give-away solution is that customers do not have the same incentive to use a product they get for free as they have with a product they have paid for. With a free product package they have nothing to loose by not using it. Because of this, and the risk already related to the usage fee model, we recommend Cyberlab to choose the startup price solution. \\ \\ 
For the customers point of view a usage fee model will appear more appealing than a fixed package price because of the low startup price. Customers observe that the product price is higher than other existing video games on the market, but they also know what value propositions this exergame holds, which justify the price difference. They also have information about prices on equipment used at physiotherapy clinics, which makes Cyberlab’s usage fee model affordable. Customers also have other reasons for why they prefer this solution. One example is that a physiotherapy clinic might see a security in buying the product with a usage fee model. The startup price is manageable, and they can control additional costs themselves according to how much they want to use the game. \\ \\
Figure \ref{fig:RevenueAll} presents six revenue stream examples from both the fixed price model and the usage fee model. We have included three different fixed prices, two different usage fees, and usage fee with and without start price. We still assume that Cyberlab is able to sell 400 units, and that the game, with the usage fee model, is used as much as estimated. The revenue streams are presented on a per year basis, during a lifetime of five years. Number of units sold per year is distributed by the S-curve, described earlier in this section. We clearly see that the solution with a usage fee model, with a start price and a fee of 50 NOK, gives the highest revenue. By using this model, Cyberlab can expect to experience high profit. See Appendix F for calculations.  

\begin{sidewaysfigure}
\centering
\scalebox{0.55}
{\includegraphics{revenueall}}
\caption[Revenue examples for both fixed price model and usage fee model]{This graph represents several revenue examples for both the fixed price and the usage fee solution. Revenue is represented on a per year basis. See Appendix F for calculations.}
\label{fig:RevenueAll}
\end{sidewaysfigure}


\cleardoublepage
\chapter{Discussion}
This chapter will provide a discussion on our work and our thoughts around the business model. The emphasis will be on the aspects considered that are not included in the business model, and we will discuss and state the reasons for the choices we have made.  \\ \\
In chapter 7, we describe and study four possible customer segments, where we recommend Cyberlab to focus on public clinics and private clinics with contribution from the government. Remaining private clinics will at the moment not be considered as a target customer in the business model. These clinics have other decision making processes, different economy aspects and often a more limited customer base compared to public physiotherapy clinics. With no support from the government these clinics might have a more restricted economy, which complicates purchase of new products. As mentioned, private clinics do not have the same access to customers as public clinics, and one reason for this is that customers who consult private clinic have to pay for the therapy session themselves. Elderly may not have that kind of money or willingness to pay for a therapy session at private physiotherapy clinics (Appendix, mail fra Nina). We believe elderly will be challenging to reach through private clinics, so we recommend Cyberlab not to focus on this entity. \\ \\
We were thinking about elderly as a target customer segment for Cyberlab as they are the end users of the product. However, after working with and studied the business model with this group as a customer segment, we experienced that they are not the proper target for Cyberlab right now. Some reasons for this are worth mentioning. Elderly today usually do not have any connection to computer games or video games. Even though they might be used to technology like television and mobile phones, for them to adopt a new technology like video games will be very difficult. Many have never seen or used video games before, and they might not see the purpose of using a game like this. A developer trying to sell an exergame by presenting the products value proposition will not be sufficient to convince an elderly to use their product. We believe that it will be very difficult for Cyberlab to reach directly out to this customer segment. This might be more realistic in some years, when the exergame has been used at physiotherapy clinics and elderly has experienced the game with assistance in a safe environment.\\ \\
With the usage fee pricing model, the game has to be monitored to be able to calculate how much time the game has been in use. This has to be remotely controlled. In addition, the goal in the long run for this game is that patients eventually can start using this game at home as a supplement to their own regular treatment. This will also require remote control because in some cases the physiotherapists should be able to follow their patients’ progression. A system like this will most likely contain some private information of the user. To be able to customize the game, it will require a profile for each user, with all the information needed. This profile will also serve as a way for the physiotherapist to keep remotely track on the patient’s improvements and to adjust the program accordingly. It is important to acknowledge some of the issues introduced with the use of technology together with health care. The user profile will probably be provided on a webpage that can be retrieved from a remote server. Anything on the web always face some security issues, and we could spend a lot of time discussing all the different threats and vulnerabilities. However, this is out of the scope of this project and is rather a subject for future work. \\ \\ 
The value propositions mentioned in chapter 7 provide physiotherapists great advantages of owning an exergame, but after talking with physiotherapists and gathering knowledge about the new “Samhandlingsreformen”, we will recommend Cyberlab to focus on another value proposition. This will be to include the ability to be a multiplayer game, which makes it possible for physiotherapists to serve more than one player at the same time. To get a consultation hour at a physiotherapy clinic there is often long waiting lists, so an important aspect for physiotherapists is to be efficient to serve as many patients as possible, without losing quality in the work done. With a multiplayer feature, the exergame can be used to ease the physiotherapists’ workload. A physiotherapist can then consult and exercise with more patients in one hour than a physiotherapist can manage without the exergame. This will be valuable in the work of shortening the long waiting lists. \\ \\
In addition to the channels mentioned for raising awareness about the product, physiotherapists mention "NAV Hjelpemiddelsentralen" as a channel for knowing about existing health-related equipment. "NAV Hjelpemiddelsentral" is a Norwegian entity with competence in the area of equipment for health-related purposes. Their responsibility is to have expertise about existing equipment, communicate and facilitate for people who need help. They should know about different disabilities and provide customized solutions to help them. There exist different categories of equipment, where technical equipment is one of them. For a tool to be categorized as technical equipment is has to be helpful and necessary for increasing a user’s physical abilities. This imply; being a tool that can help a functionally disabled person to better manage everyday activities. The idea is that people with disabilities can apply for financial support from the government for buying this type of equipment. As Cyberlab’s exergame has the intention of increase physical strength and promote physical activity, and not qualifies as absolutely needed equipment, it will most likely not be categorized as technological equipment \cite{navhjelpemiddel} \cite{navhjelpemiddelsentralen}.  \\ \\ 
Through conferences, magazines or just word of mouth physiotherapists can also get an impression of which products that are popular right now. The popularity mark might attract customers. Physiotherapists working at private clinics, with a more limited access to customers than public clinics has, might chose to buy a popular product to attract new and more customers. This will not have any functionality if every clinic owns the product, and it will neither be relevant for Cyberlab’s customer segment, as they do not need to attract customers. \\ \\ 
We have suggested that Cyberlab shall start with focusing on delivering the game to clinics in Trondheim. Selling the game in only Trondheim to an acceptable price will not make profit. We believe that Cyberlab is looking at some years before they will make profit. The physiotherapy community is small and they have a lot of similarities. By the time this game has become well-known in Trondheim, the game should have gotten attention from clinics in other cities as well. We suggest that Cyberlab hire some of the physiotherapists that are already using the game to present it on conferences and teach other physiotherapist how to use it. In this way, the game will be both more believable to other physiotherapists, as well as Cyberlab do not need to do the hazel with travel around promoting their game. \\ \\
In chapter 7 we propose, describe and analyse two possible revenue models for Cyberlab, a fixed price model and usage fee model. We will mention two other revenue models we have decided not to focus on, because we do not feel they were appropriate for Cyberlab in this business model. \\ \\
\emph{Licensing:} \\ A possible revenue model we studied was a license agreement, where Cyberlab could generate revenue by selling permissions to physiotherapists in exchange for a license fee. This will allow physiotherapist to generate revenue by resale of the exergame to elderly. Elderly trust physiotherapists and their products, and would therefore be a good channel for bringing the exergame home to elderly. However, we decided to not look deeper into this revenue model for the same reasons as for why elderly not is a suitable customer segment for Cyberlab.\\ \\
\emph{Advertising:}\\ Another revenue model we took into consideration was a game financed by fees for advertising. As with the license agreement, we experienced that this not is an appropriate option with regard to the end users of the product. Feedback from the interviews stated that a video game for elderly has to be as easy to use and as intuitive as possible, with no disruptive elements. A video game with ads will not appear user friendly as is would be confusing and maybe remove focus from the game.\\ \\
From our interviews we experienced that there is a demand for receiving a product package with services like delivery, installation and introduction. Our first thought was therefore to include these services in all the packages, and add wages for installation and expenses related to travel in the package price. After studying this solution in more detail we saw that this could be very expensive for the customers. The price would also vary in accordance to distance and help needed. In addition, it was not possible for us to calculate some reasonable numbers related to this cost, and it has therefore not been taken in to consideration in our analysis. However, we recommend that Cyberlab should offer the possibility to deliver this kind of services as there is a demand for it. \\ \\   
We have made some additional assumptions when it comes to what should be included in the package Cyberlab is going to sell to their customers. In our calculations we have included the Kinect sensor only, but ideally, they should offer a package with a Windows machine and a screen as well. If these packages were offered, Cyberlab would have to buy the extra equipment. If they could buy the equipment in big quantities, they could buy them from a wholesaler to cost price, and sell with profit. However, this would require Cyberlab to have a huge stock. To keep a stock is both risky and inconvenient. Having a number of TV-sets and computers stored in their premises will be very space consuming. In addition, it is risky in the sense of not knowing how much they will sell. Also, there might be new versions and upgrades that will make the "old" versions unattractive. This especially accounts for the Kinect sensor. Having a stock will also mean that Cyberlab needs to have the liquidity to be able to buy the quantities, because they will not sell it right away. In addition, this means that Cyberlab has to predict how many customers they will sell to during a fixed period of time, which there is very hard to do. \\ \\
Another prediction we have done in our calculations is how much workload is required for each task. It is hard to predict how many errors and bugs will occur in software and therefore it is impossible to know how many hours are spent on this per year. To make these predictions more precise, we would have needed more experience with software developing. We assume Cyberlab has this kind of experience, and will probably be more suited to make this kind of predictions.\\ \\
The workload assigned the different tasks do not always equal full-time. For example a FTE=1.0 can mean that one person is assigned the project 100 percent for one year, it can mean that four people are assigned the project 100 percent for three months, or it can mean that four people is assigned this project for one year, but only with 25 percent workload each. We have not specified how many people will work on each task in this project. It will be up to Cyberlab to decide what is most efficient for them. The marketing task, is only assigned a FTE= 0.5 in the first year after the game’s release and FTE=0.2 the remaining four years. It Cyberlab does not already have a marketing person; they can consider to hire a marketing consultant to do this job. \\ \\
One of the reasons of choosing Microsoft Kinect as the technology for this game is the free, open SDK, that enables a third party to develop games for the platform. What we have not been able to find out, is what rights developers have to sell this game with a profit. It is important for Cyberlab to form some kind of agreement with Microsoft telling who is getting what from the sales of the game. \\ \\
Cyberlab’s key partners are Norut and Microsoft. This means that they are dependent on them to be able to make this game.
Physiotherapists are Cyberlab's customer segment, but they can also serve as a partner. This game fits very well into where the Norwegian health system is going towards (see “Samhandlingsreformen” chapter 3). We mean that this game has a potential to work as a tool for “everyday rehabilitation”. A goal for Cyberlab should be to get this game in as a “regular” tool for prevention and rehabilitation. The government is responsible for the health sector, so they could be a natural partner for the future. Having the government as a partner will solve most of the financial issues. Getting this game into a medical program financed by the government will make this game very credible for the end user. The government could then provide the game to hospitals, care centers, physiotherapy clinics, training groups, and for special cases, when the elderly for example need the game in their own house. \\ \\ 
Et lite avsluttende avsnitt her.

\cleardoublepage
\chapter{Conclusion}
There is great uncertainty related to the potential of this game. It is a new technological tool which will face a small and immature market. In addition, the fact that the game is not yet developed made it hard for us to understand the game, as well as describe the game to physiotherapists that we interviewed. However, with support in the Norwegian health sector's new focus and a positive attitude from physiotherapists we have talked to, we believe that the game has a potential on this market. The great amount of research that has been done on video games for health-related purposes the last couple of years, also suggest potential for the exergame. This requires Cyberlab to develop an entertaining and easy-to use game, customized with the right exercises for elderly. If they do this, they can expect to become successful in this market and gain significant profit.  
\section{Future Work}
Future work for Cyberlab will be to find out how their exergame should be made. They know which exercises that are good for the target user, but they have to find a way to integrate them.  This should be done in a way that makes the game entertaining and easy to use, which also can be customized for the users’ needs and interests. As a part of this, it needs to be done more research on which type of games that are suitable for elderly.  In the making of the exergame, they should suggest to offer the opportunity to multiplay, as this would most likely be much appreciated by both the physiotherapists and the end user. \\ \\
The infrastructure of the game has to be well thought through. They how to find out where data will be stored, and how the data can be transferred from entity to entity. This includes the security aspects of having private information stored on a server.Cyberlab should also take the ethical aspects of storing and processing personal data into account. \\ \\
Last, we will suggest to look into what kind of business potential the game has with other customer segments. This includes customer segments discussed in Chapter 9, but especially other countries in Europe, since this exergame is developed as a part of an EU-funded project. The market potential will increase tremendously if they expanded to the rest of Europe. This requires new business models. \\ \\
\cleardoublepage
\bibliography{bibl}
\bibliographystyle{unsrt}
\addcontentsline{toc}{chapter}{Bibliography}
\pagenumbering{gobble}
\addcontentsline{toc}{chapter}{Appendix}
\cleardoublepage
\appendix
\chapter*{Appendix}
 Appendix A, B, C and D, include the interviews reports and will be provided in Norwegian only.

\section*{Appendix A - Intervju 22/10/12 kl. 1300}
\label{A}

\emph{Intervjuobjekt:}\\
Navn: Jorunn Helbostad \\
Utdannelse: Hovedfag og doktorgrad i fysioterapi, ved Universitetet i Bergen.\\
Arbeidssted: Forske på St. Olavs Hospital, Trondheim\\ \\
\emph{-Av deres pasienter fra 65 år og oppover, hva er problemet? Har dere pasienter som er hos dere for opptrening etter for eksempel en skade (rehabilitering)? Har dere pasienter som er hos dere for generell trening fordi de ønsker å styrke kroppen?}\\
Det finnes private og kommunale fysikalske klinikker. Disse har avtale med trygdesystemet. For å få time hos fysioterapeut må man ha rekvisisjon. Pasienten kommer  i kontakt med fysioterapeuter gjerne fordi de har et definert problem. Det hender for eksempel hjemmesykepleiere tar kontakt på vegne av en pasient de pleier. Det er sjeldent den eldre tar kontakt selv. \\ \\
\emph{-Hvordan er oppfølgingen under behandlingen? Hvordan er oppfølgingen etter behandlingen? Pleier dere å gi pasienten program som de må trene på hjemme mellom hver time?}\\ 
Ofte er det ikke nok å være hos fysioterapeuten 1-2 ganger i uka. Det er en utfordring å få de til å gjøre noe hjemme. De eldre ønsker gjerne å "bli friske", de er ikke veldig motiverte til å trene hjemme på egenhånd. Vanlig å oppfordre til å bevege seg mer hjemme, enten ved å gi en form for treningsprogram eller si noe som "husk å være fysisk aktiv". Dette kan være at fysioterapeuten skriver en lapp med strekmennesker som forklarer øvelser, eller et dataprogram hvor man kan sette sammen et program med øvelser og printe ut til pasienten, eller noen sier bare noe så generelt som at de må ut å bevege seg.\\ \\ 
\emph{-Opplever dere at det er pasienter som har problemer med å komme seg til behandling? Hender det dere må dra på hjemmebesøk?}\\ 
Nei, de som er dårlige til beins, får dekket drosjetransport til behandlingen. Men det er klart at mange vegrer seg for å gå ut. Det er også mange som vegrer seg for å bevege seg innendørs.\\ \\
\emph{-Er det noen som uttrykker at de ønsker hyppigere trening?}\\ 
Sjedent. De vil vel gjerne få en slags "pille/medisin" og bare bli frisk og rask.\\ \\
\emph{-Er det mange som uttrykker ensomhet/ulykkelighet?}\\ 
Det er veldig få som identifiserer seg selv som en person som er redd for å falle. I prosjekter hvor det har blitt foreslått forskjellige tiltak blir man ofte møtt med svar som "Det høres ut som en fin ting. Gi det til noen som trenger det". Det er mange som ikke sier ifra at de har falt. Å forebygge noe som "ikke har skjedd" er vanskelig. Dersom man jobber med fallforebygging, bør man ikke nevne ordet "fall". Det bør fokuseres mer på positive ting, som å styrke kropp for å kunne leke med barnebarna, gå på kafe osv.\\ \\
Det er opprettet noen treningsgrupper i Trondheim som er ment å forebygge fall. Men de reklamerer ikke med dette. I stedet reklamerer de med feks: "Vil du greie mer enn før...?".\\ \\
\emph{-Hvordan får dere høre om nye behandlingsmetoder, hjelpemidler, vektøy osv?}\\ 
Vi har "Fysioterapauten" som er et tidsskrift for fysioterapauter. Her blir stilen holdt ganske ren. Ellers er det jo også artikler i blant annet aviser og magasiner. Man drar på kurs og konferanser, men da gjerne innenfor et bestemt fagområde. Et bra sted å lansere nye produkter er kanskje på konferansene eller i "Fysioterapauten". Det er ofte snakk om etterutdanningskurs, ikke så mye om nye produkter. \\ \\
\emph{-Hva er interessen for nye ting?} \\ 
Her er det snakk om en ganske konservativ gruppe. Man vil veldig gjerne ha en dokumentasjon på at det fungerer. Produktet bør ha en lav brukerterskel. Hvor lett er det å bruke? Det må lette arbeidsmengden eller forbedre arbeidet for at det skal være interessant å ta i bruk. Man må også finne ut hvem som skal betale dette. Helsesektoren? Kommunen? Man må gjerne "stå for produktet" og klare å få fram at det er verdt å betale for. Pris på produktet har nok mye å si! \\ \\
\emph{-Må nye produkter være godkjent for medisinsk bruk?}\\ 
Et spill som dette havner litt i en gråsone. Det finnes lover og regler, men jeg tror ikke man trenger medisinsk godkjenning for å ta i bruk dette spillet.\\ \\
\emph{-Hvordan foregår en kjøpsprosess hos dere?} \\ 
Det vanligste er nok at man kjøper for å eie selv. Det er interessant å leie eller prøve produktet en viss tid for å være sikker på at det er et godt kjøp. Ofte skjer det at man kjøper inn et produkt, men så blir det gjerne liggende fordi man ikke tar seg tid til å lære det. Her kreves det opplæring! Det som også gjerne skjer er at en ivrig person tar initiativ til å kjøpe et nytt produkt og lærer seg hvordan det skal brukes, for så å kanskje slutte. De gjenværende har ikke lært seg å bruke produktet, og så blir det liggende. \\ \\
\emph{-Hender det at dere kjøper inn produkter for så å selge dem videre til kundene deres?}\\ 
Fysioterapautene kunne jo kjøpt spillet og eventuelt en lisens med på kjøpet, men jeg tror kanskje at eldre vil vegre seg for å kjøpe. Hvis kommunen så på det her som noe bra, så kunne kommunen ha kjøpt inn og lånt ut til eldre. \\ \\ Det er alltid en utfordringen med ny teknologi - hvem skal betale? Prosjekter har strandet fordi man ikke blir enige om hvem som skal betale.\\ \\ På fylkeskommunalt nivå har man hjelpemiddelsentralen. Hjelpemidler som kan lette hverdagen til folk kan bli kjøpt inn av hjelpemiddelsentralen og leid ut videre. Jeg er ikke helt sikker på hva som er grensen mellom trening og "fungere bedre i hverdagen" \\ \\
\emph{-Hva slags forhold har dere til leverandørene deres?} \\ 
På avansert utstyr kan man kjøpe serviceavtale. Men det er gjerne ingen som kjøper fordi det er for dyrt. Det er behov for oppgraderinger og oppfølging. Man ønsker gjerne tilpassede programmer, det vil gi større lyst til å prøve/bruke produktet. Sånn sett foretrekker jeg å samarbeide med små bedrifter, for da kan det være lettere.\\ \\ 
\emph{-Hva tenker dere om å bruke det videospillet som vi har beskrevet som en alternativ og annerledes behandlingsmetode \\
-for generell trening?\\
-tilpasset rehabilitering?}\\ 
For å kunne si noe om dette, ville jeg sett og prøvd spillet. For at det skulle vært interessant måtte det kunne lette arbeidsdagen min som fysioterapeut og gi meg muligheten til å tilby bedre hjelp til pasientene. Dersom jeg ikke syns øvelsene er relevante, ville jeg ikke brukt spillet. Spillet må være bedre enn det jeg kan tilby selv og øvelsene må kunne tilpasses. Når jeg har en pasient vil jeg finne ut hva som er pasientens problem ved å undersøke pasienten. Ut i fra problemet jeg finner, vil jeg legge opp et program ut ifra hver enkelt pasient. Innhold og vanskelighetsgrad må være tilpasset behovet. \\ \\ 
\emph{-Hva slags verdi tror du er bevegelsesstyrkende videospill kunne gitt til en bruker?}\\ 
Det kan oppleves både som spennende og som en barriere for pasienten. Mange eldre opplever teknologi som en barriere. Spillet må fenge pasienten. Spillet bør ha mulighet for individuell tilpasning. For å redusere fall bør øvelsen inneholde balanse og styrke og det må være mulig å tilpasse vanskelighetsgrad slik at det kan bli vanskeligere. Med øvelser med fokus på styrke og balanse er det bevist at man kan redusere fall med 20-60 prosent. Det teknologi kan bidra til er å gjøre det mer underholdende og motiverende. Tilbakemelding er en viktig motivasjonsfaktor. Dersom man for eksempel får tilbakemelding på at øvelsen du gjorde tilsvarte at du var 10 år yngre, ville du sannsynligvis trene en ekstra gang dagen etter. De fleste pasienter ville ikke tatt i bruk et slikt produkt på egenhånd. Måtte fått det anbefalt av for eksempel fysioterapeut. For at fysioterapeuten skal kunne følge med på pasientens progresjon, må det være lett tilgjengelig for dem.\\ \\ 
\emph{-Generelt} \\ 
Dere bør sjekke ut hjelpemiddelsentralen på www.nav.no. Her kan dere lese litt om regelverk. Folketrygden dekker ikke sport- og fritidsutstyr. Dere må tenke på: hvis spillet skal brukes, hvordan får dere fysioterapautene til å si ja? Hvordan får dere fysioterapautene med på laget? Det kan sikker være en lur idé å snakke med både fysioterapauter og ledelse.\\ \\ Sånn til slutt så vil jeg si at jeg har tro på dette prosjektet!\\ \\
\emph{-Nye kontakter} \\ 
-Fysikalsk institusjon - fastlønnet stilling\\
-Høre med Sylvi Sand, hun vet hvem av de private klinikkene som driver med eldre. Har ansvar for fagutvikling \\
-Pensjonistenes fellesorganisasjon - Hornemannsgården. Spørre spørsmål angående forebygging. Høre med ledelsen at det er ok, når det passer. 

\newpage
\section*{Appendix B - Intervju 07/11/12 kl. 0830}
\label{B}
\emph{Innledende}\\ \\
Vi er kommunalt drevet. Dette er et gratis tilbud for de som ikke kan benytte seg av privat institutt eller trenger et mer sammensatt tilbud hvor kartlegging og forflytning i hjemme er viktig, hverdagsrehabilitering). Privatpraktiserende med driftstilskudd får fast lønn og er også fastansatte av kommunen, hvor HELFO i tilegg dekker en del av lønnen. \\ \\
Private institutter har sin måte å jobbe på, som er avhengig av at pasienten kommer til fysioterapeuten i en treningssal oftest + benkbehandling. Jeg er usikker hvor mye kompetanse fysioterapeuter ved institutt har i fallforebyggende tiltak. Her kan det være utfordrende å implementere ny teknologi som ikke er gjort så mye forskning på. Fysioterapeuten ved institutt må også kjøpe produkter selv. Har man lokaler til et slikt spill dere tenker på? Man må ha plass til tv og plass til å bevege seg. Kan det være et problem at man trener med mange andre? Kan det kanskje være ganske forstyrrende for andre pasienter som benytter samme treningssal? \\ \\
For at kommunale fysioterapauter skal ta i bruk nye produkter må det være begrunnet i forskning eller pilotprosjekter. Det er ganske høy terskel når det ikke er dokumentert for å ta i bruk nye ting. Med dette spillet ville nok et pilotprosjekt hvor noen klinikker prøvde det ut gratis i for eksempel 6 måneder vært nødvendig. Og så må man dokumentere effekten i artikler.  Eldre må også finne det nyttig, og de må kunne bruke produktet. Det er vanskelig for dem å venne seg til ny teknologi. I tillegg handler egentlig alt om økonomi. Man er ganske nøye på hva man bruker penger på.  Hvis dette er et produkt som vil effektivisere måten fysioterapeutene jobber på samt til at man kan dokumentere effekt av å bruke et slikt produkt vil økonomien ikke være et problem.\\ \\
Vi jobber med voksne og seniorer. Multifunksjonelle. Vi er i bildet en periode, før institutt senere tar over. Man skiller mellom over og under 18 år.  \\ \\
Det er i kommunen den største satsningen på fallforebygging skjer. \\ \\ 
\emph{Intervjuobjekt}\\ \\
Navn: Ena Zvizdic\\
Alder: 33\\
Utdanning: Bachelor i fysioterapi på HIST. \\ \\

\emph{- Av deres pasienter fra 65 år og oppover, hva er problemet? }\\
- Har dere pasienter som er hos dere for opptrening etter for eksempel en skade (rehabilitering)?\\
- Har dere pasienter som er hos dere for generell trening fordi de ønsker å styrke kroppen?\\
Vi får som oftest pasienten inn når hverdagsaktiviteter er blitt en utfordring. Det er mange som ikke ønsker hjelp før de faktisk begynner å slite med forflytning i hjemme. Vi ønsker å ta de inn tidligere, som for eksempel allerede når de ber om trygghetsalarm. Det finnes noen treningsgrupper, “seniortrim”, hvor de eldre kan møte opp, betale 30 kr og få delta på en treningstime. Dette kan jo være et interessant sted for dere å besøke, så kan dere også snakke med de eldre.  For å forklare spillet, så kan dere jo vise bilder av Xbox og forklare at man kan spille og få opp ting på en TV-skjerm. Kan også spørre om de har noe forhold til slike spill eller om kanskje de har hørt om det gjennom barnebarn. Viktig å få tilbakemelding fra ”brukeren” om hva de synes og hva slags forhold de har til teknologi. Her er det store individuelle forskjeller.\\ \\
\emph{-Opplever dere at det er pasienter som har problemer med å komme seg til behandling?}\\
Ja, det er et stort problem. Ofte når de blir henvist til oss, er de redde for å bevege seg utendørs, spesielt om vinteren. Det er også mange som er redde for å falle inne. Mange har for eksempel blitt rulatorbruker, og når man er avhengig av et hjelpeverktøy kan det være vanskelig å gi slipp. Det går en del utover selvtilliten. \\ \\
\emph{-Er det noen som uttrykker at de ønsker hyppigere trening?}\\
-som selv tar initiativ til å drive med fysisk aktivitet?\\
Dette er veldig personavhengig. Det går en del på sårbarheten til pasienten. Hvis det er snakk om en ressurssterk person så er de ofte mer motivert for å komme seg videre. De som er mer sårbare trenger kanskje mer oppmuntring. Noen pasienter er så dårlige og langt nede at vi jobber mest for å se til at de klarer å vedlikeholde den funksjonen de har.\\ \\
\emph{-Er det mange som uttrykker ensomhet/ulykkelighet?}\\
Dette avhenger av hvor lenge problemet har vart. Noen slår seg til ro med at ting er som det er. Vi har brukere som takker nei til behandling/opptrening fordi de ønsker å klare seg med den hjelpen de har fra kommunen i tillegg til støtte fra familien. Og så har du de som står på, for eksempel en 97-åring som ikke bruker noen form for hjelpemidler eller har bistand fra Trondheim kommune.\\ \\
Kan ikke se på eldre som en og samme målgruppe. De er også forskjellige mennesker med forskjellige behov og interesser. Noen har for eksempel vært veldig fysisk aktive i form av trening når de var yngre, mens andre kanskje har vært fysisk aktive i form av å jobbe i hagen og har derfor ikke så mye forhold til annen fysisk aktivitet i form av trening.\\ \\ 
\emph{-Pleier dere å gi pasienten program som de må trene på hjemme mellom hver time?} \\
Det er litt forskjellig. Da må vi se på dette med ressurser. Og hvor motivert er pasienten? Vi kan bruke ekstra tid på pasienten fordi vi er fastlønnet, for å gi best mulig kvalitet på tjenesten. Dette gjør også at vi ikke klarer å ha så mange pasienter på en dag. En utfording med dataspill er at vi mister kontakten med pasienten. Det blir vanskeligere å gi tilbakemelding. Det er viktig med tilbakemelding så de gjør øvelsene rett. En annen utfordring med spillet er at de ikke ser seg selv i bevegelse. Når vi er til stede så kan vi utfordre pasienten (med tanke på balansetrening) og vi kan også sikre pasienten til å ikke ramle under utførelsen av øvelsen. Vi har den faglig kompetansen og blikket til å utfordre dem litt ekstra enn det de klarer å gjøre selv. Det er ofte det som har en stor betydning om de får fremgang. Instruksjon i utførelsen av øvelsene er viktig og det å gi dem tilbakemelding på det de mestrer og ikke mestrer.  \\ \\
Dette dataspillet må være noe veldig enkelt. Ikke for mange tastetrykk i menyen. Maks to trykk. En mulighet til å programmere øvelsene ville vært veldig bra. Da kunne vi kommet til brukeren “som tilfeldigvis” hadde Xbox og programmert et tilpasset program for denne brukeren. \\ \\
\emph{-Hvordan får dere høre om nye behandlingsmetoder, hjelpemidler, verktøy osv?}\\
Litt forskjellig. Vi har NAV-hjelpemiddelsentralen som fatter vedtak rundt hjelpemidler. Noen hjelpemidler går gjennom bestillingsordning som en godkjent rekvirent/bestiller kan bestille uten å måtte gjengi begrunnelsen i søknaden. “Godkjent- bestiller” vil si at personen har fått opplæring til å foreta en vurdering hvorvidt en pasient trenger et hjelpemiddel. Andre hjelpemidler må man søke om med en grundig begrunnelse.. Treningshjelpemiddel får man bare støtte til frem til fylte 26 år. \\ \\
Det å sende ut et blad med reklame om produktet tror jeg ikke vil fungere noe særlig. Man er opptatt av å prøve produktet og få vite at det fungerer. Det å kunne prøve ut spillet deres gratis i 6 måneder og prøve det på pasientene være kunne vært interessant. Det å dokumentere effekten av et produkt gjennom forskning er KJEMPE VIKTIG. Man er forsiktig med hva man anbefaler videre. \\ \\
Vi vil gjerne få tak i de eldre før de blir for dårlige. Ofte kommer vi ikke i kontakt med dem før de begynner å streve med det hverdagslige.\\ \\ 
\emph{-Hvordan er det med godkjenning av nye produkter?} \\
Jeg kjenner ikke  krav rundt å godkjenne nye produkter/hjelpemidler. For oss vil det som sagt være viktig at det er en påvist effekt ved bruk av hjelpemidler. Noen hjelpemidler fungerer i veldig varierende grad. \\ \\
Hvis jeg hadde funnet noe jeg ville bruke måtte jeg tatt opp med andre kollegaer og så til enhetslederen. Kommunen gir en viss sum penger til de forskjellige enhetene, så vi som en fysioterapienhet får da en liten pott til kurs, verktøy osv. Enhetslederen min vurderer ut i fra behov hvor mye som skal brukes til hvilke formål. fysioterapeuten mener  at et treningsverktøy er nødvendig for å gi et tilbud med kvalitet, vil enhetsleder vurdere det. \\ \\
\emph{-Hvordan er det med utstyret deres, kjøper dere det selv eller leier dere?} \\
Vi kjøper alt selv. Utenom bilene våre, de leies. I kommunen er det fire bydeler med hver sitt fysioterapisenter. Skulle alle sentrene hatt alt ustyr ville det blitt dyrt for oss. Forskjellige verktøy er fordelt utover de forskjellige bydelene. Fysioterapauter kan åne ut utstyr/hjelpemidler til pasienter gratis. Der er det knapt på ressursene, etter som man ikke har så mange av hvert hjelpemiddel. Vi er drevet av folks skattepenger, så vi er nøye med hva vi bruker penger på.\\ \\ 
\emph{-Hvordan mottar/får dere nye produkter? Får dere bare produktet for å finne ut av det selv eller følger det med kurs/installasjon på kjøpet? }\\
Hvis vi f.eks. kjøper ny elektrisk rullestol, som kan være et viktig produkt for mange av våre pasienter, så kan det hende leverandøren kommer og viser hvordan den fungerer. Leverandørene kan da komme tilbake når han har et nytt produkt, for så å spørre om han kan vise det for oss. Vi har personalmøte hvor alle i kommunen samles, og dette kan være et veldig bra sted å promotere produktet som har vist seg å ha en dokumentert effekt. Vi er dessverre ikke interessert i å høre reklame på presonalmøter. Det som tas opp på personalmøter(nettverksmøte) må være relevant for faglig utvikling eller drift av enheten. Dette vurderes av fagleder og enhetsleder. Ansatte får også komme med tips i forhold til hva de ønsker at skal tas opp på nettverksmøter.
Personalmøter gjelder hele enheten, på nettverksmøter deles vi i fysioterapeuter som jobber med eldre/voksne og fysioterapeuter som jobber med unge/barn. \\ \\
\emph{-Har dere noen kontakt med leverandøren etter at kjøpet er ferdig?}\\
Etter første kontakt kan det hende leverandørene sender brosjyrer med informasjon om nye produkter og lignende. Disse ligger da rundt på kontoret, og man leser dem når man har tid. Det er veldig matnyttig. Det kan forresten være en tanke å nå de eldre med avis. De sjekker ikke så mye nyheter på nett, og de følger kanskje ikke like godt med på TV-reklame, men aviser leser dem. \\ \\
\emph{-Hva tenker dere om å gi tilbakemeldinger på nye produkter?}\\
Vi er veldig ivrige etter å gi tilbakemeldinger på nye produkter. Det hadde også vært interessant om vi kunne fått tilbakemelding fra hvordan det går med produktet. Om man kjørte prøveprosjekter for flere institutter og kommuner kunne det vært interessant med tilbakemeldinger på hvordan det går, hvor bra får pasienter utført øvelsene sine, hva er pasientopplevelsen og hva er opplevelsen av produktetet for fysioterapautene. \\ \\
Jeg har tro på dette prosjektet, og tror det kan bli ganske stort om kanskje 10 år. Det vil da være når de som har et forhold til teknologi blir eldre. Det er bare et spørsmål om tid. Gruppen eldre mennesker som nå er 80 år vil kanskje være en vanskelig målgruppe å nå. \\ \\
Jeg var på en forelesning der det var snakk om velferdsteknologi. Her er det blant annet bruk av sporingsteknologi. Sporingsteknologi er utstyr som kan beregne og opplyse om geografisk posisjon. I dag finnes f. eks GPS-løsninger som kan bæres på kroppen, festes på en rullator eller liknende, til nytte for personer med svekket orienteringsevne samt for deres omsorgsansvarlige. Hvordan man etter hvert kan bruke roboter i helsetjenester. Ideen bak bruk av roboter er også spennende. Da kan pasienten ha en robot hjemme hos seg og så kan en sykepleier sitte på et annet sted å styre roboten ved å for eksempel spørre om de har tatt medisinen sin og lignende. Sånn sett er jo dette med teknologi veldig i vinden allerede.\\ \\
Trondheim kommune var en del av en storprosjekt (Eldergames) på dataspill som skulle utvikles og testes ut og tanken var at man skulle bruke dette verktøyet til å kartlegge, følge opp og trene kognitiv funksjon. Dette var da et slags hukommelsesspill, hvor 
4 personer satt og spilte sammen. Et av utfallene ved denne undersøkelsene var at det sosiale var svært  viktig for deltakere. Det var 20 stk som deltok, og kun en av dem var mann. Damer var lettere å rekruttere for å prøve ny teknologi, noe som er ganske interessant. Etter hvert som spillet ble utviklet kunne man spille på tvers av landegrenser, f.eks. med en i spania. Man kunne sende ikoner til hverandre, for eksempel smileys, slik at man kunne kommunisere uten at man trengte å kunne språket. Dette spillbordet har de på Valentinlyst velferdssenter i dag, men det ble ikke kjøpt av noen andre. Kanskje på grunn av pris? \\ \\
Websiden til projektet: {http://www.eldergames.eu/} \\ \\
\emph{-En tanke med dette spillet er at det skal være arbeidsavlastende for fysioterapeuter. For eksempel kan man behandle 3 pasienter samtidig på en time i stedet for bare 1.  Hvem tenker du at dette er mest fordelsaktig for?}\\
Dette er en fordel for fysioterapeutene. Vi har ventelister. Samhandlingsreformen som kom i 2012 utfordrer oss i å tenke og jobbe litt annerledes, mer forebyggende og helsefremmende arbeid. Målet med samhandlingsreformen er å forebygge mer, behandle tidligere og samhandle bedre. Dette gjør også at kommunen skal overta mange av oppgavene andre linjetjenester har hatt. Dette kan fort føre til større press på alle enhetene i Trondheim kommune, også fysioterapitjenesten. Hvis spilelt viser seg å fungere bra med tanke på effekt og kvalitet og at vi i tillegg kan spare tid på pasientene, ville vi nok brukt det. Det blir kommunen som sparer penger, fordi det er de som betaler. Det økonomiske wil ikke ha noe å si for fysioterapeuten direkte. Spillet må ha effekt og det bør motivere pasienten. 

\newpage
\section*{Appendix C - Intervju 08/11/12 kl. 1500}
\label{C}

\emph{Intervjuobjekt:}\\
Navn: Nina Skjæret\\
Alder: 26 \\
Utdannelse: fysioterapi på HIST og master i bevegelighetsvitenskap på Dragvoll. \\
Arbeidssted: Ilen Fysioterapi og Idrett\\ \\
\emph{Litt generell snakk i begynnelsen:}\\
Vi har en god blanding pasienter. Pasientene våre er de som er villige til å betale for å slippe å stå i kø og for å få god oppfølging. Det er mange som ikke vil betale for privat fysioterapitime, selv om det egentlig bare kanskje er 50 kr mer enn egenandelen et annet sted. Dette med betalingsvilje gjelder alle, men kanskje spesielt eldre. Eldre pasienter blir ofte gående til fysioterapeut lenger enn andre pasienter. \\ \\ 
\emph{-Kommer pasienten til dere tidlig i fasen av en plage eller først når de har blitt veldig dårlige?}\\
Det varierer veldig. Det kommer ann på hvor ressurssterk pasienten er. Noen er vant til å gå tur i marka, gå på ski osv. De vil typisk komme tidlig, når de kanskje merker de ikke klarer dette lenger f.eks på grunn av et vondt kne. \\ \\
\emph{-Litt om hvordan dere jobber med denne aldersgruppen. Du forteller i mailen at dere har seniortrim med fokus på balanse, beveglighet og styrke. Hvordan kommer dere i kontakt med disse personene i første omgang? Hva er deres motivasjon for å komme dit?}\\
Vi bruker annonse i Adressa-avisen. Nå har vi en veldig stabil gruppe. Vi har en treningstime i uken, men etter jul vurderer vi å ha flere timer. Ikke nødvendigvis for at de samme skal kunne komme flere ganger hver uke, men fordi det er ikke alltid et tidspunkt passer for alle. Vi vel derfor ha flere treningstimer for at flere skal få mulighet til å trene, og vi vil ha det på forskjellige tidspunkt så det passer for flere. Har treningsgruppene i perioder, for eksempel på høsten, fra jul til påske og fra påske til sommer. Noen kvier seg for å bli med i en allerede oppstartet gruppe. Det er greit å ha en egen periode fra påske til sommer, for da er det mange andre tilbud som stopper opp. \\ \\
\emph{-Ser dere fysisk forbedring? Etter ca. hvor lang tid?}\\
Ja. Men det kommer veldig an på utgangspunktet til pasienten. Nå har vi en veldig sprek gruppe. Nå har vi drevet på i 10 uker og de siste 3-4 gangene har vi sett forbedring. \\ \\
\emph{-Oppfordrer dere de til å trene hjemme mellom hver treningstime? For eksempel ved å gi de et tilpasset program?}\\
Nei. Det er flere av deltagerene som allerede er med på fler aktiviteter utenom, som å gå tur, gå på ski osv, og vi oppfordrer dem ikke til å gjøre noe mer. Med en-til-en-pasientene er det noe helt annet. Da oppfordrer vi til trening hjemme. \\ \\
\emph{-Hvordan legger dere opp treningen når dere trener i disse gruppene du snakket om?} \\
Vi har fokus på bevegelighet. Først få opp puls. Vi lager en hinderløype der det også er balanseøvelser. Så har vi en styrkedel og uttøying til slutt. Vi har et sett av 16 øvelser, hvor vi bruker 10 hver gang. Jeg og kollegaen min bytter på å ha timen. Noen ganger tar for eksempel han oppvarmingen og så kommer jeg å tar balansedelen for eksempel. \\ \\
\emph{-Hva tror du er den største fordelen for deltageren?}\\
De fleste uttrykker at de liker treningen og at de syns det er godt å få beveget på seg på denne måten. Og så liker de det sosiale ved det. Vi har lagt inn en liten sosial del etter timen, hvor vi serverer kaffe og banan. Da tar også vi oss tid til å sette oss ned med dem og er sosiale med dem. \\ \\
\emph{-Hvor mye koster det å delta på en slik time?}\\
60 kr. Det er ganske billig i forhold til lignende tilbud. \\ \\
\emph{-Hvordan får dere høre om nye behandlingsmetoder, hjelpemidler, verktøy osv?} \\
Vi har jo noen forhandlere, f.eks. AlfaCare som er ganske idrettsrettet. Vi bruker for det meste treningsutstyr, vi har veldig lite teknisk. Dette er for det meste fordi det er så forferdelig dyrt. Sånt har man ikke råd til når kommunen ikke støtter oss med noe. Leverandørene vi allerede har, forteller om produkter. Jeg jobber også med forskning, og har i den forbindelse lest om dette, og jeg har vært på konferanser. Gjennom slike ting får man også høre en del om ting som finnes. Vi får innimellom nyhetsbrev fra leverandørene våre. Vi får “Fysioterapauten”, også får vi et eget magasin for privatpraktiserende, “Fysioterapi”, som er litt mer teknisk. \\ \\
\emph{-Hva skal til for at dere skal kjøpe/ta i bruk nye produkter/hjelpemidler/verktøy?}\\
Vi må få vite at det fungerer, og det er viktig at det fungerer inn i vår praksis. Vi kan ikke kjøpe noe som ikke fungerer. Det må i såfall være at vi kan tiltrekke oss nye eller flere kunder ved det.\\ \\
\emph{-Hva vil det bety for dere av troverdighet for produktet at det ligger et EU-finansiert prosjekt bak? } \\
Vi hadde nok likevel krevd at det hadde vært utprøvd. Det må vises at det fungerer. Det kunne f.eks vært mulig at vi testet ut produktet, mot penger selvfølgelig. Som sagt, det må være dokumentert at produktet fungerer for at vi skal bruke penger på det. \\ \\
Man forventer gjerne noe annet enn spill som behandlingsbetode når man går til en fysioterapaut. Jeg ser for meg at det kunne vært spennende å brukt et slikt spill som en økt i en gruppetime, f.eks balansetrening, oppvarming eller kondisjon. \\ \\
\emph{-Hva tenker du om at eldre kan ta i bruk dette spillet hjemme?}\\
Jeg tror det er mye mer reelt om 20-30 år når våre foreldre blir gamle. Det skal mye til for at eldre i dag skal kjøpe dette og ha det hjemme. \\ \\
\emph{-Hvordan foregår en kjøpsprosess hos dere?}\\
Vi tar alle beslutningene. \\ \\
\emph{-Kjøper dere verktøy for å eie selv eller pleier dere å leie?}\\
Hittil har vi kjøpt alt. Vi har bygd oss en pengebase, og vi har en ønskeliste. Men vi er forsiktige med å leie, fordi vi fortsatt er ustabile. \\ \\
\emph{-Hender det at dere kjøper inn produkter for å leie dem videre til pasienter?}\\
Vi kan kjøpe inn produkter for å selge videre til pasienter. Strikker er et eksempel på dette. De eldre kommer til oss og lurer på “hvor får jeg kjøpt dette?” og da sier vi at vi kan kjøpe det inn for dem. Vi selger det da til forbrukspris (leverandørpris? innkjøpspris?)\\ \\
\emph{-Finnes det noe regler rundt dette med videresalg?}\\
Det er ikke regler for videresalg på treningsutstyr/behandlingsutstyr. Behandlingsutstyr har fritak for moms. \\ \\
\emph{-Hvordan starter dere å ta i bruk nye produkter? Hender det at dere får introduksjonshjelp?}\\
Det kommer veldig ann på. Kanskje spesielt ved tekniske produkter. Kontakten med AlfaCare er god, der er det på det nivået at man kan sende en facebookmelding og si at vi har problemer med f.eks en skrue på en sykkel, så kommer de og fikser dette. \\ \\
\emph{-Er det viktig for dere å kunne komme med tilbakemeldinger på et produkt? Eller ville dere bare forkastet et produkt om det ikke var godt nok?}\\
Det kommer veldig ann på produktet. En sykkel er en sykkel. Kanskje mer viktig på tekniske ting. For eksempel på journalsystemet og websiden spør man gjerne om forbedringer. Det å komme med tilbakemeldinger på tekniske ting kan jo være viktig for behandling senere. Kan kanskje komme med tilbakemeldinger på hva som kunne gjort produktet mer tilpasset oss. Vi vil jo ha et produkt som er mest mulig optimert i forhold til vårt arbeid. \\ \\
\emph{Hva slags verdi tror du er bevegelsesstyrkende videospill kunne gitt til en bruker?}\\
Må ha den rette gruppa. Kan for eksempel passe bra for “sportsidiotene”, når konkurranse-innstinktet slår til. Noen liker dans, noen ski osv. Bør tilpasses til interesser.\\ \\
Jeg skjønner veldig godt balanse -og kondisjonsdelen av dette spillet, men jeg tror ikke det ville funket så bra for styrkedelen.\\ \\ 
\emph{-Se for deg at dette spillet allerede har fått mange gode tilbakemeldinger og at det har påvist positiv effekt. Det har allerede blitt tatt i bruk, og du ønsker å få det inn til din praksis. Vi vil beskrive noen fiktive forslag til pakketilbud vi kan se for oss. Det er også satt på en grovt estimert pris. Dette er kun for å se og høre reaksjonen din, og kanskje få tilbakemelding på hva du mener om forslagene.}\\ \\
\emph{Case:\\
Pakke og “klare seg selv”: 2300 for Xbox Kinect + spill: 4 000 kr.
Pakke med installasjon, introduksjon og opplæring:  8 000 kr.
Pakke med lisensavtale, man får pakken gratis og betaler kun for bruk: 10 kr per time \* 4 timer per dag \* 5 dager \* 47 uker = 9 400 kr.}\\ \\
\emph{Ser du noen fordeler/ulemper ved hver av disse forslagene? Er det en du kunne tenkt deg mer enn noen andre?}\\ \\
Dette var billig! Slik vi har det nå så ville nok en lisensavtale vært bra. På et slikt produkt ville jeg ikke bare fått det i posten for så å måtte sette det opp og finne ut av det selv. Det har jeg ikke tid til. Så det å ha en person som kommer å installerer spillet og viser innstillingene og hvordan det kan bruker, det er nødvendig. \\ \\
Leasing på trykkbølge er opp i mote 11 000 i måneden, og hvis du skal kjøpe ligger det mellom 100 000 - 200 000.

\newpage
\section*{Apendix D - Mailsamtale 23.11.12}
\label{D}

\emph{Intervjuobjekt:}\\
Navn: Sylvi Sand \\
Arbeid: Enhet for fysioterapitjenester, fagleder voksne/eldre \\ \\
\emph{-Hvordan får dere høre om nye behandlingsmetoder, hjelpemidler, verktøy osv.? Blir man som fysioterapeut oppfordret til å følge med på hva som er nytt eller tar man initiativ til dette selv? Er det noen faste eventer, som for eksempel messer eller foredrag, som dere deltar på i løpet av et år?} \\
Alle fysioterapeuter har plikt til å holde seg faglig à jour. Dette løses på mange forskjellige måter. Behandlingsmetoder blir kjent gjennom interne nettverksmøter der kompetanse på satsingsområdene våre, artikler og fagblader , kurs, konferanser. \\ \\ 
Hjelpemidler som er godkjent gjennom NAV systemet får vi kjennskap til gjennom Hjelpemiddelsentralen og gjennom firma som tilbyr demonstrasjoner. \\ \\
\emph{-I tillegg har vi et spørsmål som går litt mer direkte på økonomi. Vi forsto det som at dersom man benytter seg av den kommunale fysioterapitjenesten så er dette gratis. Men har du noe anslag på hva kostnaden for en fysioterapitime egentlig er? Grunnen til at vi lurer på dette er at vi ønsker å gjøre dette treningsspillet til et godt økonomisk alternativ. Det skal ikke bare et godt treningstilbud for eldre og arbeidsavlastende for fysioterapeuten, men må også være billigere enn hva som finnes i dag.} \\
 Her oppfordrer jeg dere til  å se nærmere på all forskningen rundt exergaming, som viser at spill er nyttig supplement (for å gjøre behandlingen enda mer effektiv) – ikke i stedet for behandling hos fysioterapeuten.\\ \\ 
Man kan ikke sette = mellom behandling og trening
Derfor blir en  timespris for fysioterapi ikke riktig  utgangspunkt for det dere er ute etter. Hvis du må ha en timespris, oppfordrer jeg heller til å ta utgangspunkt i hva en personlig trener koster. \\ \\
\emph{-Et eksempel er dersom man ved bruk av dette spillet kan få muligheten til å ha tre pasienter inne til behandling på en time, i stedet for å kun ha en pasient per time. Dette vil gi mindre kostnader per time per pasient, og siden kommunen betaler for behandlingen så vil en slik ordning føre til reduksjon av kostnader for kommunen. Siden det er kommunen som også står for eventuell betaling av nye produkter ønsker vi å få fram at bruk av et slikt spill kan være hensiktsmessig økonomisk. Vi vil også prøve å sette en passende pris på spillet, og derfor må vi se på hva slags reduksjoner av kostnader dette spillet kan gi.} \\ 
For å selge noe til den kommunale fysioterapitjenesten som er viktigere enn alt annet:Metoden må være kunnskapsbasert. Det må være dokumentert at dette virker. Brukervennlighet er et annet viktig poeng. Så kan man snakke om økonomi \\ \\
\emph{-Er vanlig å ta betalt dersom man blir med på et pilotprosjekt? Hva slags avtaler pleier man å gjøre dersom man går inn i et pilotprosjekt?}\\ 
Hvis dere eller Cyberlab ønsker å bruke noen av våre ansatte i arbeidet med en pilot, må det rettes en henvendelse til meg som er fagleder. I forhold til prosjekter er det vanlig at forskerne/prosjektene kjøper fri ansatte i deler av stillingen (eller på timebasis) for slikt  arbeidet. Dette diskuterer/forhandler man om med ledelsen i enheten vår.

\newpage
\section*{Appendix E - Costs Calculations}
\label{E}

\textbf{Calculations of the cost of the workload assigned the different tasks} \\ \\

\begin{figure}[h]
\begin{center}
\includegraphics[scale=0.8]{calcFTE}
\label{fig:fte}
\end{center}
\end{figure}
\bigskip
\bigskip
\bigskip
\bigskip
\bigskip
\bigskip
\bigskip
\bigskip
\bigskip
\bigskip
\bigskip
\bigskip
\bigskip
\bigskip
\bigskip
\textbf{The cost of FTE=1 for the different employees} \\ \\

\begin{figure}
\begin{center}
\includegraphics[scale=0.8]{appendixlonn}
\label{fig:employee}
\end{center}
\end{figure}


\newpage
\section*{Appendix F - Revenue Calculations}
\label{F}

The tables provided shows calculated revenue and profit. The 400 estimated units will be sold during a five year period, distributed by an s-curve. Present value (PV) has been calculated with a discount rate of 4 percent. We present income per year and total revenue for both the fixed price model and the usage fee model. We provide different price examples and solutions for both models. Total profit is calculated with an approximate total cost of 3 240 000 NOK, showed in Figure \ref{fig:profitRounded}. The same rounded total cost is used in all the calculations in the financial analysis. Figure \ref{fig:profitAll} shows the yearly and total profit for all examples with the actual cost of  3 238 775 NOK. \\ \\
\bigskip
\bigskip
\bigskip
\bigskip
\bigskip
\bigskip
\bigskip
\bigskip
\bigskip
\bigskip
\bigskip
\bigskip

\begin{figure}
\begin{center}
\includegraphics[scale=0.8]{revenuepvappendixfixed}
\caption{Revenue stream for three fixed price examples}
\label{fig:revenueFixed}
\end{center}
\end{figure}

\begin{figure}
\begin{center}
\includegraphics[scale=0.8]{revenuepvappendixusage}
\caption{Revenue stream for three usage fee examples}
\label{fig:revenueUsage}
\end{center}
\end{figure}

\begin{figure}
\begin{center}
\includegraphics[scale=0.8]{revenuepvappendixassumptations}
\caption{Discount rate of 4 percent and total cost of 3 240 000 NOK is used}
\label{fig:revenuePVassump}
\end{center}
\end{figure}

\begin{figure}
\begin{center}
\includegraphics[scale=0.8]{revenuepvappendixprofitrounded}
\caption{Total profit with the approximate total cost}
\label{fig:profitRounded}
\end{center}
\end{figure}

\begin{figure}
\begin{center}
\includegraphics[scale=0.8]{revenuepvappendixprofit}
\caption{Yearly and total profit for each example}
\label{fig:profitAll}
\end{center}
\end{figure}

\newpage
\section*{Appendix G - Beregning av potenstielt marked for Cyberlab}
\label{G}

Vi har beregnet et potensielt marked for Cyberlab. Dette har vi gjort ved å se på fire kommuner og antall kommunale klinikker og private klinikker med driftstilskudd i hver av dem. Vi har for hver kommune regnet ut antall innbyggere per klinikk, for så å ta et gjennomsnitt av dette. Dette gjennomsnittet ble sett på sammen med Norges totale befolkning, noe som ga oss et estimat på 1 137 klinikker. Vi har rundet dette opp til 1 200 klinikker.

\begin{figure}
\begin{center}
\includegraphics[scale=0.8]{antallklinikker}
\label{fig:clinics}
\end{center}
\end{figure}

\begin{figure}
\begin{center}
\includegraphics[scale=0.8]{kilder}
\label{fig:sources}
\end{center}
\end{figure}




\end{document}
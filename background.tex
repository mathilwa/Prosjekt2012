\chapter{A Background Study}
To be able to understand the overall picture of the game Cyberlab is going to develop; how the game can and should be developed; what kind of technology should be used: how it can work as a substitute to an exercise program or an additional offering and how this can work in an existing physiotherapist service, we have to describe each of the entities alone. In this chapter we will describe how fall prevention is done today, what video games, and in general exergames, are, the different type of exergames relevant for Cyberlab's game and at last we will bring fall prevention and rehabilitation together with exercise games by looking at some previous research. 

\section{Fall Prevention and Rehabilitation Today}
Physiotherapy is a science related to the medical field and central to this subject is body and movement. The theoretical basis for this subject is grounded in knowledge about science and society, and the recognition that different factors contribute to the maintenance of health. In addition to injuries and diseases, quality of life, experiences as well as social and cultural factors leads to pain and disability. This understanding along with practical and clinical knowledge forms the basis for evaluation of how injuries and pain can be treated and prevented. Physiotherapy can be described as examining, treating and preventive. Included are manual techniques, exercises and possible use of technical methods. The subject of physiotherapy is performed by a physiotherapist. Physiotherapists are traditionally working in municipalities, hospitals and in private institutions, and they are working with both individual treatment and treatment in groups. The goal for physiotherapists is to make their patients’ daily activities easier to manage \cite{physiotherapy1}\cite{physiotherapy2}.\\ \\
Elderly people consult physiotherapists' in terms of rehabilitation after surgery, after a fall, stroke or other injuries, or when they feel health problems make it hard to perform everyday tasks. Physical therapy usually includes exercise with focus on increasing the patients’ flexibility, endurance and strength. Physiotherapists sets up customized training for each patient according to what kind of needs they have. Unfortunately, the time a patient spends with the physiotherapists per week is not sufficient in order to become stronger or to recover from injuries. Therefore, one sees the need for patients to practice outside of these hours at the physiotherapists. Physiotherapists may give their patients exercises they can practice at home. However, not everyone is motivated to exercise on their own and many skip the weekly exercise that is scheduled for them. \cite{physiotherapy2} 

\section{Video games in general}
"Video games are electronic, interactive games known for their vibrant colors, sound effects, and complex graphics" \cite{videogamedef}. Characters or objects are controlled by hand held game controllers, or by pure body movement captured by sensors or motion controllers. Since the first computer game was developed in 1952 there has been a tremendous evolution in the computer and video game market. Todays market consist of an endless amount of various computer games, video games and video game consoles, and this type of technology are widely used all over the world. It has been developed a game for almost every need and interest, and video games are used for various different purposes, like education and learning, exercising or just pure entertainment. In this section we will describe video game history and gaming statistics. \\ \\
The first graphical computer game was created by A.S. Douglas in 1952. This single-player "game" was based on a version of Tic-Tac-Toe and the title of the game was "OXO". "OXO" was designed for academic purposes.  Douglas wrote a PhD degree on Human-Computer interaction, and used feedback from the electronic "OXO" in his work \cite{abouthiginbotham}. Ralph Baer, a German-born television engineer, designed in 1967 the first video game console for use on standard television. "Chase" was the name of the game, and here two players were connected to a television where they controlled two squares which they used to chase each other \cite{videogameHistory}. Various features were added to this idea, and this ended up in 12 games known as the Brown Box. Baer introduced his idea to Magnavox, and in 1972 the first commercial video game called Magnavox Odyssey was produced. TV dealers did not see the potential in the Odyssey, so it didn't get very popular, and the market experienced a video-game crash. In 1985 the first Nintendo Entertainment System were released. Retailers were skeptical to marketing a new console soon after the video-game crash, but the games Nintendo introduced got popular, and soon Nintendo broke sale records and become the best-selling console in video-game history \cite{consoleHistory}. \\ \\
The Nintendo Wii is one of many gaming consoles, and it is worldwide very popular. It has sold over 30 million units in the US and in Japan there has been sold almost 10 million. These numbers combined with the international market gives a total sale of Nintendo Wii of 65.32 million units. However, the bestselling console ever is the Sony PlayStation 2, with over 138 million units sold. The bestselling video game series is the Mario franchise, with a sales number of over 225 million games \cite{statistics2012}. 
\begin{figure}[h!]
\label{fig:ConsoleWarsAll}
\begin{center}
\includegraphics[scale=0.5]{consolewarsall}
\caption[Console war]{Console war 2012 \cite{statistics2012}}
\end{center}
\end{figure}
DFC Intelligence is a marked research and consulting firm which focus on interactive entertainment and game markets. The global market for video games experienced revenue of 67 billion dollars in 2012, and in DFC Intelligence's new reports they forecast that the global video game market is expected to reach 82 billion dollars in 2017. This number includes revenue from console hardware and software, PC games and games for mobile devices \cite{videogameforcast} \cite{aboutdfcint}. \\ \\
In the last 30 years there has been a great evolution in video games. Several popular games and consoles, like e.g. Gameboy and PlayStation, has been developed, and today there exists a huge amount of various video games and consoles. Video games have become widespread entertainment, and in the US 65 percent of all households play video games. The majority of these gamers are players in the age of 18-49, and today the average age for a US gamer is 32 years old. There are a significant difference between the gamer age distribution in the US and Norway, see Figure \ref{fig:GamersUSNorway}. In Norway, a greater share of the population under the age of 19 is entertained by video games, and the gamer majority is shared between the two younger age groups. In 2010 the average gamer played video games 8 hours during a week, and in 2012 this is more than doubled, when gamers today spends 18 hours a week playing video games.  Video games are used for various different purposes, like education and learning, exercising or just pure entertainment \cite{statistics2010} \cite{statistics2012}. \\ \\
A maybe surprising fact about the gaming statistics in the US is that as much as 2 out of 5 gamers are women, and that there are more gamers over 50 than there are gamers under 18 \cite{statistics2012}. "Norsk mediebarometer", a report containing statistics around the topic of media use in Norway, show that 17 percent of the Norwegian population plays computer or video games on an average day in 2011. This includes not only children and teenagers; also a great part of elderly has started to use computer or video games. 8 percent of the population in the age 45 - 79 years use this kind of technologies on an average day, see Figure \ref{fig:GameStatisticsNorway}, where females are the most active gamers. The US has a higher number of gamers over 50, but the statistics shows that use of computer and video game in Norway has increased from 5 percent in 2010 and this measurement are the highest share ever recorded \cite{ssb2010} \cite{ssb2011}. (SKAL jeg si noe om at ingen i aldersgruppen 67-79 bruker kun TV-spill?) 
\begin{figure}
\label{fig:GamersUSNorway}
\begin{center}
\includegraphics[scale=0.5]{gamersusnorway}
\caption[Gamer age distribution]{Gamer age distribution in the US in 2012 [modified from \cite{statistics2012}] and in Norway [calculated from \cite{ssb2011} \cite{folketall2012}]}
\end{center}
\end{figure}

\begin{figure}
\label{fig:GameStatisticsNorway}
\begin{center}
\includegraphics[scale=0.9]{gamestatisticsnorway}
\caption[Use of computer or video games, Norway, 2011]{Percentage of the Norwegian population that use computer or video games on an average day in 2011, sorted by age" \cite{ssb2011}}
\end{center}
\end{figure}
       
\section{Exercise Games}
When video games are brought together with exercise, we have what is called Exercise games, or "Exergames". Technology for tracking body movement, like motion sensors and remote controls, require the players to move which stimulates exercise. The combination of movement, fun and social interaction provides exergaming great potential for new business opportunities for the entertainment, recreation and healthcare sectors \cite{gamingforhealth}. \\ \\
Today there exist numerous types of games and technologies, where Nintendo Wii, Dance Dance Revolution, PlayStation Eye Toy and Xbox Kinect are some of the more familiar technologies. This genre of games has become very popular, and due to the growing interests one has seen it as relevant to study the use in regard of health and education. The technology these games provide can in an interesting way help provide health-related information to specified target groups \cite{gamingforhealth}. In the past years exergames research has increased dramatically, and indications show that it will continue to do so \cite{chamberlin2008exergames}. Research shows tremendous promise in academic, social and physical progress of youth using exergames. The health sector is now more focused on prevention of illness instead of treatment,  where this type of research can contribute to use of exergames in health care \cite{gamingforhealth}. Games like Wii Sports and Dance Dance Revolution were designed to encourage physical activity, but many currently available exergames were not designed for this purpose. The popularity of exergames and the increasing customer appeal will improve design principles and physical requirements \cite{chamberlin2008exergames}. \\ \\
We will now describe some of the different video game consoles that can be used for exergaming.
\subsection{Dance Dance Revolution}
Dance, Dance Revolution (DDR) is a series of video games created by Konami Corporation’s Bemani music games division. DDR is a rhythmic dance simulation game and was first released as an arcade game in 1998. In few years it became very popular, and the game has had its appearance on several game console systems like Sony PlayStation, Nintendo 64, Microsoft Xbox and Nintendo GameCube \cite{bogost2005rhetoric}. DDR uses a touch-sensitive dance pad with sensors to register movements, where one shall press the right sensors in proper time with electronic dance music. Arrows on-screen gives direction on how and when to move around. The DDR games have varying difficulty, requiring different levels of physical activity. GetUpMove.com is an information website about the use of PlayStation Dance Dance Revolution as a weight loss tool. This site was launched in 2004, and one of the highlighted stories was about a young woman who lost about 95 pounds by using DDR as an exercise tool. This and similar stories got widespread exposure, and consumers started to buy DDR solely for the purpose of exercise \cite{bogost2005rhetoric}. In 2003, 5 years after the first release, Konami announce that DDR has reached a total sale of 6,5 million units worldwide \cite{gamespot}. 8 years later, in 2011, the number of sold units had reached over 13 million, which is  about 1 million units sold every year since the first release in 1998 \cite{gaygamer}. 

\subsection{PlayStation EyeToy}
In the early 2000s the PlayStation 2 EyeToy was released by Sony Inc. It was the first in this category of games to introduce a device that could translate human motions into a controller input and allow players to physically interact with virtual objects using their own body and without being connected to wires. \cite{eyetoy}. Human body movements are translated real-time into the controller input by a USB camera  and can also map the player’s face onto in-game characters. Eye Toy is easy to set up and its applications offer a lot of different environment and can be played by one or more players \cite{eyetoyrehab}.

\subsection{PlayStation Move}
PlayStation 3 Move was released in September 2010. The PlayStation Move’s interface consists of the Move Eye, a RGB camera with directive microphones, and the Motion Controller, a wand with an illuminating sphere attached to it. The camera can detect the sphere and determine where the wand is, which allows the players to interact with the PlayStation 3 through motion and position. The sphere attached to the wand helps the camera to determine the distance from the wand to the camera and to track the controllers position in three dimensions. The wand is equipped with a three-axis accelerometer and a three-axis gyro sensor which are used to track rotation in overall motion and can also be used to detect if the wand is out of range (i.e. hidden behind the player back). It also consists of a geomagnetic sensor used to calibrate the wand’s orientation against the Earth’s magnetic field, which makes it possible to recognize the wand’s position accurately.  (hmmm... hvordan kan denne setningen skrives på en annen måte?) \cite{comparison}. Up to four wands are supported at one time, which makes it possible for four players to play together. The color of the sphere can be changed to any color and is usually used to show what player is active and to give visual feedback \cite{ppmove}. The SDK is not made public, so its difficult for a third-party to make original applications \cite{comparison}. 

\subsection{Nintendo Wii}
Nintendo Wii was released in 2006 as the first motion sensor game. Only one year and 20 million units sold later, it became the market leader of that times generation of consoles. It consists of a Wii remote, which is the primary controller and a secondary controller called Nunchuk. The Nunchuk is connected at the bottom of the Wii remote control \cite{hackingwii}. The Wii remote contains 12 buttons, a 3-axis accelerometer, a high-resolution highspeed IR camera, a speaker, a vibration motor, and wireless Bluetooth connectivity.
Each Wii remote has a IR camera sensor on its tip. The camera ship can track up to four simultaneous IR light with high resolution and high speed. The accelerometer within the remote control provides the Wi remote’s motion-sensing capability.The Wii remote has a total of 12 buttons and the buttons are arranged symmetric so that both hands can be utilized. A vibrator motor, LED lights and a small speaker are used for different kinds of user feedback, like varying light strength and sound-effects. The four LED lights are also used to indicate the different players ID. Communication is sent over the wireless Bluetooth connections, which enables up to four controllers to be connected at the same time.  The users of Nintendo Wii can make their own personal profile, called Mii, where the data of the player will be directly connected up on the remote used. To operate, the remote needs two AA batteries \cite{hackingwii} \cite{whatiswii}. By December 2010 over 75 million Wii consoles were sold.  To complete the original system with improved accuracy and response time, Nintendo made an enhanced version, Wii Motion Plus, which was released in November 2009 \cite{consoles}.There are several SDKs for Nintendo Wii open, which makes it possible for a third party to develop applications which utilize the controller \cite{comparison}. 

\subsection{Wii Balance Board}
The Wii Balance board is an add-on accessory for the Wii Fit, which is a video game created by Nintendo to work with the Wii platform. Just like the Wii Remote controller, the balance board can read your body movements and give them back on the screen as you are playing \cite{whatiswiifit}. The balance board contains multiple pressure sensors which track body movements. The board has a area of 551 mm {*} 316 mm. A third party can also build applications for the balance board using the SDK WiimoteLib \cite{comparison}. It has been shown that game-based balance programs like Wii Balance Board compared to traditional training is easier, more motivating and more enjoyable \cite{taylor2011activity}.

\subsection{Microsoft Kinect for Xbox 360 and Windows}
Microsoft Kinect was released in 2010 and became quickly extremely popular. Only 25 days after its release it had sold 2.5 million units and by January 2012 Xbox 360 had sold over 66 million consoles and more than 18 million Kinect motion sensors \cite{consoles},  \cite{kinectsold}. Microsoft Kinect is a flexible low-cost motion sensor device for the Xbox 360 game console  and for Windows PCs that can track human motion. It is a webcam-based add-on peripheral for the console, which enables the user to play and interact with the game without physically holding a sensor device. Instead the player can interact with the game console through a natural user interface using gestures and voice commands (\cite{kinect}, siste setning kanskje litt avskrift.. se på senere). The device gives full-body 3D motion capture capabilities and gesture recognition by help of a RGB camera and a depth sensor \cite{kinect}. One advantage with Kinect is that it has an interface that senses players various motions and it also senses other objects in the field, which makes a natural environment where the players can interact with virtual objects in the real world. \cite{comparison}. It also exists a Kinect sensor for Windows. The sensor is designed to operate on computer running Windows 7, Windows 8, Windows Embedded Standard 7, and Windows Embedded POSReady 7. All the users need is the Kinect sensor, a computer and a Kinect for Windows application. Kinect for Windows SDK was released in June 2011 and enables developers to build Kinect applications with C++, C\# or Visual Basic using Microsoft Visual Studio 2010. This enables any third party to develop Kinect for Windows applications. \cite{kinectwindows}.

\section{Using Exergames for Fall Prevention and Rehabilitation: A background Study}
Games are becoming popular as a tool for exercise and rehabilitation. There have been a lot of different studies done on how some of the previous described types of video game consoles can be used for this. Most of the studies have been done on adolescents, but there also exists some research done directly on elderly. In this section we will review some of the interesting findings we did.\\ \\
Taylor et al. \cite{taylor2011activity} did a study where they searched through already done studies to draw a picture on how games can be used for exercise and rehabilitation. They reviewed some of the interesting findings they did in their paper.  From the studies they reviewed they found a trend; the  Energy Expenditure (EE) while playing Wii was greater than when doing sedentary activities, but not greater than brisk walking. This suggests that playing Wii sports could not replace real sports activities. Playing DDR on the other hand, maximum heart rate and oxygen consumption were greater compared with Wii Tennis, suggesting that DDR can substitute physical activity.
In their research they also found a study on what attitudes people have against DDR to encourage exercise. 40 postmenstrual women, aged 45-75 years old were asked. The overall attitude was positive; The game was fun and it gave a potential to improve coordination. However, they also expressed a concern about a long learning process. It is also found that playing against a human gave greater arousal ratings and physiological responses to gaming than when playing against a computer, which benefit the enjoyment, suggesting that this can be beneficial for older people. 
From their study they can conclude that computer-based rehabilitation is not a new phenomenon and that one of the main reasons for this is that games have the ability to increase motivation and produce distraction from daily, boring and painful treatments. Wii is seen as an attractive game for rehabilitation, both at home or in institutions. Wii is actually already in use within the National Health Service in UK and is commonly used for the elderly and patients with pathologies. \cite{taylor2011activity} \\ \\
Another study they found was that non-disabled  elderly (70 +/- 5,7 years) was positive to the EyeToy; they enjoyed it and found it easy to use. For patients with stroke it appeared to be less suitable, which could be even worse if they had to hold on to a controller. This suggests that EyeToy is more suitable for patients with stroke than for example Wii.  (litt usikker om vi bør ta dette fra den orginal kilden eller ikke)
Even though these type of games are initially meant as entertainment systems, there are a number of studies that have used the hardware and developed software to turn for example the Wii into a useful rehabilitation tool. The importance of these games are entertainment that motivates for actual sports. This is very important in for example rehabilitation. \cite{taylor2011activity} \\ \\
Staiano and Calvert write about how exergames are more and more used in the health sector. Gaming consoles are already integrated into equipments at gyms and health clubs. An example is Concept 2’s rowing machine. Here the people exercising are motivated through competition and through virtual trainers who monitor their progress and encouraging them to proceed to the next level. Also some schools are starting to integrate these games into their curriculum. In all of West Virginia’s 765 public schools they have integrated DDR in their physical education. This has proven to be very effective and popular and some students lost 5-10 pounds after playing DDR daily. \cite{staiano2011exergames}. \\ \\
Williams et al. did a study to see if exergames, more specifically Nintendo WiiFit, was an applicable type of exercise to reduce the falling statistics of community-dwelling people over 70 years. A group who attended WiiFit exercise sessions was compared with a group who went to a local falls group. 77 percent of the participants said that if the exercise programme was more available, people like themself would use it. 92 percent of the participants expressed that they wanted to exercise with the WiiFit in the future, while 61 percent would choose to exercise with the WiiFit rather than attend a falls group. An improvement in Berg Balance Scale (BBS)\footnote{Berg Balance Scale (BBS), a performance based measure using 14 activities of daily living (range 0-56)\cite{excell}} at 4 weeks was seen in the group that played WiiFit, meaning that there is a potential to improve balance in this population. Despite this, there was no change in The Falls Efficacy Scale - International (FES-I)\footnote{The Falls Efficacy Scale - International (FES-I) (http://ageing.oxfordjournals.org/content/34/6/614.short)} at 4 weeks. The qualitative data for the group that played WiiFit showed improved confidence for the participants. The conclusion of the study is that WiiFit is acceptable in  older people with a history of falls and that it has the potential to improve balance and confidence. Further work has to be done to find and develop an acceptable exercise programme with the potential to improve balance in older individuals. \cite{excell}\\ \\
Chang et al. did a study where they prototyped a Kinect game that was designed to help motivate people with motor disabilities to do their exercise more frequently and to improve the motor proficiency and quality of life. Because of the inconvenience of having to wear sensors in some of the other relevant technologies, Chang et al. chose to use Kinect. They developed a game, called “Kinerehab”, that was meant to assist therapists in rehabilitating students in public school settings. To detect the students’ movements Kinerehab uses image processing technology of Kinect. To engage and motivate the student for physical rehabilitation, the system is made with an interactive interface that has both audio and video feedback. For making it easy for therapists to review the progress of each students quickly, the system also includes details of students rehabilitation conditions which is automatically recorded in the system. Two students, a 16 year old girl diagnosed with having acquired muscle atrophy and insufficient muscle endurance, and a 17 year old boy diagnosed with cerebral palsy, were chosen to participate in the study. The girl used a wheelchair and could only stand with assistance. The study included two phases: a baseline phase  where no assistive technology was applied, and the intervention phase where the Kinerehab was used. Both phases were done twice, beginning with the baseline phase, continuing with the intervention phase and so on. In both phases the same exercises were done. The result showed that both participants increased the number of correct movements significantly in the intervention phase. On average the number of correct movements was 49 in the first baseline phase (5 sessions), while 170 in the first intervention phase (11 sessions). Both students indicated that the game motivated them to do the exercises and that they wanted to continue using it. The therapists said it would increase their workload a lot. This suggests that Kinect can be a viable rehabilitation tool, but further work, where more people with disabilities participate, should be done. \cite{kinect} \\ \\

Et konkluderende avsnitt her...






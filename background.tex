\chapter{A Background Study}
\ac{vm}
\section{Rehabilitation}

\section{Exercise Games}
Digital games where exercise and game play are brought together are called exergames. Technology for tracking body movement, like motion sensors and remote controls, require the players to move which stimulates exercise. The combination of movement, fun and social interaction provides exergaming great potential for new business opportunities for the entertainment, recreation and healthcare sectors \cite{gamingforhealth}. \\ \\
Today there exist numerous types of games and technologies, where Nintendo Wii, Dance Dance Revolution, Playstation Eye Toy and Xbox Kinect are some of the more familiar technologies. This genres of games has become very popular, and due to the growing interests one has seen it as relevant to study the use in regard of health and education. The technology these games provide can in an interesting way help provide health-related information to specified target groups \cite{gamingforhealth}. In the past years exergames research has increased dramatically, and indications show that it will continue to do so \cite{chamberlin2008exergames}. Some research done on exergaming shows tremendous promise in academic, social and physical progress of youth using exergames. The health sector is now more focused on prevention of illness in stead of treatment,  where this type of research can contribute to use of exergames in health care \cite{gamingforhealth}. Games like Wii Sports and Dance, Dance Revolution were designed to encourage physical activity, but many currently available exergames were not designed for this purpose. The popularity of exergames and the increasing customer appeal will improve design principles and physical requirements \cite{chamberlin2008exergames}. 
\subsection{Dance Dance Revolution}
Dance, Dance Revolution (DDR) is a series of video games created by Konami Corporation’s Bemani music games division. DDR is a rhythmic dance simulation game and was first released as an arcade game in 1998. In few years it became very popular, and the game has had its appearance on several game console systems like Sony Playstation, Nintendo 64, Microsoft Xbox and Nintendo GameCube \cite{bogost2005rhetoric}. \\ \\ 
DDR uses a touch-sensitive dance pad with sensors to register movements, where one shall press the right sensors in proper time with electronic dance music. Arrows on-screen gives direction on how and when to move around. The DDR games has varying difficulty, requiring different levels of physical activity. GetUpMove.com is an information website about the use of Playstation Dance Dance Revoluition as a weight loss tool. This site was launched in 2004, and one of the highlighted stories was about a young woman who lost about 95 pounds by using DDR as an exercise tool. This and similar stories got widespread exposure, and consumers started to buy DDR solely for the purpose of exercise \cite{bogost2005rhetoric}. By a research done in 2006 it was clear that most articles stated that playing exergames had a positive health effect. Even though DDR falls short in meeting all the requirements stated from the American College of Sports Medicine (ACSM), studies has shown that players can meet the ACSM’s minimum requirement by playing DDR at the beginner level \cite{chamberlin2008exergames}. \\ \\ 
In 2003, 5 years after the first release, Konami announce that DDR has reached a total sale of 6,5 million units worldwide \cite{gamespot}. This number does not include Arcade units. 8 years later, in 2011, the number of sold units had reached over 13 million, which is  about 1 million units sold every year sinced the first release in 1998 \cite{gaygamer}. 

\subsection{PlayStation EyeToy}
In the early 2000s the PlayStation 2 EyeToy was released by Sony Inc. It was the first in this category of games to introduce a device that could translate human motions into a controller input and allow you to physically interact with virtual objects using your own body and without being connected to wires. The EyeToy is a USB camera that tracks body movements and translate them into a video game \cite{eyetoy}. Human body movements are translated real-time into the controller input by a USB camera  and can  map the player’s face onto in-game characters. Eye Toy is easy to set up and its applications offer a lot of different environment and can be played by one or more players \cite{eyetoyrehab}.

\subsection{PlayStation Move}
PlayStation 3 Move was released in September 2010. The PlayStation Move’s interface consists of the Move Eye, a RGB camera with directive microphones, and the Motion Controller, a wand with an illuminating sphere attached to it. The camera can detect the sphere and determine where the wand is, which allows the players to interact with the PlayStation 3 through motion and position. The sphere attached to the wand helps the camera to determine the distance from the wand to the camera and to track the controllers position in three dimensions. The wand is equipped with a three-axis accelerometer and a three-axis gyro sensor which are used to track rotation in overall motion and can also be used to detect if the wand is out of range (i.e. hidden behind the player back). It also consists of a geomagnetic sensor used to calibrate the wand’s orientation against the Earth’s magnetic field, which makes it possible to recognize the wand’s position accurately.  (hmmm... hvordan kan denne setningen skrives på en annen måte?) \cite{comparison}. Up to four wands are supported at one time, meaning that four players can play together. The color of the sphere can be changed to any color and is usually used to show what player is active and to give visual feedback \cite{ppmove}. The SDK is not made public, so its not difficult for a third-party to make original applications (Sjekke dette?) \cite{comparison}. 

\subsection{Nintendo Wii}
Nintendo Wii was released in 2006 as the first motion sensor game. Only one year and 20 million units sold later, it became the market leader of that times generation of consoles. It consists of a Wii remote, which is the primary controller and a secondary controller called Nunchuk. The Nunchuk is connected at the bottom of the Wii remote control \cite{wii} \cite{hackingwii}. The Wii remote contains 12 buttons, a 3-axis accelerometer, a high-resolution highspeed IR camera, a speaker, a vibration motor, and wireless Bluetooth connectivity.
Each Wii remote has a IR camera sensor on its tip. The camera ship can track up to four simultaneous IR light with high resolution and high speed. The accelerometer within the remote control provides the Wi remote’s motion-sensing capability.The Wii remote has a total of 12 buttons and the buttons are arranged symmetric so that both hands can be utilized. A vibrator motor, LED lights and a small speaker are used for different kinds of user feedback, like varying light strength and sound-effects. The four LED lights are also used to indicate the different players ID. Communication is sent over the wireless Bluetooth connections, which enables up to four controllers to be connected at the same time.  The users of Nintendo Wii can make their own personal profile, called Mii, where the data of the player will be directly connected up on the remote used. To operate, the remote needs two AA batteries \cite{hackingwii} \cite{whatiswii}. By December 2010 over 75 million Wii consoles were sold.  To complete the original system with improved accuracy and response time, Nintendo made an enhanced version, Wii Motion Plus, which was released in November 2009 \cite{consoles}.There are several SDKs for Nintendo Wii open, which makes it possible for a third party to develop applications which utilize the controller \cite{comparison}. 

\subsection{Wii Balance Board}
The Wii Balance board is an add-on accessory for the Wii Fit, which is a video game created by Nintendo to work with the Wii platform. Just like the Wii Remote controller, the balance board can read your body movements and give them back on the screen as you are playing \cite{whatiswiifit}. The balance board contains multiple pressure sensors which tracks the body movements. The board has a area of 551 mm {*} 316 mm. A third party can also build applications for the balance board using the SDK WiimoteLib \cite{comparison}. It has been shown that game-based balance programs like Wii Balance Board compared to traditional training is easier, more motivating and more enjoyable \cite{taylor2011activity}.

\subsection{Kinect}
Microsoft Kinect was released in 2010 and became quickly extremely popular. Only 25 days after its release it had sold 2.5 million units and by January 2012 Xbox 360 had sold over 66 million consoles and more than 18 million Kinect motion sensors \cite{consoles}, \cite{kinectsold}. (With this it became the fastest selling electronic device of all times. <-huskerikke hvor jeg har funnet denne). Microsoft Kinect is a flexible low-cost motion sensor device for the Xbox 360 game console  and for Windows PCs that can track human motion. It is a webcam-based add-on peripheral for the console, which enables the user to play and interact with the game without physically holding a sensor device. Instead the player can interact with the game console through a natural user interface using gestures and voice commands (\cite{kinect}, siste setning kanskje litt avskrift.. se på senere). The device gives full-body 3D motion capture capabilities and gesture recognition by help of a RGB camera and a depth sensor \cite{kinect}. The advantages of Kinect are that it has an interface that senses players various motions and it also senses other objects in the field, which makes a natural environment where the players can interact with virtual objects in the real world. Kinect also have some limitations. Compared to Nintendo Wii and PlayStation Move Kinect's camera has a relatively low resolution of 30 Hz \cite{comparison}. Another advantage  is that its possible for a third party to write application for Kinect. Kinect for Windows SDK was released in June 2011 and enables developers to build Kinect applications with C++, C\# or Visual Basic using Microsoft Visual Studio 2010 \cite{kinectwindows}.
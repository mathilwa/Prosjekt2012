\chapter{Information Gathering}

In this section we will describe what kind of methods we have used to make a business model for the Kinect based exercise game. 

\section{First Phase}
Research and previous work have been well examined, to find out what has already been done within this field. Based on our foundational knowledge we developed three different business model where we looked at three different customer segments, namely \emph{the end user}, \emph{physiotherapists (and the health sector in general)}, and \emph{voluntary exercise groups}. During our work on the business model we discovered many loose ends, and at one point we could discard some of our theories. We ended up with physiotherapists as our main customer segment.

\section{Qualitative Research}
The next step in the process was to do a qualitative research on the chosen customer segment and on fall problematic in elderly. Qualitative research means to get an in-depth understanding of a phenomenon. As a method we used qualitative interviews as a data gathering tool. This is the most common tool used in qualitative research.  \cite{interview2} \\ \\
We interviewed three physiotherapists working with the older patient group. Our supervisor helped us get in contact with one physiotherapist, who again put us in contact with other physiotherapists. In addition to that, we contacted come physiotherapists ourselves. We choose to look at both clinics controlled by the government and also one private clinic. The reason why we did this, is because the two entities have two quite different economic models. In the private clinic, the owners are the payers, while in the public clinics, the government is the payer.

\subsection{Interview Methods}
There are different types of qualitative interview methods \cite{interview} \cite{interview2}: \\ 
1. \emph{Structured Interview:} The main topic for the interview is decided and a complete interview guide is prepared beforehand. \\ 
2. \emph{Unstructured Interview:} This is a very flexible method where the topic is decided, but there is usually no interview guide. This allows the interviewer to improvise suitable questions during the interview. \\ 
3. \emph{Semi-structured Interview:} This is a mix between method 1 and 2. The interviewer has an interview guide with some prepared questions, but these questions serve more as guidelines, and allows the interviewer to improvise suitable questions during the interview. This is the most commonly used interview method, and is often called "qualitative interview". \cite{interview} \cite{interview2} \\ \\
We used semi-structured interview because of several reasons. First, all of our interviewees where physiotherapists and we had some specific questions about their routines directed towards the business model we were developing. But since they were all working in different clinics the questions had to be adapted towards their field of expertise, and therefore there was room for improvisation. Second, since neither of us are professionals in this field, it would be impossible for us to foresee everything that should be asked about. Last, we wanted to make the interview more like a natural conversation, without locking ourselves to specific questions. This allowed the conversation to flow more naturally, providing us with some unexpected information, that we did not think about beforehand. \\ \\
In accordance to the normal structure of an interview \cite{interview2}, we started with an introduction, telling about who we were, what our project was about and what we were there to find out. Then we followed up with some basic questions about the interviewee, like name, age, work and education. In this way, we got to know each other better before the questioning started. We had two different main topics we wanted to discuss; First, how they worked and their relationships with their patients. This was to identify if there was a need and also how the product could fit into their working situation. Second, we had some questions more directed towards the business model; How they got to know about products and how they acquired them. In each topic, we had some defined questions to guide us, but not to limit us. There was room for improvisation at all times. We also edited the interview guide between the interviews based on the experiences we gained.

\subsection{Possible Pitfalls}
When doing a qualitative interview there can always occur some unexpected problems or difficulties. We will go through some of the limitations that may have affected our interviews, based on a list of possible pitfalls we found in \cite{interview}:\\
\emph{Artificiality of the interview:} All our interviewees were strangers to us. The interviewee was asked to answer questions  and give opinions under time pressure.\\
\emph{Lack of trust:} Because we did not know the interviewees, them not trusting us could have been an issue. This means that the interviewees may have held back what they think of as sensitive information. This information may have been important information for us, and a possible holdback of this information would make the gathering incomplete. \\
\emph{Lack of time:} In one of our interview, we had a time-limit. Whether this had a positive or negative effect is hard to say. Time limit can result in an incomplete data gathering, but also lead to the opposite where the interviewee creates opinions under time pressure which can result in more data, but possibly less reliable data. In our case, we do not feel that time constraints resulted in any unreliable data. \\
\emph{Ambiguity of language:} Sometimes a meaning of a word can be ambiguous. Both the interviewer and the interviewee can misunderstand the meaning. Since our knowledge within this field was limited, it appeared some misunderstandings during the interview. The misunderstandings were discovered when the interviewee read through the interview report, and were fixed at the same time. In this way it did not affect our final report.  

\section{Interview Discussion}
In this section we will discuss some of the findings we did in our interviews. We will only take into consideration what we find important for our project. Everything is taken from the interviews conducted, in addition to our own opinions and perception of the interviews. For the interested reader, you can find the interview reports in the appendix.\\ \\
As already mentioned, two of our interviews were with physiotherapists working in government controlled clinics, from now on called public clinics. The third interviewee worked in a private clinic, which she owned with a partner. All of the interviewees had elderly people as a patient group they focused on, but not necessarily their only patient group. The private clinic offered an exercise program where elderly could meet and exercise with other elderly once a week. This is also something that is arranged by the government, called "Seniortrim". The physiotherapist who had the training group in her clinic said that the feedback they got from the patients were that it is nice to get some physical movement and that it is very social. The training group in the private clinic costs 60 NOK per session and the one in the public costs 30 NOK per session. This is two interesting offerings, where we see a potential for the exercise game to be implemented, because this is a arena where they can play more people together as well as getting assistance from a professional. \\ \\
Two of our interviewees mentioned that they wish to get the patients in for an examination earlier. Most of the patients go to the physiotherapist first when their problem gets serious. Getting them in earlier means a chance for prevention instead of rehabilitation. Everything really depends of the patients background; whether or not it is important for the patients to be able to continue working out everyday activities, or if they just accept the fact that they are getting older and that they are not able to do everything they did before. Typically, people that is used to be very active in the sense of often taking a walk, go skiing during the winter etc., will be more eager to keep a good physical health. For those that are used to get their physical activity from for example gardening, the relation to other type of physical activity may not be present. Thus, it is very important to not look at elderly as one common patient group. They are also different people with different interests.\\ \\
A normal problem is that elderly are afraid of walking outside their house, making it hard for them to attend their appointments and also other type of activities. Some people are very motivated to improve themselves while others are satisfied with how things are. Usually is one hour at the physiotherapist not enough and it is hard to get the patients to exercise at home, often because of the lack of motivation. It is common to provide the patients with an exercise program that they can perform at home, but there are some problems related to this. First, the patient may not be motivated enough to actually perform the program. Second, it is hard for the physiotherapist to give feedback to make sure that the exercises are done right. And third, there are no one there to make sure that they do not fall and hurt themselves. This can also be related to the Kinect based exercise game. One of our interviewee mentioned specifically that one of the problems of motivating the patients is that they do not identify themselves as  persons being afraid of falling, so when working with fall prevention it is important to not mention the word "fall". \\ \\
With regard to our business model we asked all our interviewees where they typically hear about new products, treatment methods and tools. Different channels were mentioned, like "Fysioterapeuten", an academic magazine for physiotherapists, conferences and from suppliers they already had an established relationship with. Every one of them pointed out how important it is that the product we are trying to sell is proved to work. It is not enough to have an ad in a magazine or newspaper, or show up demonstrating on a conference, if you do not have well documented good effect. If the product has a proved good effect, the staff meeting of the physiotherapists could be an effective arena. Again, she emphasized that if some one would come to a meeting to show a product, they had to know that it was proved good. They are not interested in hearing some random marketing. The threshold for the use of a product should be very low. It must be easy to use and must also ease the workload for the physiotherapist. If it is not better than what the physiotherapist can offer herself, she would not use it. In the public clinics the government pays for everything. Every year, a certain amount of money is given to each clinic. The clinic is then responsible for how they want to spend the money. If a physiotherapist believes in - and wishes to get a new product, the leader will be responsible to consider it and decide if they should buy it. This is different in the private clinics. Here every decisions are made by the owner(s). \\ \\
It is most common to buy and own a product. Leasing is not that common, but not irrelevant. The private clinic that we interviewed was only one year old, meaning that the economy had not had time to get very stable yet. Because of this vulnerability they was careful about buying expensive new products and they were also sceptic to subscribe on a rental agreement. We represented a fictive scenario where they could get an opportunity of paying only for the use of the game for the interviewee. She was very positive to this scenario. All of the interviewees mentioned that it might be interesting to rent, or even try for free, a product for a certain period of time to check out if it was interesting to buy.  One of our interviewees mentioned that for this specific game, it would be necessary to run a pilot project, where some clinics tried the game out for a couple of months. In this way they could test the game and document the effect. It is not enough for the potential buyer to know that this is an EU-funded project and that it sounds like a nice product. The product must be well proved over a longer period of time. \\ \\
The main goal for the exercise game is to prevent elderly from gradually loosing their physical abilities and in that way loose their confidence in doing everyday activities. It is showed that exercise only ones a week is not enough \cite{gruppetrening-trheim}, this could also one of our interviewees confirm. At the same time, some exercise is better than none, and the physiotherapist who had the group training once a week for elderly told that you could actually see improvements after six to seven sessions. For the elderly to be able to improve themselves even more, they will have to exercise more. A plan for the future should therefore be to provide the game for the elderly to use in their own home. It is not natural for us to believe that an older person would go in the store and buy this game themselves. Therefore, it was interesting for us to see if physiotherapist could work as a possible channel where the patient could get this game. 
In the public clinics it is normal to let patients borrow products, but the products are limited.  If the government believed in a product, they could buy some copies that the patients could borrow. In the private clinic it is normal to sell products, like for example special shoe soles, but they do not typically let patients barrow products. \\ \\
It is not common for physiotherapists to physically go to the store and buy a product. An already established relationship with a supplier is the most common channel where new products are bought. Often after buying a product, the supplier will contact the clinic later with improved or new products. It is quite normal to maintain a relationship with the supplier after a product is bought. This is usually done by the supplier providing the clinic with brochures with news and by offering support if something about the product was wrong. When the product is bought, it usually gets delivered at the clinic. Physiotherapists have already enough to do as it is, and have not always time to set up and learn a new product. It depends on the product whether they get an introduction or not. It would for instance often be necessary to get an introduction to technical products. All the physiotherapists we interviewed stated that the opportunity to come with feedback on the product was important. The reason for this was that they wanted the product to be customized for their and their patients need. \\ \\
One of our interview talked about how this game could fit into the Norwegian government’s new reform "Samhandlingsreformen". Of course this depends on whether the game is proved to be effective in health improving purposes and in cost. In "Samhandlingsreformen", from now on called "the reform", there is an increased focus on prevention and early intervention. Welfare technology shall support these strategies where possible. In Norway there is a goal to offer everyone in the need of it, a place in care homes . Every 75-year old are offered supervision to promote health and own coping. Every person who is in need of health care services and can deploy them in their own house, should be provided this. The services offered in the private home should be improved. If the services in the private home could be improved to the level where it helps the user being able to live in their own home for three more months, it will be equal to 10 percent of the capacity in care homes. This will depend on better offers outside the institutions. The main focus now should be "hverdagsrehabilitering", from now on called  "everyday rehabilitation" in english. This means that a person will get health care first after an evaluation that proves that he or she actually has a need for rehabilitation. The target group are the ones with moderate limitations in functional level. The goal is to postpone their need for extensive help and help them to a more dependent everyday life.  "Everyday rehabilitation" has the home care staff as a basis and physio - and ergotherapy as an "engine". If a patient is considered to being in the need of help and support in their own home, they will first meet with a physio -or ergotherapist. The new strategy for the health care system just described s based on the "Fredericia-model". This model is developed in the Danish municipal "Frederica" and is about how physical, social and cognitive capabilities can be maintained and improved, so that functional disabilities in the older generation will be postponed. The experiences from this model have been very positive. 30 percent receive "everyday rehabilitation" instead of ordinary home care. Approximately 45 percent of these end the rehabilitation process with no need of any further help, and 40 percent end the rehabilitation process with need of less help than one can assume they would need.\cite{budsjett} \\ \\
All of the interviewed physiotherapists were positive to Cyberlab’s exercise game. At the same time, neither of them could say anything about whether they would use it or not, and if they thought it would be good for elderly, before they had actually tried the game. The game has to ease the workload of the physiotherapist and also offer something equal or better than what they can offer themselves. The ability to customize the program was an aspect that was very important for all of the interviewees. To make it possible for elderly to use this kind of game it has to be easy to use and self-explanatory. In addition, the game should give some kind of feedback to the user with instructions on whether they are doing the exercise right or wrong. This can work both as a motivation factor for the participant and as a tool for the physiotherapist to keep track on how the patient is doing. The opportunity to not only customize in the sense of the right exercises, but also different themes, should make the game more fun, and more motivating. It is important to remember that elderly today are not that familiar with technology. This game will probably be more relevant when the next generation gets older, especially with regard to the use in private homes. Today, we can assume that the older patient would not go out and buy this game themselves, but rather use it after a recommendation from their physiotherapist. An other thing that was pointed out was were the game could be used. The different clinics may not have space to have for instance four people playing a video game in one room. Most likely they will have to be in their own room, so they will take the game seriously and also, so they do not disturb other patients. This must be up to each clinic to find a way to work around. An issue implementing this game in a private clinic, might be that they do not have that big customer base with older people. This is because in an private clinic the customer has to pay for the treatment. At the same time, it could be an interesting tool for them to use to tempt customer to choose them as a clinic. If Cyberlab was to run a pilot project, the difference with running it in the private -and the public clinic is that the private clinic would have to get paid to do the job. With this reasoning, it will be easier and more cost efficient to run this kind of project with the public clinics. \\ \\
From our interviews, we can conclude that physiotherapists are the right customer segment for Cyberlab. Today, it would not be relevant to go straight to the end user, because they are not familiar with this kind of technology. A physiotherapist is a professional who the patient can trust. If a professional say something, we are likely to believe in it. From the interviews conducted, we also found that the best way to get the physiotherapists attention will be to run a pilot project, to test the product and show the game's effect. This should be started in the public sector. It is important that the game can be customized both in the favour of the physiotherapist providing it and the patient using it. The game has to be proved improving the patients physical health as well as their mental health. And at last, it should be fun, motivating and easy to use. 

- forandre vanskelighetsgra osv..
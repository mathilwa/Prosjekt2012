\chapter{Health Politics in Norway Today}
The government states that it is a public responsibility to promote health and prevent diseases to make sure that the population gets the care they need. The goal is to overall get a healthier population. Good health is necessary for an individual to acquire the quality of life they wish for. It is also important for the society, especially economically. There is a huge amount of elderly ahead of us, an the society has to prepare for that. In Norway there is a goal to offer everyone in the need of it, a place in care homes by 2015.  To be able to meet all the requirements set, there is a need for a change in the health sector in Norway.\\ \\
In this chapter we will describe some of the goals for the future of the Norwegian health sector. This is done for us to be able to see the potential of this game within the health care system.  It is also necessary to look at different offers existing today, which will help us understand how and where Cyberlab's exergame can be used. This will be provided in the last section. 

\section{Samhandlingsreformen}
"Samhandlingsreformen" \cite{budsjett}\cite{regjering}, from now called "the reform" presents a new way to organize the health services in Norway and was put in action 01.01.2012. In the reform there is an increased focus on prevention, early intervention and close collaboration between different entities. The health services should be offered closer to where people live and there should be offered more comprehended and coordinated treatment. Welfare technology shall support these strategies where possible (described in the next section). Technical solutions and methods can make it possible to treat more patients in a better way. \\ \\
Every 75-year old are offered supervision to promote health and own coping. Every person who is in need of health care services that can be deployed in their own house should be provided this. The services offered in the private home needs to be improved. If the services in the private home could be improved to a level where it enables the user to live in their own home for three more months, it will be equal to 10 percent of the capacity in care homes. This will depend on better offers outside the clinics.\\ \\ The main focus in the health promoting work will be on "hverdagsrehabilitering", from now on called  "everyday rehabilitation". This means that a person will get health care first after an evaluation that proves that he or she actually has a need for rehabilitation. The target group is the ones with moderate limitations in functional level. The goal is to postpone their need for extensive help and help them to achieve a dependent everyday life.  "Everyday rehabilitation" has the home care staff as a basis and physiotherapy and ergotherapy as "engine". If a patient is considered to be in the need of help and support in their own home, they will first meet with a physiotherapist or ergotherapist, who will examine them and evaluate their need. The new strategy for the health care system just described is based on the "Fredericia-model". This model is developed in the Danish municipal "Frederica" and is about how physical, social and cognitive capabilities can be maintained and improved, so that functional disabilities in the older generation will be postponed. The experiences from this model have been very positive. 30 percent receive "everyday rehabilitation" instead of ordinary home care. Approximately 45 percent of these ended the rehabilitation process with no need of any further help, and 40 percent ended the rehabilitation process with need of less help than one can assume they normally would need \cite{budsjett}\cite{regjering}.
\section{Welfare Technology}
As a part of the new reform, welfare technology should be implemented if possible. Welfare Technology can be defined as: "Technological assistance that contributes to increased safety, social interaction, mobility, and physical and cultural activity. In addition welfare technology can help to strengthen an individual's ability to be independent despite sickness and social, mental or physical disabilities. Also it can work as technological support for relatives, as well as contribute to improving the services offered, when it comes to utilization of resources, availability and quality. In many cases welfare technology can prevent the need for health services or admission to an institution" (Translated from Norwegian from \cite{welfare}).\\ \\
Today there is huge attention around the topic of welfare technology in Norway. It is seen as an important tool in the future demographic challenges and in the health promoting work.  The goal is that the need for health services should be decreased and that people can take care of themselves longer. This means that there is a need for services that can be implemented in people's home. An example can be a technological tool to prevent fall and loneliness.  The use of welfare technology can improve the services offered, increase the flexibility and make it easier to interact with different actors. This introduces a new arena for innovation and value creation, and can give rise to positive socio-economic effects. \\ \\
The market for welfare technology is very immature. There are no initiators that can make sure it will be established public and private demand. It will be the different municipals' work to establish a public demand. There is a need for robust solutions when it comes to customization, entry level, support and maintenance. To find good solutions, there is a need for more developed products that can be tested.\cite{welfare} \\ \\
The Public Health Department in Norway, "Helsedirektoratet", recommends that laws for the use of welfare technology should be established. There are some new aspects that have to be taken into account when integrating technology into systems containing detailed personal information. The use of technology introduces some privacy issues, where for instance some systems will contain  sensitive information about the patients. This suggest that there should be a focus on making secure systems that will preserve people’s privacy.\cite{welfare} \\ \\
The new reform together with the increased focus on the use of welfare technology suggests that an exergame can serve as a tool in the health promoting work. We will now look closer into where the game could be implemented.

\section{Fall Prevention and Rehabilitation Today}
As one of the first steps towards finding where this game will fit, we see physiotherapy as an interesting area. The reasons for this comes from what we have learned from the previous sections. We see that physiotherapists are one of the the main actors in the area of prevention and rehabilitation. They are one of the first actors elderly will meet with when they are facing a physical problem. This suggest that physiotherapy might be the right area for the exergame. Therefore, we will look a little deeper into the subject of physiotherapy. \\ \\
Physiotherapy is a science with focus on body, movement and functionality to maintain, recover and improve physical health. The theoretical basis is grounded on knowledge about anatomy, neuroscience, and physiology. In addition to this, practical and clinical knowledge contribute to evaluation of how injuries and pain can be treated and prevented. Physiotherapists are traditionally working in municipalities, hospitals and in private institutions, and they are working with both individual treatment and treatment in groups. The goal for physiotherapists is to make their patients' daily activities easier to manage, which is done by using manual techniques, exercises and technical methods \cite{physiotherapy1}\cite{physiotherapy2}. “A physical therapist seeks to identify and maximize quality of life and movement potential through prevention, intervention (treatment), promotion, habilitation, and rehabilitation“.\\ \\
Elderly people consult physiotherapists in terms of rehabilitation after surgery, after a fall, stroke or other injuries, or when they feel health problems make it hard to perform everyday tasks. Physical therapy usually includes exercise with focus on increasing the patients’ flexibility, endurance and strength. Physiotherapists set up customized training for each patient according to what kind of needs they have. Unfortunately, the time a patient spends with the physiotherapists per week is not sufficient in order to become stronger or to recover from injuries. Therefore, there is a need for patients to perform additional training outside these hours. The physiotherapists may give their patients exercises they can practice at home. However, not everyone is motivated to exercise on their own and many skip the weekly exercise that is scheduled for them \cite{physiotherapy2}. 


\section{Existing Prevention Programs}
To prevent developing disabilities elderly should regularly perform a training program that strengthen their muscles, improve balance, coordination, endurance and mobility \cite{gruppetrening-trheim}. For convenience we tried to find out if these training programs are offered in Trondheim. We found that there have been established various fitness groups to become physically stronger and to achieve better balance for those who find it difficult to go outside. These fitness groups find place at various locations around Trondheim and are offered 1-2 hours one day a week. In addition there exist senior dance, walking groups and water gymnastics \cite{trim}. These activities are good initiatives, but when one problem is that elderly are afraid to go outside, how will they manage to engage in these fitness groups? It is also shown that only once a week with physical activity is not nearly enough to increase physical strength \cite{gruppetrening-trheim}. Regular physical activity is the key to become physically stronger and obtain better balance. \\ \\
Trondheim municipality did a study where they provided a once a week group training program for elderly. Their study showed that training once a week did not improve physical function for the participants, but the participants expressed that they were less afraid of falling after starting with the group training. The study suggests that this kind of program should be combined with home training programs or other extra physical training offerings \cite{gruppetrening-trheim}. \\ \\
We found that there are already some offered training programs for elderly that can be implemented in their home:\\ \\
The \emph{"Otago"-program} is a program developed as a home training program for elderly to prevent falls. It consists of exercises that take about 30 minutes to complete which should be performed three times a week in addition to a walk twice a week. Each customer receives a booklet with instructions for the individual exercises prescribed in addition to ankle cuff weights. The participants needs to record the days they complete the program for follow-up purposes. For follow-up an instructor should do home visits every six months and telephone them every month. The instructor can then increase the difficulty in the prescribed exercises for each individual. The program has been tested and evaluated for 1016 home living people aged 65 to 97. The program was shown to reduce falls and fall related injuries with 35 percent, with the highest effect on those over 80 years old and those that have had a previous fall. The participants experienced improved strength and balance, as well as they maintained their confidence so it was easier for them to do everyday activities without being afraid of falling \cite{otago} \cite{gruppetrening-trheim}.\\ \\
\emph{\ac{fame}} is an exercise program consisting of tailored group and home-based exercises and builds on the core exercises from the "Otago"-program.  There are a total of three group training sessions per week, in addition to two home-training sessions. The exercise intervention is designed to improve participants dynamic balance and core and leg strength.  In the United Kingdom a study was done where they examined the effectiveness of this program for home-living women aged 65 or older who had already fallen 3 or more times within the previous year. After using FaME for 36 weeks the fall rate was reduced by one-third. The conclusion was that the exercise program should last for at least 36 weeks including at least 2 hours of training per week. For progression it is important that the intensity, resistance, and weight are continually increased \cite{fame}.\\ \\
\emph{{Ø}velsesbanken} is a Scandinavian project providing a user profile with different training programs. The different exercises are developed from the two previous described concepts and other relevant studies on balance and exercising for elderly. The program gives an idea on how you can put together an exercise program customized for each individual. It is primarily made as a tool for physiotherapists for putting together training programs for their patients to do at home. How we see it, it can also be used as a tool for each individual to make their own program, because you can also log in as a private user and make your own program. The program offers the user a choice of different exercises that can be added to a exercise program. When all exercises are chosen PDF-files can be printed with pictures and descriptions of the exercises, or the user can read them from the computer screen. It is an easy, self-explanatory and straightforward program to use. {Ø}velsesbanken is in use in Scandinavia and the summer of 2012 it had reached 4300 users. \cite{ovelsesbank}\\ \\
We have learned from this chapter that there already exist projects and training programs with focus on elderly and their physical health. It is shown that exercise only once a week is not sufficient to improve physical health. However, with a supplement, like additional home training programs, it has been seen improvements. One serious challenge is to motivate the elderly to exercise. We believe "boring" exercises on a piece of paper will not be a motivating factor. The issue about the training groups offered today is that people are afraid of leaving their house, and therefore may not attend the weekly meetings. Still, we see a potential for the exergame to be implemented in this kind of training groups. Maybe the introduction of a new and alternative training method would encourage the older population to attend these sessions? Offering this kind of sessions several days a week, could improve the participants' health and thus reduce their fair of leaving their house.  Another arena for the exergame could be in peoples home. We believe this will be a more motivating way of exercising to perform on their own, as well as it will ease the workload of the instructor who can remotely monitor their patient instead of physically visit them. However, we are sceptical to the introduction of a tool like this in peoples home, since this kind of technology is unfamiliar to many people in this age group. This suggest that a starting point for this game could be in the health sector or in voluntary training groups.  The new focus on prevention and early intervention with use of welfare technology, suggest that there is a potential for an exergame in this market. 


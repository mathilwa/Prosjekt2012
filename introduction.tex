\chapter{Introduction}
\section{Scope}
\section{Problem Definition}
Games are becoming more and more important in the health sector. One specific genre of games is exercise games (exergames). In this assignment we will focus on the business opportunities and economics of games in the health sector, with special focus on exergames for rehabilitation and elderly. The work has been done in collaboration with Cyberlab, a company focusing on serious games for education and training. The work will provide input to an EU funded project focusing on elderly afraid of falling.

In particular, the following studies will be done: 
\begin{itemize}
\renewcommand{\labelitemi}{$\bullet$}
\item A background study of exergames in general and exergames for rehabilitation and elderly in particular.
\item	A description of specific business cases.
\item	Analyzing the business potential of this type of games in the health sector using Osterwalder business model.
\item	Miscellaneous aspects to consider about this type of game
\end{itemize}

\section{About the Project}
This project assignment is based on an EU-funded project called "GameUp". The purpose of "GameUp" is to use technologies that are proved to improve motivation to encourage elderly to be more physically active. The goal is to sustain and enhance mobility in older persons so that they can live longer at home which will result in a better quality of life. To achieve this they want to increase physical activity capabilities in order to increase motivation and self-efficacy towards mobility. For making it convenient and easy to use, they will develop a platform for social and exercise games using low cost motion sensors and commercial modules and products \cite{gameup}.\\ \\ This project is a European cooperation with different partners involved. On of these partners is Cyberlab, a small company located in Trondheim. They are working with developing simulators and simulation-based games primarily for technical education and training, but also for promotion and exemplification of technical products and services (nesten avskrift fra deres hjemmeside, \cite{cyberlab}. Their work in the "GameUp" project is to develop a exercise game using Microsoft Kinect as a platform. (Tor-Ivar).

\section{Limitation of Scope}

\section{Related Work}

\section{Outline}
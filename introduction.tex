\chapter{Introduction}
The background and motivation for this project assignment is an EU-funded project, called "GameUp". The purpose of "GameUp" is to use technologies that are proved to improve motivation to encourage elderly to be more physical active to maintain mobility. Sustaining and enhancing mobility in older people will reduce the risk of falling and enable them to live longer at home, which ultimately may result in a better quality of life. As a part of this, they will develop convenient and easy to use exercise games and social games using low cost motion sensors and commercial modules and products \cite{gameup}.\\ \\ The project is an European cooperation with different partners involved. Cyberlab is one of them. This is a company in Trondheim working with development of simulators and simulation-based games primarily for technical education and training, but also for promotion and exemplification of technical products and services \cite{cyberlab}. Their responsibility in the "GameUp" project is to develop the necessary exergames using Microsoft Kinect as a sensor and input device. In addtion, Cyberlab is responsible for the exploitation tasks of the project.  \\ \\

\section{Objectives}
Fall is a very common event for elderly and can have serious consequences. The fair of fall can make participating in regular training sessions outside their home a challenge for elderly. The lack of physical activity can result in immobility and loneliness. As a part of the “GameUp” project Cyberlab will develop a Microsoft Kinect based exergame for elderly.  In this assignment we will do a background study on:\\
1. the problems of the fall to understand the motivation for the exergame\\
2. how this is solved today and how the game can fit into the Norwegian health care system\\
3. what exergames are and what kind of technologies can be used to develop a game for exercising\\
4. previous research on the use of exergame\\
5. Cyberlab’s choice of technology is a reasonable choice\\ \\
Based on the background study, we will make an analysis of the game’s business potential. In order to do this, we chose Alexander Business Model Ontology as a framework. We will: \\
6. Provide a theoretical summary of Alexander Osterwalder’s Business Model Ontology\\
7. Discuss the methods used for information gathering \\
8. Discuss the information gathered \\
9. Propose a business model for the exergame, with a thorough financial analysis \\
10. Discuss aspects not included in the business model \\
11. Conclude with a recommendation for Cyberlab

\section{Contribution}
Our contribution in this project is a thorough analysis of the business opportunities of Cyberlab’s exergame. As a part of this we will provide a financial analysis of the project. This will be found in chapter 7. Our results will be based on previous work, our own opinions and interviews conducted by ourselves. A discussion on our interviews will be found in chapter 6, and a complete report is provided in appendix A-D. This project is meant to serve as a guideline for Cyberlab, and will hopefully provide them with valuable work.  

\section{Scope and Limitation}
In this project we have focused on the Norwegian market with its conditions, with Trondheim as our main base. We have studied training possibilities for elderly in Trondheim, and we have only interviewed physiotherapists from Trondheim. The GameUp-project is an EU-funded project, meaning that the product Cyberlab is going to develop could be sold all over Europe.  Selling this product in other countries might be very different, and will require a different business model. Since we have based our assumptions and conclusions upon information gathered from a limited amount of interviews conducted in Trondheim, this might have limited our data and made it incomplete. One important limitation to keep in mind is is that the game is not yet developed. This limits our ability to understand the game properly, and it has also limited the ability of our interviewees to provide supplementary answers about the interest for an exergame.  The market we have observed is very immature when it comes to this type technology. With lack of information about the market and the demand for a game like this, it is very difficult to predict a realistic outcome for Cyberlab’s product.

\section{Outline}
The project report is structured in the following way:
\begin{itemize}
\renewcommand{\labelitemi}{$\bullet$}
\item \textbf{Chapter 2} presents the motivation for the exergame, a brief the description of the game that Cyberlab is going to develop and an example case of an arena where the game can be used.
\item In \textbf{Chapter 3} we will look into the Norwegian health care system and its goal for the future. We will also look into what kind of exercise programs exists today.
\item	\textbf{Chapter 4} will describe the technical aspects of an exergame. We will describe what video games, and in particular exergames, are, and provide a description of the different exergame technologies. We will also look into previous research on video games used for exercise and rehabilitation.
\item	\textbf{Chapter 5} provides a summary of Alexander Osterwalder's Business Model Ontology
\item	In \textbf{Chapter 6} we will describe the Qualitative Research and, in particular the use of qualitative interviews. Then we will provide a discussion of the interviews we conducted.
\item In \textbf{Chapter 7} we will use the business model described in chapter 5 to analyse Cyberlab's exergame.
\item  \textbf{Chapter 8} discuss the business model with a critical view.
\item Finally, in \textbf{Chapter 9} we provide a conclusion on the work done and suggest future work.
\end{itemize}
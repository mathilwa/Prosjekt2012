\chapter{Introduction}
The background and motivation for this project assignment is an EU-funded project, called "GameUp". The purpose of "GameUp" is to use technologies that are proved to improve motivation to encourage elderly to be more physically active. The goal is to sustain and enhance mobility in older people so they can live longer at home which may result in a better quality of life. As a part of this, they will develop convenient and easy to use exercise games, from now on called exergames, and social games using low cost motion sensors and commercial modules and products \cite{gameup}.\\ \\ The project is an European cooperation with different partners involved. Cyberlab is one of them. This is a company in Trondheim working with development of simulators and simulation-based games primarily for technical education and training, but also for promotion and exemplification of technical products and services \cite{cyberlab}. Their responsibility in the GameUp project is to develop the necessary exergames using Microsoft Kinect as a sensor and input device. The games will be made with special focus on elderly who is afraid of falling. In addtion, Cyberlab is responsible for the exploitation tasks of the project.  \\ \\

\section{Contribution}
Our contribution in this project is a thorough analysis of the business opportunities of Cyberlab’s exergame. As a part of this we will provide a financial analysis of the project. Our results will be based on previous work, our own opinions and interviews conducted by ourselves. This project is meant to serve as a guideline for Cyberlab, and will hopefully provide them with valuable work.  

\section{Scope and Limitation}
In this project we have focused on the Norwegian market with its conditions, with Trondheim as our main base. We have studied training possibilities for elderly in Trondheim, and we have only interviewed physiotherapists from Trondheim. The GameUp-project is an EU-funded project, meaning that the product Cyberlab is going to develop could be sold all over Europe.  Selling this product in other countries might be very different, and will require a different business model. Since we have based our assumptions and conclusions upon information gathered from a limited amount of interviews conducted in Trondheim, this might have limited our data and made it incomplete. One important limitation to keep in mind is is that the game is not yet developed. This limits our ability to understand the game properly, and it has also limited the ability of our interviewees to provide supplementary answers about the interest for an exergame.  The market we have observed is very immature when it comes to this type technology. With lack of information about the market and the demand for a game like this, it is very difficult to predict a realistic outcome for Cyberlab’s product.

\section{Outline}
Hvordan kapittelene henger sammen..
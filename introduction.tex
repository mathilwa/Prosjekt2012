\chapter{Introduction}
The background and motivation for this project assignment is an EU-funded project, called "GameUp". A common problem for elderly is reduced physical strength and decreased balance. This problem might lead to low self-confidence in performing everyday activities, which again can lead to reduced mobility and high risk of falling. This is a vicious circle, resulting in reduced quality of life, inactivity and loneliness. Sustaining and enhancing mobility in older people will reduce the risk of falling and enable them to live longer at home, which ultimately may result in a better quality of life. The purpose of "GameUp" is to use technologies that are proved to improve motivation to encourage elderly to be more physical active to maintain mobility.  As a part of this, it will be developed convenient and easy to use exercise games and social games using low cost motion sensors and commercial modules and products \cite{gameup}.\\ \\ The project is an European cooperation with different partners involved, where Cyberlab is one of them. This is a company situated in Trondheim that are working with development of simulators and simulation-based games primarily for technical education and training, but also for promotion and exemplification of technical products and services \cite{cyberlab}. Their responsibility in the "GameUp" project is to develop the necessary exercise games using Microsoft Kinect as a sensor and input device. In addition, Cyberlab is responsible for the exploitation tasks of the project.  \\ \\
We will look at the exercise games Cyberlab is going to develop as one exercise game.  Exercise games are commonly called "exergames". Throughout this project assignment, we will refer to this game as "exergame" or just "game".  To be able to develop a business model for the exergame, we will do a background study on the following:
\begin{itemize}
\renewcommand{\labelitemi}{$\bullet$}
\item The problem of falls, to understand the motivation for the "GameUp" project and the development of an exergame. 
\item How the problem is solved today and how the game can fit into the Norwegian health care system.
\item What kind of technologies that can be used to develop a video game for exercise and rehabilitation.
\end{itemize}
Based on the background study, we will provide an analysis of the game’s business potential to see if it has economic potential. 

\section{Contribution}
Our contribution in this assignment is a thorough analysis of the business opportunities of Cyberlab’s exergame. We will describe the problem of falls to understand the motivation for the game, see Chapter 2.1, and we provide a study of the Norwegian health sector and existing solutions offered to discover where this game can fit, see Chapter 3. We will also provide a review of video game history and existing technologies related to exergames, which can be found in Chapter 4.1 and 4.2. Furthermore, we will include a summary of previous research that documents positive health effect by using video games as an exercise and/or rehabilitation tool. The summary can be found in Chapter 4.3. We have conducted interviews, which gave us valuable information about the exergame’s potential. A discussion of the interviews will be found in Chapter 7, and a complete interview report is provided in Appendix A-D. Our results will be based on previous work, our own opinions and interviews conducted by ourselves. The main focus of this assignment is the development of a business model for this project, and as a part of this we will provide a thorough financial analysis of Cyberlab’s cost and revenue streams. This is the most important contribution in this assignment. The financial analysis uncovers two potential revenue models for Cyberlab. A detailed description and discussion of both of them, including several examples, are provided in Chapters 8.4.2 and 8.4.3. This assignment is meant to serve as a guideline and recommendation for Cyberlab, and will hopefully provide them with valuable work.

\section{Scope and Limitations}
In this project we have focused on the Norwegian market with its conditions, and with Trondheim as our main base. We have studied training offerings for elderly in Trondheim, and we have only interviewed physiotherapists from Trondheim. The "GameUp" project is an EU-funded project, meaning that the product Cyberlab is going to develop will possibly be sold all over Europe. Selling this product in other countries might be different from selling in Norway, and will require a different business model. Due to lack of time, we have not taken the rest of Europe as a potential market into consideration, as it would have required too much research and work for this project assignment. It was also a request from Cyberlab to only focus on the Norwegian market. Since we have based our assumptions and conclusions upon information gathered from a limited amount of interviews conducted in Trondheim, this might have limited our data and made it incomplete. One important limitation to keep in mind is that the game is not yet developed. This limits our ability to understand the game properly, and it has also limited the ability for the interviewees to provide supplementary answers about the interest for an exergame. We did not have the time to perform more than three interviews. However, we feel the interviewees gave us sufficient information and feedback. The market we observed is very immature when it comes to using this video game technology as equipment used for therapy. With lack of information about the market and the demand for a game like this, it is very difficult to predict a realistic outcome for Cyberlab’s product.

\section{Outline}
The project report is structured in the following way:
\begin{itemize}
\renewcommand{\labelitemi}{$\bullet$}
\item \textbf{Chapter 2} presents the motivation for the exergame, a brief description of the game that Cyberlab is going to develop and an example case of an environment where the game can be used.
\item In \textbf{Chapter 3} we will look into the Norwegian health care system and its goal for the future. We will also look into what kind of exercise programs existing today.
\item	\textbf{Chapter 4} will describe the technical aspects of an exergame. We will describe what video games, and in particular exergames, are, and provide a description of the different exergame technologies. We will also look into previous research on video games used for exercise and rehabilitation.
\item	\textbf{Chapter 5} provides a summary of Alexander Osterwalder's Business Model Ontology
\item	In \textbf{Chapter 6} we will briefly describe qualitative research and in particular the use of qualitative interviews. 
\item	\textbf{Chapter 7} will provide a discussion of the interviews we conducted.
\item In \textbf{Chapter 8} we will use the business model described in Chapter 5 to analyse the business potential and economics of Cyberlab's exergame.
\item  \textbf{Chapter 9} discuss the business model with a critical view.
\item Finally, in \textbf{Chapter 10} we provide a conclusion on the work done and we suggest future work.
\end{itemize}
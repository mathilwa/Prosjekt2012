\chapter*{Appendix}


\section*{Intervju 22/10/12 kl. 1300}

\emph{-Intervjuobjekt:} \\ Jorunn Helbostad, utdannet ved Universitetet Bergen, hovedfag og doktorgrad i fysioterapi. Jobber som forsker. \\ \\
\emph{-Av deres pasienter fra 65 år og oppover, hva er problemet? Har dere pasienter som er hos dere for opptrening etter for eksempel en skade (rehabilitering)? Har dere pasienter som er hos dere for generell trening fordi de ønsker å styrke kroppen?}\\
Det finnes private og kommunale fysikalske klinikker. Disse har avtale med trygdesystemet. For å få time hos fysioterapeut må man ha rekvisisjon. Pasienten kommer  i kontakt med fysioterapeuter gjerne fordi de har et definert problem. Det hender for eksempel hjemmesykepleiere tar kontakt på vegne av en pasient de pleier. Det er sjeldent den eldre tar kontakt selv. \\ \\
\emph{-Hvordan er oppfølgingen under behandlingen? Hvordan er oppfølgingen etter behandlingen? Pleier dere å gi pasienten program som de må trene på hjemme mellom hver time?}\\ 
Ofte er det ikke nok å være hos fysioterapeuten 1-2 ganger i uka. Det er en utfordring å få de til å gjøre noe hjemme. De eldre ønsker gjerne å "bli friske", de er ikke veldig motiverte til å trene hjemme på egenhånd. Vanlig å oppfordre til å bevege seg mer hjemme, enten ved å gi en form for treningsprogram eller si noe som "husk å være fysisk aktiv". Dette kan være at fysioterapeuten skriver en lapp med strekmennesker som forklarer øvelser, eller et dataprogram hvor man kan sette sammen et program med øvelser og printe ut til pasienten, eller noen sier bare noe så generelt som at de må ut å bevege seg.\\ \\ \emph{-Opplever dere at det er pasienter som har problemer med å komme seg til behandling? Hender det dere må dra på hjemmebesøk?}\\ 
Nei, de som er dårlige til beins, får dekket drosjetransport til behandlingen. Men det er klart at mange vegrer seg for å gå ut. Det er også mange som vegrer seg for å bevege seg innendørs.\\ \\
\emph{-Er det noen som uttrykker at de ønsker hyppigere trening?}\\ 
Sjedent. De vil vel gjerne få en slags "pille/medisin" og bare bli frisk og rask.\\ \\
\emph{-Er det mange som uttrykker ensomhet/ulykkelighet?}\\ 
Det er veldig få som identifiserer seg selv som en person som er redd for å falle. I prosjekter hvor det har blitt foreslått forskjellige tiltak blir man ofte møtt med svar som "Det høres ut som en fin ting. Gi det til noen som trenger det". Det er mange som ikke sier ifra at de har falt. Å forebygge noe som "ikke har skjedd" er vanskelig. Dersom man jobber med fallforebygging, bør man ikke nevne ordet "fall". Det bør fokuseres mer på positive ting, som å styrke kropp for å kunne leke med barnebarna, gå på kafe osv.\\ \\
Det er opprettet noen treningsgrupper i Trondheim som er ment å forebygge fall. Men de reklamerer ikke med dette. I stedet reklamerer de med feks: "Vil du greie mer enn før...?".\\ \\
\emph{-Hvordan får dere høre om nye behandlingsmetoder, hjelpemidler, vektøy osv?}\\ 
Vi har "Fysioterapauten" som er et tidsskrift for fysioterapauter. Her blir stilen holdt ganske ren. Ellers er det jo også artikler i blant annet aviser og magasiner. Man drar på kurs og konferanser, men da gjerne innenfor et bestemt fagområde. Et bra sted å lansere nye produkter er kanskje på konferansene eller i "Fysioterapauten". Det er ofte snakk om etterutdanningskurs, ikke så mye om nye produkter. \\ \\
\emph{-Hva er interessen for nye ting?} \\ 
Her er det snakk om en ganske konservativ gruppe. Man vil veldig gjerne ha en dokumentasjon på at det fungerer. Produktet bør ha en lav brukerterskel. Hvor lett er det å bruke? Det må lette arbeidsmengden eller forbedre arbeidet for at det skal være interessant å ta i bruk. Man må også finne ut hvem som skal betale dette. Helsesektoren? Kommunen? Man må gjerne "stå for produktet" og klare å få fram at det er verdt å betale for. Pris på produktet har nok mye å si! \\ \\
\emph{-Må nye produkter være godkjent for medisinsk bruk?}\\ 
Et spill som dette havner litt i en gråsone. Det finnes lover og regler, men jeg tror ikke man trenger medisinsk godkjenning for å ta i bruk dette spillet.\\ \\
\emph{-Hvordan foregår en kjøpsprosess hos dere?} \\ 
Det vanligste er nok at man kjøper for å eie selv. Det er interessant å leie eller prøve produktet en viss tid for å være sikker på at det er et godt kjøp. Ofte skjer det at man kjøper inn et produkt, men så blir det gjerne liggende fordi man ikke tar seg tid til å lære det. Her kreves det opplæring! Det som også gjerne skjer er at en ivrig person tar initiativ til å kjøpe et nytt produkt og lærer seg hvordan det skal brukes, for så å kanskje slutte. De gjenværende har ikke lært seg å bruke produktet, og så blir det liggende. \\ \\
\emph{-Hender det at dere kjøper inn produkter for så å selge dem videre til kundene deres?}\\ 
Fysioterapautene kunne jo kjøpt spillet og eventuelt en lisens med på kjøpet, men jeg tror kanskje at eldre vil vegre seg for å kjøpe. Hvis kommunen så på det her som noe bra, så kunne kommunen ha kjøpt inn og lånt ut til eldre. \\ \\ Det er alltid en utfordringen med ny teknologi - hvem skal betale? Prosjekter har strandet fordi man ikke blir enige om hvem som skal betale.\\ \\ På fylkeskommunalt nivå har man hjelpemiddelsentralen. Hjelpemidler som kan lette hverdagen til folk kan bli kjøpt inn av hjelpemiddelsentralen og leid ut videre. Jeg er ikke helt sikker på hva som er grensen mellom trening og "fungere bedre i hverdagen" \\ \\
\emph{-Hva slags forhold har dere til leverandørene deres?} \\ 
På avansert utstyr kan man kjøpe serviceavtale. Men det er gjerne ingen som kjøper fordi det er for dyrt. Det er behov for oppgraderinger og oppfølging. Man ønsker gjerne tilpassede programmer, det vil gi større lyst til å prøve/bruke produktet. Sånn sett foretrekker jeg å samarbeide med små bedrifter, for da kan det være lettere.\\ \\ 
\emph{-Hva tenker dere om å bruke det videospillet som vi har beskrevet som en alternativ og annerledes behandlingsmetode \\
-for generell trening?\\
-tilpasset rehabilitering?}\\ 
For å kunne si noe om dette, ville jeg sett og prøvd spillet. For at det skulle vært interessant måtte det kunne lette arbeidsdagen min som fysioterapeut og gi meg muligheten til å tilby bedre hjelp til pasientene. Dersom jeg ikke syns øvelsene er relevante, ville jeg ikke brukt spillet. Spillet må være bedre enn det jeg kan tilby selv og øvelsene må kunne tilpasses. Når jeg har en pasient vil jeg finne ut hva som er pasientens problem ved å undersøke pasienten. Ut i fra problemet jeg finner, vil jeg legge opp et program ut ifra hver enkelt pasient. Innhold og vanskelighetsgrad må være tilpasset behovet. \\ \\ 
\emph{-Hva slags verdi tror du er bevegelsesstyrkende videospill kunne gitt til en bruker?}\\ 
Det kan oppleves både som spennende og som en barriere for pasienten. Mange eldre opplever teknologi som en barriere. Spillet må fenge pasienten. Spillet bør ha mulighet for individuell tilpasning. For å redusere fall bør øvelsen inneholde balanse og styrke og det må være mulig å tilpasse vanskelighetsgrad slik at det kan bli vanskeligere. Med øvelser med fokus på styrke og balanse er det bevist at man kan redusere fall med 20-60 prosent. Det teknologi kan bidra til er å gjøre det mer underholdende og motiverende. Tilbakemelding er en viktig motivasjonsfaktor. Dersom man for eksempel får tilbakemelding på at øvelsen du gjorde tilsvarte at du var 10 år yngre, ville du sannsynligvis trene en ekstra gang dagen etter. De fleste pasienter ville ikke tatt i bruk et slikt produkt på egenhånd. Måtte fått det anbefalt av for eksempel fysioterapeut. For at fysioterapeuten skal kunne følge med på pasientens progresjon, må det være lett tilgjengelig for dem.\\ \\ 
\emph{-Generelt} \\ 
Dere bør sjekke ut hjelpemiddelsentralen på www.nav.no. Her kan dere lese litt om regelverk. Folketrygden dekker ikke sport- og fritidsutstyr. Dere må tenke på: hvis spillet skal brukes, hvordan får dere fysioterapautene til å si ja? Hvordan får dere fysioterapautene med på laget? Det kan sikker være en lur idé å snakke med både fysioterapauter og ledelse.\\ \\ Sånn til slutt så vil jeg si at jeg har tro på dette prosjektet!\\ \\
\emph{-Nye kontakter} \\ 
-Fysikalsk institusjon - fastlønnet stilling\\
-Høre med Sylvi Sand (72549553, sylvi.sand@trondheim.kommune.no), hun vet hvem av de private klinikkene som driver med eldre. Har ansvar for fagutvikling \\
-Pensjonistenes fellesorganisasjon - Hornemannsgården. Spørre spørsmål angående forebygging. Høre med ledelsen at det er ok, når det passer. Inger Olsen (73841703) - daglig leder


